\section{Introduction} \label{sec:1}

The need for more accuracy has led to ever more complex models in engineering and physics. As a consequence, direct numerical methods for solving partial differential equations, a powerful modeling tool, have become inefficient due to limitations in computational power and resources.

Over the past decade, reduced basis methods have emerged as a popular approach to reduce the complexity of large systems of partial differential equations \cite{Ito:1998up,Ito:1998ch,Ito:2001ev}. These methods construct a low dimensional subspace, the reduced space, that approximately represent the solution to the system of differential equations. The projection of the original system onto the reduced space then allows the exploration of the solution with a significantly reduced computational complexity \cite{hesthaven2015certified,quarteroni2015reduced}.

Hamiltonian systems are an important class of systems that appear in engineering and physics. In these systems, preserving system energy is essential in a correct numerical solutions. Therefore, developing model reduction techniques that preserves the symplectic symmetry is essential. The classical reduced basis methods, do not generally preserve the conservation laws and symmetries, e.g. the system energy, of the full system \cite{Amsallem:2014ef,prajna2003pod}. This often results in an unstable or qualitatively wrong solution. 

It is demonstrated by \cite{Lall:2003iy,Carlberg:2014ky,Peng:2014di}, that if the basis to the reduced space is not chosen carefully, the symplastic symmetry of Hamiltonian and Lagrangian systems will be destroyed over the course of model reduction. To resolve this issue Peng et al. \cite{Peng:2014di} and Lall et al. \cite{Lall:2003iy} construct a reduced ordered configuration space that inherits symmetries of the full configuration space. Then, by using a proper time integrator scheme one can preserve symmetries.

Most models in engineering, appear as a dissipative perturbation of a Hamiltonian system. In these systems, conservation of energy is taken as a fundamental principle of the system dynamics, while deriving and dissipative forces, e.g. friction, can change the energy level of the system \cite{vanderSchaft:2014:PST:2693645.2693646}. As the system energy is no longer preserved for such systems, the methods mentioned above can no longer be applied directly \cite{peng2016geometric}.

For dissipative and forced Hamiltonian system, Peng et al. \cite{peng2016geometric} suggest a symplectic model reduction method that preserves Hamiltonian and the dissipative structure of the original system. However, since this method uses a symplectic integrator for a non-conservative system, there is no guarantee that the evolution of the energy is translated correctly to the reduced system.

In the context of network modeling and circuit simulation, a considerable attention is geared toward developing structure preserving, and in particular energy preserving, model reduction techniques. Model reduction of port-Hamiltonian system is given \cite{Polyuga:2010gj,beattie2011structure,chaturantabut2016structure} and the references therein. These methods construct use a Krylov or a Proper Orthogonal Decomposition (POD) method to construct a reduced port-Hamiltonian system that preserves the passivity, and thus the stability of the original system. In general these methods do not guarantee the correct distribution of the energy among the energy consuming and energy storing units. Further, over long time integration, accumulation of local errors might produce an erroneous solution.

In this paper we propose the Reduced Dissipative Hamiltonian Method (RDH), a structure preserving model reduction approach for dissipative Hamiltonian systems. By considering a canonical heat bath, also known as the hidden strings \cite{Figotin:2006jy,Figotin:2005}, the reduced system is extended to a closed and conservative system. Therefore, a symplectic time integrator can guarantee the conservation of the system energy and the dissipated energy. Further, the hidden strings assure that the local errors in the dissipation of energy do not accumulate, resulting in a correct evolution of the system energy.

This paper is organized as follow. Section \ref{sec:2} covers the required topics on Hamiltonian systems, dissipative Hamiltonian system and Hamiltonian extension. In Section \ref{sec:3} we discussion the symplectic model reduction for Hamiltonian systems and introduce the Reduced Dissipative Hamiltonian Method. Accuracy, stability, and efficiency of the RDH method is then discussed in Section \ref{sec:4}, through simulation of the dissipative wave equation and a linear port-Hamiltonian system of an electrical circuit. We offer conclusive remarks in Section \ref{sec:5}.
