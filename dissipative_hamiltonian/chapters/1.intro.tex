\section{Introduction} \label{sec:1}

The need for increased accuracy has led to more complex models and the use of large systems of partial differential equations in engineering and science. As a consequence, direct numerical methods for solving PDEs have become computationally demanding and, at times impractical.

During the past decade, reduced basis methods have emerged as a powerful approach to reduce the cost of evaluating large systems of partial differential equations \cite{Ito:1998up,Ito:1998ch,Ito:2001ev}. These methods construct a low-dimensional linear subspace, the reduced space, that approximately represents the solution to the system of differential equations. The projection of the original system onto the reduced space then allows the exploration of the solution with a significantly reduced computational complexity \cite{hesthaven2015certified,quarteroni2015reduced}.

Hamiltonian systems are an important class of systems that appear in engineering and science. In these systems, preserving system energy is essential to obtain a correct numerical solution. Therefore, the development of model reduction techniques that preserve the symplectic symmetry is crucial. However, the classical reduced basis methods do not generally preserve the conservation laws and intrinsic symmetries of Hamiltonian systems \cite{Amsallem:2014ef,prajna2003pod}. This often results in an unstable or a qualitatively wrong solution. 

It is demonstrated in \cite{Maboudi:2016,Lall:2003iy,Carlberg:2014ky,Peng:2014di}, that if the basis for the reduced space is not chosen carefully, the symplectic symmetry of Hamiltonian and Lagrangian systems will be destroyed by model reduction. To resolve this issue, in \cite{Maboudi:2016,Peng:2014di,Lall:2003iy} a reduced order configuration space is constructed that inherits symmetries of the full configuration space. By using a proper time integrator scheme, the symmetries are preserved in the reduced system. A greedy-type algorithem is developed in \cite{Maboudi:2016} for construction of a basis for such a reduced configuration space.

Most models in engineering appear as a dissipative perturbation of a Hamiltonian system. In these systems, conservation of energy is taken as a fundamental principle of the system dynamics, while dissipative forces, e.g. friction, can change the energy of the system \cite{vanderSchaft:2014:PST:2693645.2693646}. As the energy is no longer preserved for such systems, existing methods can no longer be applied directly \cite{peng2016geometric}.

For dissipative and forced Hamiltonian system, Peng et al. \cite{peng2016geometric} suggest a symplectic model reduction method that preserves the Hamiltonian and the dissipative structure of the original system. However, since this method uses a symplectic integrator for a non-conservative system, there is no guarantee that the evolution of the energy is translated correctly to the reduced system.

In the context of network modeling and circuit simulation, considerable work has been done in the development of structure preserving, and in particular energy preserving, model reduction techniques. Model reduction for port-Hamiltonian systems are given in \cite{Polyuga:2010gj,beattie2011structure,chaturantabut2016structure} and the references therein. These methods use a Krylov or a Proper Orthogonal Decomposition (POD) approach to construct a reduced port-Hamiltonian system that preserves the passivity, and, thus, the stability of the original system. However, these methods do not generally guarantee the correct distribution of the energy among the energy consuming and energy storing units. Furthermore, over long time integration, accumulation of local errors might produce an erroneous solution.

In this paper, we present the Reduced Dissipative Hamiltonian (RDH) method as a structure preserving model reduction approach for dissipative Hamiltonian systems. A key difference between this method and the other existing methods is that the RDH enables the reduced system to be integrated using a symplectic integrator. By considering a canonical heat bath, also known as hidden strings \cite{Figotin:2006jy,Figotin:2005}, the reduced system is extended to a closed and conservative system. Therefore, a symplectic time integrator can be used to guarantee conservation of the system energy and the correct dissipation of energy. Furthermore, the hidden strings assure that the local errors in the dissipation of energy do not accumulate, resulting in a correct evolution of the system energy.  

What remains of this paper is organized as follow. Section \ref{sec:2} covers the required background on Hamiltonian systems, dissipative Hamiltonian systems, and the Hamiltonian extension. In Section \ref{sec:3} we discussion the symplectic model reduction for Hamiltonian systems and introduce the Reduced Dissipative Hamiltonian method. Accuracy, stability and efficiency of the RDH method is discussed in Section \ref{sec:4}, and illustrated through simulation of the dissipative wave equation and a linear port-Hamiltonian system of an electrical circuit. We offer conclusive remarks in Section \ref{sec:5}.
