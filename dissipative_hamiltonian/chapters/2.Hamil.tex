\section{Dissipative Hamiltonian Systems} \label{sec:2}
In this section we first discuss Hamiltonian systems. Add more later

\subsection{Hamiltonian Systems} Suppose that $(\mathbb{R}^{2n},\Omega)$ is a symplectic linear vector space, where $\mathbb{R}^{2n}$ is the configuration space and $\Omega:\mathbb{R}^{2n}\times \mathbb{R}^{2n} \to \mathbb R$ is a closed, skew-symmetric and non-degenerate 2-form on $\mathbb{R}^{2n}$. Given a smooth Hamiltonian function $H:\mathbb{R}^{2n}\to \mathbb R$, \emph{Hamilton}'s equations of evolution are given as
\begin{equation} \label{eq:2.1}
	\dot {\mathbf x} = \mathbb J_{2n} \nabla_{\mathbf x} H,
\end{equation}
where $\mathbf x\in\mathbb R^{2n}$ is the configuration coordinates and $\mathbb J$ is a $2n\times 2n$ matrix such that $\Omega(\mathbf x,\mathbf y) = \mathbf x^T \mathbb J_{2n} \mathbf y$ \cite{Marsden:2010:IMS:1965128}. One by using the \emph{Symplectic Gram-Schmidt} \cite{de2006symplectic} can construct a coordinate system in which $\mathbb J_{2n}$ takes the form
\begin{equation} \label{eq:2.2}
	\mathbb{J}_{2n} = 
	\begin{pmatrix}
		0_n & I_n \\
		-I_n & 0_n
	\end{pmatrix},
\end{equation}
where $I_n$ and $0_n$ are the identity matrix and the zero matrix of size $n$ respectively. A main feature of Hamiltonian systems is the conservation of the Hamiltonian along the integral curves.
\begin{theorem} \label{theorem:2.1}
\cite{Marsden:2010:IMS:1965128} Consider the flow $\phi_t:\mathbb R\times \mathbb R^{2n} \to \mathbb R^{2n}$ of the Hamiltonian system (\ref{eq:2.1}). Then $H\circ \phi_t = H$.
\end{theorem}

In many physical problems $H$ represents the system energy and is bounded below. Here, we mostly assume that $H$ takes the form $H(\mathbf x) = \frac 1 2 \mathbf x^T K^T K \mathbf x$, where $K$ is a full rank $2n\times 2n$ matrix. This assumption leads to a linear system of evolution (\ref{eq:2.1}). Note that the Hamiltonian extension in section \ref{} and the model reduction for dissipative Hamiltonian systems in section \ref{} can be naturally extended to Hamiltonians of the form $H(\mathbf x) = \frac 1 2 \mathbf x^T K^T K \mathbf x + g(\mathbf x)$, where $g:\mathbb R^{2n} \to \mathbb R$ is arbitrary function of $\mathbf x$. 

It is natural to expect a numerical integrator that solves (\ref{eq:2.1}) to also satisfy the conservation law \ref{theorem:2.1}. Common numerical integrators, e.g. the Runge-Kutta methos, do not generally preserve the Hamiltonian. Symplectic numerical integrators are a class of numerical integrators for Hamiltonian systems that preserves the symplectic structure and ensure stability in long-time integration. The Str\:omer-Verlet time stepping scheme is an example of such numerical integrators and is given by

\begin{equation} \label{eq:2.2}
\begin{aligned}
	p_{n+1/2} &= p_n - \frac{\Delta t}{2} \nabla_qH(q_{n},p_{n+1/2}), \\
	q_{n+1} &= q_n + \frac{\Delta t}{2} \left( \nabla_pH(q_{n},p_{n+1/2}) + \nabla_pH(q_{n+1},p_{n+1/2}) \right),\\
	p_{n+1} &= p_{n+1/2} - \frac{\Delta t}{2} \nabla_qH(q_{n+1},p_{n+1/2}).
\end{aligned}
\end{equation}
More information on construction and application of symplectic integrators can be found in \cite{Hairer:1250576}.

\subsection{Dissipative Hamiltonian Systems and Hamiltonian Extensions}
