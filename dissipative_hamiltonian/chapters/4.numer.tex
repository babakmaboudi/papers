\section{Numerical Results}
In this section we illustrate the performance of the discussed metthod through numerical solution of the dissipative wave equation and a port-Hamiltonian model for a dissipative circuit.

\subsection{Dissipative wave equation}

Consider the dissipative linear wave equation
\begin{equation} \label{eq:4.1}
	\begin{aligned}
		q_{t}(t,x) &= p(t,x), \\
		p_{t}(t,x) &= c^2 q_{xx}(t,x) - r(x)  p(t,x) , \\
		q(0,x) &= q_0(x), \\
		p(0,x) &= 0.
	\end{aligned}
\end{equation}
where $x$ belongs to a one-dimensional torus of length $L$ and $r:[0,1]\to[0,1]$ is a positive semi-definite real valued function. 

We discretize the torus into $N_{\Delta x}$ equidistant points and define $\Delta x = L/N_{\Delta x}$, $x_i = i\Delta x$, $q_i=q(t,x_i)$ and $p_i=p(t,x_i)$ for $i = 1, \dots, N_{\Delta x}$. The discretization of $r$ corresponds to a diagonal and semi-positive definite matrix $r_\Delta$. Further, we discretize (\ref{eq:4.1}) using a standard centeral finite differences schemes to obtain
\begin{equation} \label{eq:4.2}
	\dot z = \mathbb J_{2n} K^T K z - R z,
\end{equation}
where $z = (q_1,\dots,q_{N_{\Delta x}},p_1,\dots,p_{N_{\Delta x}})$ and $K$ and $R$ are given as
\begin{equation} \label{eq:4.3}
	K^T K =
	\begin{pmatrix}
		I & 0 \\
		0 & c^2D_x^TD_x
	\end{pmatrix} , \quad
	R =
	\begin{pmatrix}
		0 & 0 \\
		0 & r_\Delta
	\end{pmatrix},
\end{equation}
with $D_x^TD_x = D_{xx}$ the central finite differences matrix operator. Writing (\ref{eq:4.2}) in a TDD formulation gives
\begin{equation} \label{eq:4.4}
\begin{aligned}
	& \dot z = \mathbb J_{2n} K^T f(t) \\
	& f(t) + R \int_0^t f(t) = K z.
\end{aligned}
\end{equation}
Here, since $R$ is not time dependent, it commutes with the integration operator. The Hamiltonian extension of (\ref{eq:4.4}), then takes the form (\ref{eq:2.10.a})-(\ref{eq:2.10.c}).

The initial condition used is given by
\begin{equation} \label{eq:4.5}
	q_i(0) = h( 10\times|x_i - \frac{1}{2}| ), \quad p_i = 0, \quad i=1,\dots,N
\end{equation}
where $h(s)$ is the cubic spline function
\begin{equation} \label{eq:4.6}
h(s) = 
\left\{
\begin{aligned}
& 1 - \frac{3}{2}s^2 + \frac{3}{4}s^3, \quad & 0\leq s \leq 1, \\
& \frac{1}{4}(2-s)^3, & 1< s \leq 2, \\
& 0, & s > 2.
\end{aligned}
\right.
\end{equation}

For the numerical time integration of the extended Hamiltonian system, the Str\"omer-Verlet time stepping scheme (\ref{eq:2.3}) is used. Further in each time step, the system of linear equations (\ref{eq:2.4.1}) is solved to recover $z$. System parameters are summarized below.
\vspace{0.5cm}
\begin{center}
\begin{tabular}{|l|l|}
\hline
Domain length & $L = 1$ \\
No. grid points & $N = 500$ \\
Space discretization size & $\Delta x = 0.002$ \\
Time discretization size & $\Delta t = 0.01$ \\
Wave speed & $c^2 = 0.1$ \\
\hline
\end{tabular}
\end{center}
\vspace{0.5cm}

The first numerical experiment, corresponds to an inhomogeneous dissipative media. Here, $r_{\Delta} = \text{diag}(r_1,\dots,r_{N_{\Delta x}})$, with $r_i = 0.1 + 0.9(i/N_{\Delta x})$, and diag is the MATLAB notation for constructing a diagonal matrix.

Figure \ref{fig:4.1}.a shows the solution of the original dissipative wave equation (\ref{eq:4.1}) at $t \in \{0,2.5,5,7.5\}$. For a nonzero $r_\Delta$ the solution will converge to $(q(t=\infty,x),p(t=\infty,x)) = (\rho,0)$ where $\rho$ is the center of mass of $q_0$. 

To construct the reduced basis for different methods, we gather the snapshots of (\ref{eq:4.1}). We construct RDH reduced system according to the Algorithm \ref{alg:3.1}. The cotangent lift is used to construct an ortho-symplectic reduced basis $A$ (\ref{eq:3.5}). The performance of the RDH method is then compared to the POD, and the method proposed in \cite{peng2016geometric} referred to as the PSD.

Figure \ref{fig:4.1}.b Shows the decay of the singular values of the snapshot matrix for the POD, PSD and the RDH methods. Note that the snapshots for the PSD and the RDH are different since they have different canonical representations. The fast decay of the eigenvalues in all methods is a strong indicator for the existence of a low dimensional reduced system. The reduced bases are then constructed using 20, 40 and 60 number of modes.

The $L^2$ error between the full systems and the RDH, the PSD and the POD method are presented in Figure \ref{fig:4.1}.c. We notice that symplectic methods provide a more accurate solution with respect to the POD method. In fact, the POD method does not yield a stable reduced system.  Further it is seen that enriching the PSD reduced basis does not yield a significant enhancement in the accuracy of the reduced system. This is because in the PSD method, a non-conservative system is being numerically integrated with a symplectic integrator. This results in an incorrect evolution of energy and eventually, in a qualitatively wrong numerical solution.

On the other hand, We notice that the RDH method with 40 modes provides a significantly more accurate solution compared to the PSD method with 60 modes. The RDH method provides a conservative reduced system where the dissipated energy is trapped in the hidden strings. The conservation of energy is then guaranteed using a symplectic integrator. Due to this reason, we observe remarkable increase in the accuracy by enriching the RDH reduced basis.

Figure \ref{fig:4.1}.d shows the preservation of the energy in different methods. The conservation law expressed in theorem \ref{theorem:2.1} is destroyed through the POD model reduction. That is why we observe the blow-up of the system energy. The symplectic methods however, preserves the energy significantly better. Due to the reason discussed above, enriching the PSD basis does not significantly affect the preservation of energy. On the contrary, The RDH not only provides a noteworthy accuracy in preserving the energy, but also provides significant improvement under enrichment.

In Figure \ref{fig:4.1}.e we show the transfer of the energy from the TDD system to the hidden strings, for the full system and the RDH reduced system. We notice that the RDH method preserves the total energy of the extended Hamiltonian system. Further, the transfer of the energy to the hidden strings in the full model is correctly translated in the reduced system.

\begin{figure}[t]
\begin{tabular}{cc}
\includegraphics[width=0.5\textwidth]{./figs/wave/solution} & 
\includegraphics[width=0.5\textwidth]{./figs/wave/singular} \\
(a) & (b) \\
\includegraphics[width=0.5\textwidth]{./figs/wave/error} & 
\includegraphics[width=0.5\textwidth]{./figs/wave/energy} \\
(c) & (d) \\
\multicolumn{2} {c} {\includegraphics[width=0.7\textwidth]{./figs/wave/energy_conserved}} \\
\multicolumn{2} {c} {(e)} 
\end{tabular}
\caption{(a) The solution to the original dissipative wave equation (\ref{eq:4.1}), (b) The decay of the singular values for the POD, the PSD and the RDH methods, (c) The $L^2$ error for the different methods, (d) Energy preservation for different methods, (e) Energy preservation of the Hamiltonian extension for the original and the reduced system. ``FM'' and ``RM'' stand for the full model and the reduced model respectively.} \label{fig:4.1}
\end{figure}

The second numerical experiment is the dissipative wave equation (\ref{eq:4.1}) in a near-zero dissipation regime. The purpose of this numerical experiment is to show that the RDH yields a stable solution in the elimination of the dissipation. Also to demonstrate that the PSD converges to the RDH as $\| \chi \|_{\infty}\to 0$. 

The numerical setting is taken to be identical to the previous numerical experiment, but with the difference that $r_i = 0.001$, for $i=1,\dots,N_{\Delta x}$. 

\subsection{Port-Hamiltonian Systems}
Port-Hamiltonian systems are popular in network modeling and electrical engineering. In the framework of port-Hamiltonian modelling, Energy conservation and Hamiltonian structure is the fundamental principle of the dynamics of the system. Ports in the system network then allows the exchange of energy with the environment in the form of input forces, capacitors and dissipations \cite{vanderSchaft:2014:PST:2693645.2693646}. Port-Hamiltonian systems can be viewed as a forced and dissipative Hamiltonian system.
\begin{figure}[t]
\begin{center}
	\includegraphics[width=0.7\textwidth]{./figs/porthamil/circuit}
\end{center}
\caption{$n$-dimensional ladder network} \label{fig:4.2}
\end{figure}

Consider the $n$-dimensional linear ladder network in Figure \ref{fig:4.2}. Here $C_i$, $L_i$ and $R_i$, $i=1,\dots,n$, are the capacitance, inductance and resistance of the corresponding capacitors, inductors and resistors respectively and $R_{n+1}$ is the load capacitor. The port-Hamiltonian model of the linear ladder network takes the form
\begin{equation} \label{eq:4.7}
		\dot x = (J_{2n} - R)Q^TQx + u.
\end{equation}
Here $x = (c_1,\phi_1,\dots,c_n,\phi_n)^T$ where $c_i$ and $\phi_i$, for $i=1,\dots,n$, are the charge and the flux of $C_i$ and $L_i$ respectively. $Q$ and $R$ are given as
\begin{equation}
	Q = \text{diag}(C_1^{-1},L_1^{-1},\dots,C_n^{-n},L_n^{-n}), \quad R = \text{diag}(0,R_1,\dots,0,R_n+R_{n+1}),
\end{equation}
$u=(1,0,\dots,0)^T$ is a constant input current and $J_{2n}$ is a skew-symmetric $2n\times 2n$ matrix with -1 and 1 on the superdiagonal and subdiagonal respectively. 

The energy associated to a port-Hamiltonian system of the form (\ref{eq:4.7}) at time $t$, is given as $H(x(t)) = \frac 1 2 x^T Q^T Q x$. Since $J_{2n}$ is skew symmetric it implies that $\frac d {dt} H(x) = u^T Q^T Q x - x^T Q^T Q R Q^T Q x \leq u^T Q^T Q x$ which is referred to as the \emph{passivity} of the system (\ref{eq:4.7}) \cite{vanderSchaft:1996es,Willems:1972ek}.

Since $J_{2n}$ is full ranked, one can always find a coordinate transformation $ x= T \tilde x$ such that $T^{-1} J_{2n} T^{-T} = \mathbb J_{2n}$. The dissipative Hamiltonian formulation for (\ref{eq:4.7}) takes the form
\begin{equation} \label{eq:4.8}
	\dot {\tilde x} = \mathbb J_{2n} \tilde Q^T\tilde Q \tilde x - \tilde Rx + \tilde u,
\end{equation}
where $\tilde Q = QT$, $\tilde R = T^{-1}RT^{-T}Q^TQ$ and $\tilde u = T^{-1} u$. Note that in this case, since $T$ is orthogonal and $R$ is diagonal, then $\tilde R$ is symmetric and semi-positive definite. The TDD formulation of (\ref{eq:4.8}) takes the form
\begin{equation} \label{eq:4.9}
	\dot{\tilde x} = \mathbb{J}_{2n} \tilde Q^T f(t) + \tilde u, \quad f(t) + \tilde R \int_0^t f(t) = \tilde Q \tilde x.
\end{equation}
The extended Hamiltonian formulation (\ref{eq:2.10.a})-(\ref{eq:2.10.c}) with a quadratic Hamiltonian $H_{\text{ex}}$ can be then carried out according to section \ref{??}. We note that due to the input $\tilde u$, the Hamiltonian $H_{\text{ex}}$ is time dependent. As the matter of fact $\frac{d}{dt} H_{\text{ex}} = \tilde u^T\tilde Q^T \tilde Q \tilde x$. If we define $\overset{\circ}{H}_{\text{ex}} : \mathbb R^{2n}\times \mathcal H^{2n}\times \mathbb R^{2}\to \mathbb R$ as
\begin{equation} \label{eq:4.10}
	\overset{\circ}{H}_{\text{ex}}(\tilde x,\phi,\theta,t,e) = H_{\text{ex}}(\tilde x,\phi,\theta,t) + e, \quad \dot e = - \partial_t H_{\text{ex}},
\end{equation}
it is easily checked that $\frac d {dt} \overset{\circ}{H}_{\text{ex}} =0$ \cite{Hairer:1250576}. However for the time integration of the Hamiltonian system related to $\overset{\circ}{H}_{\text{ex}}$, since the evolution of $\tilde x$, $\phi$ and $\theta$ does not explicitly depend on $e$, we can apply a symplectic integrator directly on (\ref{eq:4.10}). Thus, the passivity of (\ref{eq:4.7}) will be preserved through a symplectic time integration of (\ref{eq:4.9}).

Using an ortho-symplectic reduced basis $A$ The reduced dissipative Hamiltonian method can be applied to (\ref{eq:4.9}) to construct a reduced system of the form (\ref{eq:3.12.a})-(\ref{eq:3.12.c}) together with the extended Hamiltonian $\tilde H_{\text{ex}}$. We immediately notice that $\frac{d}{dt} \tilde H_{\text{ex}} = (A^+ \tilde u)^T A^T \tilde Q^T \tilde Q A y$, meaning that the reduced system is passive. Further the dissipative Hamiltonian structure of the reduced system indicates that the reduced system also carries a port-Hamiltonian structure.

We consider a 100-dimensional ($n=50$) port-Hamiltonian system for the ladder network discussed above. We take $C_i=1$, $L_i = 1$, $R_i=0.2$ for $i=1,\dots,50$, and $R_{51} = 0.4$. The cotangent lift is used to construct a reduced basis and the RDH method is used to construct a reduced system according to Algorithm (\ref{alg:3.1}). 

The solution of the RDH method is compared to the main results of \cite{Polyuga:2010gj}, where a passivity-preserving model reduction is developed using the moment matching method at infinity.

\begin{figure}[t]
\begin{tabular}{cc}
\includegraphics[width=0.5\textwidth]{./figs/porthamil/error_capac} & 
\includegraphics[width=0.5\textwidth]{./figs/porthamil/error_flux} \\
(a) capacitors & (b) inductors
\end{tabular}
\caption{Error between the full model and the reduced model obtained by the reduced dissipating Hamiltonian method ``RDH'' and the moment matching method ``MM''. (a) The average error of charge in capacitors in time. (b) the average error of the flux in inductors in time.} \label{fig:4.3}
\end{figure}
