\section{Model Order Reduction} \label{sec:3}
In this section we first explain the main results of \cite{Peng:2014di} regarding model reduction of Hamiltonian systems. In section \ref{sec:3.3} we introduce the Reduced Dissipative Hamiltonian method.

\subsection{Symplectic Model Order Reduction} \label{sec:3.1}
Consider a Hamiltonian system of the form (\ref{eq:2.1}) together with a quadratic Hamiltonian of the form $H(z) = \frac 1 2 z^T K^T K z$. In this paper we focus on reducing the complexity of numerical evaluation of (\ref{eq:2.1}) with respect to time $t$. Nevertheless, one can naturally extend the results of this paper to a Hamiltonian system that depends on a set of physical or geometrical parameters belonging to a compact subset of a Euclidean space.

The main idea behind model order reduction is that the manifold, $\mathcal M = \{ z(t) : t \in [0,T] \}$ can be approximated by a low dimensional symplectic linear subspace \cite{hesthaven2015certified,quarteroni2015reduced}. A basis for such a subspace is called a \emph{reduced basis}, and its span is referred to as the \emph{reduced space} \cite{hesthaven2015certified}.

Suppose that a reduced symplectic basis $A \in \mathbb R^{2n\times 2k}$ is provided with $k \ll n$. The approximated solution to (\ref{eq:2.1}) in this basis is given by
\begin{equation} \label{eq:3.1}
	z =Ay,
\end{equation}
where $y\in \mathbb R^{2k}$ is the coordinates of the approximation with respect to the basis $A$. Substituting (\ref{eq:3.1}) into (\ref{eq:2.1}) yields
\begin{equation} \label{eq:3.2}
	A \dot y = \mathbb{J}_{2n} \nabla_{z} H(A y).
\end{equation}
Multiplying both sides with the symplectic inverse of $A$ and using the chain rule we obtain
\begin{equation} \label{eq:3.3}
	\dot y = A^+ \mathbb{J}_{2n} (A^+)^T \nabla_{y} H(A y).
\end{equation}
As $(A^+)^T$ is a symplectic matrix it yields that $A^+ \mathbb{J}_{2n} (A^+)^T = \mathbb{J}_{2k}$. By defining the reduced Hamiltonian $\tilde H : \mathbb{R}^{2k} \to \mathbb R$, defined as $\tilde H (y) = H(Ay)$, we recover the reduced system
\begin{equation} \label{eq:3.3.1}
	\begin{aligned}
	\dot {y}(t) &= \mathbb J_{2k} \nabla_{y} \tilde H, \\
	y(0) &= A^+ z_0,
	\end{aligned}
\end{equation}
Equation (\ref{eq:3.3.1}) is called the \emph{symplectic Galerkin projection} \cite{Peng:2014di} of the Hamiltonian system (\ref{eq:2.1}). Conventional model reduction routines, e.g. Galerkin or Petrov-Galerkin methods \cite{hesthaven2015certified,quarteroni2015reduced}, do not yield a Hamiltonian reduced system. Hence, the reduced system does not necessarily preserve the conservation law in theorem \ref{theorem:2.1}. This results in a qualitatively wrong and often unstable solution \cite{Peng:2014di}. On the other hand the reduced system obtained by the symplectic Galerkin projection, is a Hamiltonian system, therefore the system energy is preserved over time integration \cite{Peng:2014di}. The following theorem from \cite{Peng:2014di} guarantees the boundedness of the solution of the reduced system obtained by the symplectic Galerkin projection.

\begin{theorem}
\cite{bhatia2002stability} Let $S$ be a bounded open subset of $\mathbb R^{2n}$ that contain $z_0$. Further assume that $H(z_0)<H(z)$ or $H(z_0)>H(z)$ for all $z\in \partial S$ the boundary of $S$. For a given symplectic basis $A$, if $z_0$ is in the range of $A$, then both the original system and the reduced system obtained by the symplectic Galerkin projection are bounded.
\end{theorem}

\begin{proof}
The proof is a direct consequence of the conservation law in theorem \ref{theorem:2.1} and the fact that the image of S under the projection $AA^+$ remains open and bounded.
\end{proof}


In the next section we introduced the Proper Symplastic Decomposition to construct a symplectic reduced basis.

\subsection{Proper Symplectic Decomposition (PSD)} \label{sec:3.2} Consider the set $S = \{z(t_i)| t_i\in[0,T],\ i=1,\dots,N\}$, of samples of the solution to (\ref{eq:2.1}), referred to as the \emph{snapshots}. The PSD, requires a symplectic basis $A$ of size $2k$ to have the minimum symplectic projection error over $S$ . In other words, a PSD symplectic basis is the solution to the optimization problem
\begin{equation} \label{eq:3.4}
	\begin{aligned}
	& \underset{A \in \mathbb R^{2n\times 2k}}{\text{minimize}, }
	& & \sum_{s\in S} \| s - AA^+s\|_2^2, \\
	& \text{subject to}
	& & A^T \mathbb{J}_{2n}A = \mathbb{J}_{2k}.
	\end{aligned}
\end{equation}

The Proper Orthogonal Decomposition (POD) \cite{hesthaven2015certified,quarteroni2015reduced} method, is based on a similar idea where the projection operator is an orthogonal projection and the constrain requires $A$ to be an ortho-normal basis. For the POD, the Schmidt-Mirsky-Eckart-Young theorem \cite{Markovsky:2011:LRA:2103589} provides a computationally efficient solution to the minimization problem which requires the SVD decomposition of the snapshot matrix $S$. As the direct solution to (\ref{eq:3.4}) is usually expensive, we must seek alternative methods to obtain near optimal solutions.

A heuristic approach to find a solution to (\ref{eq:3.4}) is proposed by \cite{Peng:2014di}, suggesting that the symplectic basis $A$ should take the form
\begin{equation} \label{eq:3.5}
	A = 
	\begin{pmatrix}
		\Phi & 0 \\
		0 & \Phi
	\end{pmatrix},
\end{equation}
where $\Phi \in \mathbb{R}^{n\times k}$ is an orthonormal matrix. By constructing the \emph{combined snapshot matrix}
\begin{equation} \label{eq:3.6}
	S_{\text{combined}} = [q_1,\dots,q_N,p_1,\dots,p_N], \quad (q_i^T,p_i^T)^T = z(t_i),
\end{equation}
we define $\Phi=[u_1,\dots,u_k]$, where $u_i$ is the $i$-th left singular vector of $S_{\text{combined}}$. The symplectic basis obtained this way is referred to as the \emph{Cotangent Lift}. It is shown in \cite{Peng:2014di} that among all symplectic bases of the form (\ref{eq:3.5}) the cotangent lift minimizes the projection error. Note that the basis $A$ constructed using the cotangent lift is an ortho-symplectic basis.

\subsection{The Reduced Dissipative Hamiltonian Method} \label{sec:3.3}

Since the theory of symplectic model reduction in section \ref{sec:3.2} is based on the conservation law in theorem \ref{theorem:2.1}, it can no longer be applied to dissipative Hamiltonian systems. We expect that for dissipative systems, the evolution of energy is not correctly translated into the reduced system. The Reduced Dissipative Hamiltonian method considers a Hamiltonian extension to a dissipative Hamiltonian system to construct a closed system. A symplectic model reduction can then be naturally applied to conserve the total energy.

Consider a dissipative Hamiltonian system of the form (\ref{eq:2.4}) with a quadratic Hamiltonian, $H(z) = z^TK^TKz$. Since $K^TK$ is symmetric and positive definite, it has a unique Cholesky factorization $K^TK = L^T L$ where $L$ is upper triangular \cite{strang09}. So we can write 
\begin{equation} \label{eq:3.7}
H(z) = z^T L^T L z.
\end{equation}
Further, suppose that the solution $z(t)$ lies on a low-dimensional symplectic subspace such that $z = Ay$, where $A$ is an ortho-symplectic matrix of the size $2n\times 2k$ and $y$ is the expansion coefficients of $z$ in the basis of $A$. The Hamiltonian $H$ then takes the form
\begin{equation} \label{eq:3.8}
	H(z) = H(Ay) = y^T A^T L^T L A y.
\end{equation}
Since $A^T L^T L A$ is a symmetric and positive definite matrix, it has a unique Cholesky factorization $\tilde L^T \tilde L$ where $\tilde L = A^T L A$ is an upper triangular matrix of size $2k \times 2k$. Writing (\ref{eq:2.4}) in terms of the reduced coordinates $y$ and the Hamiltonian (\ref{eq:3.7}) reads
\begin{equation} \label{eq:3.9}
		A\dot{y}(t) = \mathbb J_{2n} L^T f(t),
\end{equation}
together with the complementary equation
\begin{equation} \label{eq:3.10}
	f(t) + \sqrt 2 \cdot \sqrt{\chi} \phi(t,0) = LAy.
\end{equation}
Multiplying (\ref{eq:3.9}) with $A^+$ and (\ref{eq:3.10}) with $A^T$ yields
\begin{align} \label{eq:3.11}
	& \dot y(t) = \mathbb J_{2k} A^T L^T f(t), \\
	& A^T f(t) + \sqrt{2} A^T \sqrt{\chi} \phi(t,0) = A^T L A y
\end{align}
Here we used the fact that $A^+\mathbb J_{2n} = \mathbb{J}_{2k} A^T$. If we define $f = A \tilde f$, $\phi = A \tilde \phi$, $\theta = A\tilde \theta$ and the \emph{reduced susceptibility} as $\tilde \chi = A^T \chi A$ we recover the reduced Hamiltonian system
\begin{subequations}
\begin{align}
		\label{eq:3.12.a} & \dot{y}(t) = \mathbb J_{2k} {\tilde L}^T \tilde f(t), \\
		\label{eq:3.12.b} & \partial_t \tilde \theta(t,x) = \partial_x^2 \tilde \phi(t,x) + \sqrt 2 \delta_0(x) \cdot \sqrt{\tilde \chi}  \tilde f(t), \\
		\label{eq:3.12.c} & \partial_t \tilde \phi(t,x) = \tilde \theta(t,x),
\end{align}
\end{subequations}
together with the auxiliary equation
\begin{equation} \label{eq:3.13}
	\tilde f(t) + \sqrt{2} \sqrt{\tilde \chi} \tilde \phi(t,0) = \tilde L y.
\end{equation}
Equations (\ref{eq:3.12.a}) to (\ref{eq:3.12.c}) is a Hamiltonian system on the symplectic linear vector space $\mathbb R^{2k} \times \mathcal H^{2k}$ and contributes to the \emph{reduced TDD system}
\begin{equation}
	\dot {y} = \mathbb J_{2k} \tilde L^T \tilde f(t), \quad \tilde f(t) + \int_0^t \tilde \chi\cdot \tilde f(s)\ ds = \tilde L y.
\end{equation}
Therefore, the system energy will be conserved along integral curves of (\ref{eq:3.12.a}-\ref{eq:3.12.c}).

We point out that the transformation that transforms (\ref{eq:2.10.a}-\ref{eq:2.10.c}) to (\ref{eq:3.12.a}-\ref{eq:3.12.c}) is given by
\begin{equation}
	\mathcal A = \begin{pmatrix}
		A& 0 \\
		0& A
	\end{pmatrix} : \mathbb R^{2n} \times \mathcal H^{2n} \to \mathbb R^{2k} \times \mathcal H^{2k}.
\end{equation}
This is a symplectic transformation, since $\mathcal A^T J_{2n} \mathcal A = J_{2k}$. Further, the dissipation of energy in the reduced system only depends on the reduced susceptibility. Thus, the choice of $A$ should be independent of the hidden strings $(\phi, \theta)$. In other words, if the reduced space is chosen to be a symplectic subspace, then the actions of model reduction and Hamiltonian extension commute. We summarize the algorithm for model reduction of dissipative Hamiltonian systems in Algorithm \ref{alg:3.1}.

\begin{algorithm}
\caption{The Reduced Dissipative Hamiltonian Method (RDH)} \label{alg:3.1}
\begin{enumerate}
	\item Construct the Hamiltonian extension (\ref{eq:2.10.a}-\ref{eq:2.10.c}) to the original TDD system (\ref{eq:2.4}).
	\item Collect the snapshots $z(t_i)$, $i=1,\dots,N$ through time integration of the extended Hamiltonian.
	\item Construct an ortho-symplectic basis $A$.
	\item Define $\tilde L = A^T L A$, $\tilde \chi = A^T \chi A$ and construct the reduced dissipative Hamiltonian system (\ref{eq:3.12.a}-\ref{eq:3.12.c})
\end{enumerate}
\end{algorithm}

Note that Algorithm \ref{alg:3.1} does not depend on the choice of the method to construct an ortho-symplectic basis $A$. Thus, an SVD-based method, such as the cotangent lift, or a greedy-natured method can be applied.
