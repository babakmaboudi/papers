\section{Model Order Reduction} \label{sec:3}
In this section we first explain the main results of \cite{Maboudi:2016,Peng:2014di} regarding model reduction of Hamiltonian systems and, subsequently we introduce the Reduced Dissipative Hamiltonian method.

\subsection{Symplectic Model Order Reduction} \label{sec:3.1}
Consider a Hamiltonian system of the form (\ref{eq:2.1}) together with a quadratic Hamiltonian of the form $H(z) = \frac 1 2 z^T K^T K z$. In this paper we focus on reducing the complexity of the numerical evaluation of (\ref{eq:2.1}) with respect to time $t$. Nevertheless, one can extend the results of this paper to a Hamiltonian system that depends on a set of physical or geometrical parameters belonging to a compact subset of a Euclidean space.

The main idea behind model order reduction is that the solution manifold, $\mathcal M = \{ z(t) : t \in [0,T] \}$ can be approximated by a low dimensional linear subspace \cite{hesthaven2015certified,quarteroni2015reduced}. A basis for such a subspace is called a \emph{reduced basis}, and its span is referred to as the \emph{reduced space} \cite{hesthaven2015certified}.

Suppose that a reduced symplectic basis $A \in \mathbb R^{2n\times 2k}$ is provided with $k \ll n$. The approximate solution to (\ref{eq:2.1}) in this basis is expressed as
\begin{equation} \label{eq:3.1}
	z =Ay,
\end{equation}
where $y\in \mathbb R^{2k}$ are the coordinates of the approximation with respect to the basis $A$. Substituting (\ref{eq:3.1}) into (\ref{eq:2.1}) yields
\begin{equation} \label{eq:3.2}
	A \dot y = \mathbb{J}_{2n} \nabla_{z} H(A y).
\end{equation}
Multiplying both sides with the symplectic inverse of $A$ and using the chain rule we obtain
\begin{equation} \label{eq:3.3}
	\dot y = A^+ \mathbb{J}_{2n} (A^+)^T \nabla_{y} H(A y).
\end{equation}
As $(A^+)^T$ is a symplectic matrix it implies that $A^+ \mathbb{J}_{2n} (A^+)^T = \mathbb{J}_{2k}$. By defining the reduced Hamiltonian $\tilde H : \mathbb{R}^{2k} \to \mathbb R$, as $\tilde H (y) = H(Ay)$, we recover the reduced system
\begin{equation} \label{eq:3.3.1}
	\begin{aligned}
	\dot {y}(t) &= \mathbb J_{2k} \nabla_{y} \tilde H, \\
	y(0) &= A^+ z_0,
	\end{aligned}
\end{equation}
Equation (\ref{eq:3.3.1}) is called the \emph{symplectic Galerkin projection} \cite{Peng:2014di} of the Hamiltonian system (\ref{eq:2.1}). Conventional model reduction techniques, e.g. Galerkin or Petrov-Galerkin methods \cite{hesthaven2015certified,quarteroni2015reduced}, do not yield a Hamiltonian reduced system and the reduced system does not necessarily preserve the conservation law in Theorem \ref{theorem:2.1} which results in a qualitatively wrong and often unstable solution \cite{Peng:2014di}. On the other hand the reduced system, obtained by the symplectic Galerkin projection, is a Hamiltonian system, and the system energy is therefore preserved over time \cite{Peng:2014di}. The following theorem guarantees the boundedness of the solution of the reduced system obtained by the symplectic Galerkin projection for quadratic Hamiltonians.

\begin{theorem}
\cite{Peng:2014di} Let $S$ be a bounded open subset of $\mathbb R^{2n}$ that contain $z_0$. Furthermore, assume that $H(z_0)<H(z)$ or $H(z_0)>H(z)$ for all $z\in \partial S$, the boundary of $S$. For a given symplectic basis $A$, provided $z_0$ is in the range of $A$, then both the original system and the reduced system obtained by the symplectic Galerkin projection are bounded.
\end{theorem}

In the next section we introduced the greedy method to construct a symplectic reduced basis \cite{Maboudi:2016}.

\subsection{The Greedy Approach to Symplectic Basis Generation} \label{sec:3.2}
Greedy generation of the reduced basis is an iterative procedure which, in each iteration, adds the two best possible basis vectors to the symplectic basis to enhance overall accuracy. Suppose that at the $k$-th generic step of the greedy basis selection, an ortho-symplectic basis $A_{2k} = \{ e_1,\dots,e_k,f_1,\dots,f_k\}$ of size $2k$ is provided. The first step in an iteration of the greedy basis selection comprises finding $t^{k+1}\in[0,T]$ such $z(t^{k+1})$ is worst approximated by the current reduced space. In other words
\begin{equation} \label{eq:3.4}
	t^{k+1} := \underset{t\in [0,T]}{\text{argmax }} \| z(t) - A_{2k}{A_{2k}}^+z(t) \|.
\end{equation}
Suppose that $e_{k+1}$ is the vector obtained by $\mathbb J_{2n}$-orthogonalization (the symplectic Gram-Schmidt process \cite{Salam2014}) of $z(t^{k+1})$ with respect to $A_{2k}$. Then the enriched basis $A_{2k+2}$ takes the form
\begin{equation}
	A_{2k+2} = \{ e_1,\dots,e_k, e_{k+1},f_1,\dots,f_k,\mathbb J_{2n}^T e_{k+1}\}.
\end{equation}
It is easily checked that $A_{2k+2}$ is ortho-symplectic. Furthermore, it is shown in \cite{Maboudi:2016} that under natural assumptions on the solution manifold $\mathcal M$, the greedy method converges exponentially fast.

For parametric problems, evaluation of the projection error in (\ref{eq:3.4}) for the entire parameter space is computationally demanding. One can use the loss in the Hamiltonian function as a cheap surrogate to the projection error, i.e.
\begin{equation}
	\omega^{k+1} := \underset{\omega\in \Omega}{\text{argmax }} | H(z(\omega)) - H(AA^+z(\omega)) |.
\end{equation}
Here $\Omega$ is a closed and bounded set of parameters for the original Hamiltonian system. Note that since the Hamiltonian function is constant in time, then $\omega^{k+1}$ can be identified prior to time integration, i.e. $H(z) = H(z_0)$ and $H(AA^+z) = H(AA^+z_0)$. We summarize the greedy method for symplectic basis generation in Algorithm \ref{alg:added1} and refer the reader to \cite{Maboudi:2016} for further details.



\begin{algorithm} 
\caption{The greedy algorithm for generation of a symplectic basis} \label{alg:added1}
{\bf Input:} Tolerated projection error $\delta$, initial condition $ z_0$
\begin{enumerate}
\item $t^1 \leftarrow t=0$
\item $e_1 \leftarrow z_0$
\item $A \leftarrow [e_1,\mathbb J^T_{2n}e_1]$
\item $k \leftarrow 1$
\item \textbf{while} $\| z(t) - A_{2k}{A_{2k}}^+z(t) \| > \delta$ for all $t \in [0,T]$
\item \hspace{0.5cm} $t^{k+1} := \underset{t\in [0,T]}{\text{argmax }} \| z(t) - A_{2k}{A_{2k}}^+z(t) \|$
\item \hspace{0.5cm} $\mathbb J_{2n}$-orthogonalize $ z(t^{k+1})$ to obtain $e_{k+1}$
\item \hspace{0.5cm} $A \leftarrow [e_1,\dots ,e_{k+1} , \mathbb J^T_{2n}e_1,\dots,\mathbb J^T_{2n}e_{k+1}]$
\item \hspace{0.5cm} $k \leftarrow k+1$
\item \textbf{end while}
\end{enumerate}
\vspace{0.5cm}
{\bf Output:} Symplectic basis $A$.
\end{algorithm}


%In the next section we introduced the Proper Symplectic Decomposition to construct a symplectic reduced basis.
%
%\subsection{Proper Symplectic Decomposition (PSD)} \label{sec:3.2} Consider the set $S = \{z(t_i)| t_i\in[0,T],\ i=1,\dots,N\}$, of samples of the solution to (\ref{eq:2.1}), referred to as the \emph{snapshots}. The PSD, requires a symplectic basis $A$ of size $2k$ to have the minimum symplectic projection error over $S$ . In other words, a PSD symplectic basis is the solution to the optimization problem
%\begin{equation} \label{eq:3.4}
%	\begin{aligned}
%	& \underset{A \in \mathbb R^{2n\times 2k}}{\text{minimize}, }
%	& & \sum_{s\in S} \| s - AA^+s\|_2^2, \\
%	& \text{subject to}
%	& & A^T \mathbb{J}_{2n}A = \mathbb{J}_{2k}.
%	\end{aligned}
%\end{equation}
%
%The Proper Orthogonal Decomposition (POD) \cite{hesthaven2015certified,quarteroni2015reduced} method, is based on a similar idea where the projection operator is an orthogonal projection and the constrain requires $A$ to be an ortho-normal basis. For the POD, the Schmidt-Mirsky-Eckart-Young theorem \cite{Markovsky:2011:LRA:2103589} provides a computationally efficient solution to the minimization problem which requires the SVD decomposition of the snapshot matrix $S$. As the direct solution to (\ref{eq:3.4}) is usually expensive, we must seek alternative methods to obtain near optimal solutions.
%
%A heuristic approach to find a solution to (\ref{eq:3.4}) is proposed by \cite{Peng:2014di}, suggesting that the symplectic basis $A$ should take the form
%\begin{equation} \label{eq:3.5}
%	A = 
%	\begin{pmatrix}
%		\Phi & 0 \\
%		0 & \Phi
%	\end{pmatrix},
%\end{equation}
%where $\Phi \in \mathbb{R}^{n\times k}$ is an orthonormal matrix. By constructing the \emph{combined snapshot matrix}
%\begin{equation} \label{eq:3.6}
%	S_{\text{combined}} = [q_1,\dots,q_N,p_1,\dots,p_N], \quad (q_i^T,p_i^T)^T = z(t_i),
%\end{equation}
%we define $\Phi=[u_1,\dots,u_k]$, where $u_i$ is the $i$-th left singular vector of $S_{\text{combined}}$. The symplectic basis obtained this way is referred to as the \emph{Cotangent Lift}. It is shown in \cite{Peng:2014di} that among all symplectic bases of the form (\ref{eq:3.5}) the cotangent lift minimizes the projection error. Note that the basis $A$ constructed using the cotangent lift is an ortho-symplectic basis.
%
\subsection{The Reduced Dissipative Hamiltonian Method} \label{sec:3.3}

Since the symplectic model reduction in Section \ref{sec:3.2} is based on the conservation law in Theorem \ref{theorem:2.1}, it can no longer be applied to dissipative Hamiltonian systems. Instead in the Reduced Dissipative Hamiltonian method, we considers a Hamiltonian extension to a dissipative Hamiltonian system to construct a closed system. A symplectic model reduction can then be naturally applied to conserve the total energy.

Consider a dissipative Hamiltonian system of the form (\ref{eq:2.4}) with a quadratic Hamiltonian, $H(z) = z^TK^TKz$. Since $K^TK$ is symmetric and positive definite, it has a unique Cholesky factorization $K^TK = L^T L$ where $L$ is upper triangular \cite{strang09}. So we can write 
\begin{equation} \label{eq:3.7}
H(z) = z^T L^T L z.
\end{equation}
Further, suppose that the solution $z(t)$ lies on a low-dimensional symplectic subspace such that $z = Ay$, where $A$ is an ortho-symplectic matrix of the size $2n\times 2k$ and $y$ is the expansion coefficients of $z$ in the basis of $A$. The Hamiltonian $H$ then takes the form
\begin{equation} \label{eq:3.8}
	H(z) = H(Ay) = y^T A^T L^T L A y.
\end{equation}
Since $A^T L^T L A$ is symmetric and positive definite, it too has a unique Cholesky factorization $\tilde L^T \tilde L$ where $\tilde L = A^T L A$ is an upper triangular matrix of size $2k \times 2k$. Writing (\ref{eq:2.4}) in terms of the reduced coordinates $y$ and the Hamiltonian (\ref{eq:3.7}) reads
\begin{equation} \label{eq:3.9}
		A\dot{y}(t) = \mathbb J_{2n} L^T f(t),
\end{equation}
together with the complementary equation
\begin{equation} \label{eq:3.10}
	f(t) + \sqrt 2 \cdot \sqrt{\chi} \phi(t,0) = LAy.
\end{equation}
Multiplying (\ref{eq:3.9}) with $A^+$ and (\ref{eq:3.10}) with $A^T$ yields
\begin{align} \label{eq:3.11}
	& \dot y(t) = \mathbb J_{2k} A^T L^T f(t), \\
	& A^T f(t) + \sqrt{2} A^T \sqrt{\chi} \phi(t,0) = A^T L A y,
\end{align}
where we use the fact that $A^+\mathbb J_{2n} = \mathbb{J}_{2k} A^T$. If we define $f = A \tilde f$, $\phi = A \tilde \phi$, $\theta = A\tilde \theta$ and the \emph{reduced susceptibility} as $\tilde \chi = A^T \chi A$ we recover the reduced Hamiltonian system
\begin{subequations}
\begin{align}
		\label{eq:3.12.a} & \dot{y}(t) = \mathbb J_{2k} {\tilde L}^T \tilde f(t), \\
		\label{eq:3.12.c} & \partial_t \tilde \phi(t,x) = \tilde \theta(t,x),\\
		\label{eq:3.12.b} & \partial_t \tilde \theta(t,x) = \partial_x^2 \tilde \phi(t,x) + \sqrt 2 \delta_0(x) \cdot \sqrt{\tilde \chi}  \tilde f(t),
\end{align}
\end{subequations}
together with the auxiliary equation
\begin{equation} \label{eq:3.13}
	\tilde f(t) + \sqrt{2} \sqrt{\tilde \chi} \tilde \phi(t,0) = \tilde L y.
\end{equation}
Equations (\ref{eq:3.12.a})-(\ref{eq:3.12.b}) is a Hamiltonian system on the symplectic linear vector space $\mathbb R^{2k} \times \mathcal H^{2k}$ and contributes to the \emph{reduced TDD system}
\begin{equation}
	\dot {y} = \mathbb J_{2k} \tilde L^T \tilde f(t), \quad \tilde f(t) + \int_0^t \tilde \chi\cdot \tilde f(s)\ ds = \tilde L y.
\end{equation}
Therefore, the system energy will be conserved along integral curves of (\ref{eq:3.12.a})-(\ref{eq:3.12.c}).

We point out that the transformation that connects (\ref{eq:2.10.a})-(\ref{eq:2.10.b}) to (\ref{eq:3.12.a})-(\ref{eq:3.12.b}) is given by
\begin{equation}
	\mathcal A = \begin{pmatrix}
		A& 0 \\
		0& A
	\end{pmatrix} : \mathbb R^{2n} \times \mathcal H^{2n} \to \mathbb R^{2k} \times \mathcal H^{2k}.
\end{equation}
This is a symplectic transformation, since $\mathcal A^T \mathcal J_{2n} \mathcal A = \mathcal J_{2k}$. Furthermore, the dissipation of energy in the reduced system only depends on the reduced susceptibility. Thus, the choice of $A$ should be independent of the hidden strings $(\phi, \theta)$. In other words, if the reduced space is chosen to be a symplectic subspace, then the actions of model reduction and Hamiltonian extension commute. We summarize the algorithm for model reduction of dissipative Hamiltonian systems in Algorithm \ref{alg:3.1}.

\begin{algorithm}
\caption{The Reduced Dissipative Hamiltonian Method (RDH)} \label{alg:3.1}
\begin{enumerate}
	\item Construct the Hamiltonian extension (\ref{eq:2.10.a})-(\ref{eq:2.10.b}) to the original TDD system (\ref{eq:2.4}).
	\item Collect the snapshots $z(t_i)$, $i=1,\dots,N$ through time integration of the extended Hamiltonian.
	\item Construct an ortho-symplectic basis $A$.
	\item Define $\tilde L = A^T L A$, $\tilde \chi = A^T \chi A$ and construct the reduced dissipative Hamiltonian system (\ref{eq:3.12.a})-(\ref{eq:3.12.b})
\end{enumerate}
\end{algorithm}

Note that Algorithm \ref{alg:3.1} does not depend on the choice of the method to construct an ortho-symplectic basis $A$. Thus, for basis generation, the greedy approach introduced in section \ref{sec:3.2} or an SVD-based method, e.g. the cotangent lift \cite{Peng:2014di}, can be applied.

The main advantage of the RDH method compared to the existing method is that it enables the reduced system to be integrated using a symplectic integrator. The reduced system constructed using the RDH is a closed Hamiltonian system, therefore the conservation law in Theorem \ref{theorem:2.1} holds and a symplectic integrator guarantees that the total energy is preserved in the reduced system. Alternative methods, e.g. \cite{peng2016geometric,Polyuga:2010gj,beattie2011structure}, either integrate the reduced system with a non-symplectic integrator, or do not construct a closed reduced system which result in accumulation of local errors or unstable solution during long time integration respectively.
