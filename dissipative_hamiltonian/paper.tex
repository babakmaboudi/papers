%%%%%%%%%%%%%%%%%%%%%%% file template.tex %%%%%%%%%%%%%%%%%%%%%%%%%
%
% This is a general template file for the LaTeX package SVJour3
% for Springer journals.          Springer Heidelberg 2010/09/16
%
% Copy it to a new file with a new name and use it as the basis
% for your article. Delete % signs as needed.
%
% This template includes a few options for different layouts and
% content for various journals. Please consult a previous issue of
% your journal as needed.
%
%%%%%%%%%%%%%%%%%%%%%%%%%%%%%%%%%%%%%%%%%%%%%%%%%%%%%%%%%%%%%%%%%%%
%
% First comes an example EPS file -- just ignore it and
% proceed on the \documentclass line
% your LaTeX will extract the file if required
\begin{filecontents*}{example.eps}
%!PS-Adobe-3.0 EPSF-3.0
%%BoundingBox: 19 19 221 221
%%CreationDate: Mon Sep 29 1997
%%Creator: programmed by hand (JK)
%%EndComments
gsave
newpath
  20 20 moveto
  20 220 lineto
  220 220 lineto
  220 20 lineto
closepath
2 setlinewidth
gsave
  .4 setgray fill
grestore
stroke
grestore
\end{filecontents*}
%
\RequirePackage{fix-cm}
%
%\documentclass{svjour3}                     % onecolumn (standard format)
%\documentclass[smallcondensed]{svjour3}     % onecolumn (ditto)
\documentclass[smallextended]{svjour3}       % onecolumn (second format)
%\documentclass[twocolumn]{svjour3}          % twocolumn
%
\smartqed  % flush right qed marks, e.g. at end of proof
%
\usepackage{graphicx}
\usepackage{amsmath}
%\usepackage{amsthm}
\usepackage{amssymb}
\usepackage{color}
\usepackage{algorithm}
\usepackage{subfig}

%
% \usepackage{mathptmx}      % use Times fonts if available on your TeX system
%
% insert here the call for the packages your document requires
%\usepackage{latexsym}
% etc.
%
% please place your own definitions here and don't use \def but
% \newcommand{}{}
%
% Insert the name of "your journal" with
% \journalname{myjournal}
%
\begin{document}

\title{Structure-Preserving Model-Reduction of Dissipative Hamiltonian Systems%\thanks{Grants or other notes
%about the article that should go on the front page should be
%placed here. General acknowledgments should be placed at the end of the article.}
}
%\subtitle{Do you have a subtitle?\\ If so, write it here}

%\titlerunning{Short form of title}        % if too long for running head

\author{Babak Maboudi Afkham         \and
        Jan S. Hesthaven %etc.
}

%\authorrunning{Short form of author list} % if too long for running head

\institute{Babak Maboudi Afkham \at
              EPFL SB MATH MCSS, MA C1 645 (B\^atiment MA), Station 8, CH-1015 Lausanne, Switzerland \\
              Tel.: +41-21-6935905 \\
              \email{babak.maboudi@epfl.ch}           %  \\
%             \emph{Present address:} of F. Author  %  if needed
           \and
           Jan S. Hesthaven \at
           EPFL SB MATH MCSS, MA C2 652 (B\^atiment MA), Station 8, CH-1015 Lausanne, Switzerland
}

\date{Received: date / Accepted: date}
% The correct dates will be entered by the editor


\maketitle

\begin{abstract}
Reduced basis methods are popular for approximately solving large and complex systems of differential equations. However, conventional reduced basis methods do not generally preserve conservation laws and symmetries of the full order model. Here, we present an approach for reduced model construction, that preserves the symplectic symmetry of dissipative Hamiltonian systems. The method constructs a closed reduced Hamiltonian system by coupling the full model with a canonical heat bath. This allows the reduced system to be integrated with a symplectic integrator, resulting in a correct dissipation of energy, preservation of total energy and, ultimately, in the stability of the solution. Accuracy and stability of the method are illustrated through the numerical simulation of the dissipative wave equation and a port-Hamiltonian model of an electric circuit.

\keywords{Model order reduction \and Symplectic model reduction \and The Reduced Dissipative Hamiltonian method}
% \PACS{PACS code1 \and PACS code2 \and more}
% \subclass{MSC code1 \and MSC code2 \and more}
\end{abstract}

\section{Introduction}
\label{sec:intro}

Reduced order models have emerged as a powerful approach to cope with increasingly complex new applications in engineering and science. These methods substantially reduce the dimensionality of the problem by constructing a reduced configuration space. Exploration of the reduced space is then possible with significant acceleration \cite{hesthaven2015certified,Haasdonk2017}.

Over the past decade, reduced basis (RB) methods have demonstrated great success in lowering of the computational costs of solving elliptic and parabolic differential equations \cite{ito1998reduced,ito2001reduced}. However, model order reduction (MOR) of hyperbolic problems remains a challenge. Such problems often arise from a set of conservation laws and invariants. These intrinsic structures are lost during MOR which results in a qualitatively wrong, and sometimes unstable reduced system \cite{Amsallem:2014ef}.

%To have a sense of this error, error estimation is important from applications point of view \cite{HaasdonkOhlberger11,RuinerEtAl12,BhattEtAl18}. But it can difficult and expensive to compute useful error bounds. When one is interested in a cheap surrogate for the error incurred or when the conserved quantity is an output of the system, it becomes imperative to preserve this structure through model order reduction.

Recently, the construction of RB methods that conserve intrinsic structures has attracted attention \cite{doi:10.1137/17M1111991,1705.00498,kalashnikova2014stabilization,farhat2015structure,doi:10.1137/110836742,doi:10.1137/140959602,beattie2011structure,doi:10.1137/140978922}. Structure preservation in MOR not only constructs a physically meaningful reduced system, but can also enhance the robustness and stability of the reduced system. In system theory, conservation of passivity can be found in the work of \cite{polyuga2010structure,gugercin2012structure}. Energy preserving and inf-sup stable methods for finite element methods (FEM) are developed in \cite{farhat2015structure,ballarin2015supremizer}. Also, a conservative MOR technique for finite-volume methods is proposed in \cite{1711.11550}.

Moreover, the simulation of reduced models incurs solution errors and the estimation of this error is essential in applications of MOR \cite{HaasdonkOhlberger11,RuinerEtAl12,BhattEtAl18}. Finding tight error bounds for a general reduced system has shown to be computationally expensive and often impractical. Therefore, when one is interested in a cheap surrogate for the error or when the conserved quantity is an output of the system, it becomes imperative to preserve system structures in the reduced model.

In the context of Lagrangian and Hamiltonian systems, recent works provide a promising approach to the construction of robust and stable reduced systems. Carlberg, Tuminaro, and Boggs \cite{Carlberg:2014ky} suggest that a reduced order model of a Lagrangian system be identified by an approximate Lagrangian on a reduced order configuration space. This allows the reduced system to inherit the geometric structure of the original system. A similar approach has been adopted in the work of Peng and Mohseni \cite{doi:10.1137/140978922} and in the work of Maboudi Afkham and Hesthaven \cite{doi:10.1137/17M1111991} for Hamiltonian systems. They construct a low-order symplectic linear vector space, i.e. a vector space equipped with a symplectic 2-form, as the reduced space. Once the symplectic reduced space is generated, a symplectic projection result in a physically meaningful reduced system. A proper time-stepping scheme then preserves the Hamiltonian structure of the reduced system. It is shown in \cite{doi:10.1137/17M1111991,doi:10.1137/140978922} that this approach preserves the overall dynamics of the original system and enhances the stability of the reduced system. Despite the success of these method in MOR of Hamiltonian systems, these techniques are only compatible with the Euclidean inner product. Therefore, the computational structures that arise from a natural inner product of a problem will be lost during MOR.

Weak formulations and inner-products, defined on a Hilbert space, are at the core of the error analysis of many numerical methods for solving partial differential equations. Therefore, it is natural to seek MOR methods that consider such features. At the discrete level, these features often require a Euclidean vector space to be equipped with a generalized inner product, associated with a weight matrix $X$. Many works enabled conventional MOR techniques compatible with such inner products \cite{sen2006natural}. However, a MOR method that simultaneously preserves the symplectic symmetry of Hamiltonian systems remains unknown. 

In this paper, we seek to combine a classical MOR method with respect to a weight matrix with the symplectic MOR. The reduced system constructed by the new method is a generalized Hamiltonian system and the low order configuration space associated with this system is a symplectic linear vector space with a non-standard symplectic 2-form. It is demonstrated that the new method can be viewed as the natural extension to \cite{doi:10.1137/17M1111991}, and therefore retains the structure preserving features, e.g. symplecticity and stability. We also present a greedy approach for the construction of a generalized symplectic basis for the reduced system. Structured matrices are in general not norm bounded \cite{Karow:2006cf}. However, we show that the condition number of the basis generated by the greedy method is bounded by the condition number of the weight matrix $X$. Finally, to accelerate the evaluation of nonlinear terms in the reduced system, we present a variation of the discrete empirical interpolation method (DEIM) that preserves the symplectic structure of the reduced system.

What remains of this paper is organized as follows. In \cref{sec:hamil} we cover the required background on the Hamiltonian and the generalized Hamiltonian systems. \Cref{sec:mor} summarizes classic MOR routine with respect to a weighted norm and the symplectic MOR method with respect to the standard Euclidean inner product. We introduce the symplectic MOR method with respect to a weighted inner product in \cref{sec:normmor}. \Cref{sec:res} illustrates the performance of the new method through a vibrating beam and the sine-Gordon equation. We offer a few conclusive remarks in \cref{sec:conc}.

\section{Dissipative Hamiltonian Systems} \label{sec:2}
In this section we first discuss Hamiltonian systems. Add more later

\subsection{Hamiltonian Systems} Suppose that $(\mathbb{R}^{2n},\Omega)$ is a symplectic linear vector space, where $\mathbb{R}^{2n}$ is a configuration space and $\Omega:\mathbb{R}^{2n}\times \mathbb{R}^{2n} \to \mathbb R$ is a closed, skew-symmetric and non-degenerate 2-form on $\mathbb{R}^{2n}$. Given a smooth Hamiltonian function $H:\mathbb{R}^{2n}\to \mathbb R$, \emph{Hamilton}'s equations of evolution are given as
\begin{equation} \label{eq:2.1}
	\begin{aligned}
	\dot {z}(t) &= \mathbb J_{2n} \nabla_{z} H, \\
	z(0) &= z_0,
	\end{aligned}
\end{equation}
where $z \in\mathbb R^{2n}$ is the configuration coordinates and $\mathbb J_{2n}$ is a $2n\times 2n$ matrix such that $\Omega(x,y) = x^T \mathbb J_{2n} y$ \cite{Marsden:2010:IMS:1965128}. One by using the \emph{Symplectic Gram-Schmidt} \cite{de2006symplectic} can construct a coordinate system in which $\mathbb J_{2n}$ takes the form
\begin{equation} \label{eq:2.2}
	\mathbb{J}_{2n} = 
	\begin{pmatrix}
		0_n & I_n \\
		-I_n & 0_n
	\end{pmatrix},
\end{equation}
where $I_n$ and $0_n$ are the identity matrix and the zero matrix of size $n$ respectively. A main feature of Hamiltonian systems is the conservation of the Hamiltonian along the integral curves.
\begin{theorem} \label{theorem:2.1}
\cite{Marsden:2010:IMS:1965128} Consider the flow $\phi_t:\mathbb R\times \mathbb R^{2n} \to \mathbb R^{2n}$ of the Hamiltonian system (\ref{eq:2.1}). Then $H\circ \phi_t = H$.
\end{theorem}

In many physical problems $H$ represents the system energy and is bounded below. Here, we mostly assume that $H$ is a quadratic Hamiltonian, i.e., it takes the form $H(z) = \frac 1 2 z^T K^T K z$, where $K$ is a full rank $2n\times 2n$ matrix. This assumption leads to a linear system of evolution (\ref{eq:2.1}). Note that the Hamiltonian extension in section ?? and the model reduction for dissipative Hamiltonian systems in section ?? can be naturally extended to Hamiltonians of the form $H(z) = \frac 1 2 z^T K^T K z + g(z)$, where $g:\mathbb R^{2n} \to \mathbb R$ is an arbitrary function of $z$. 

Under general coordinate changes, equations of evolution do not necessarily take the form of (\ref{eq:2.1}). Let $(\mathbb R^{2n},\Omega)$ and $(\mathbb R^{2k},\Lambda)$ be two symplectic linear vector spaces. A linear transformation $\alpha :\mathbb R^{2n} \to \mathbb R^{2k}$ is called a \emph{symplectic transformation} \cite{Marsden:2010:IMS:1965128} if
\begin{equation}
	\Omega(x,y) = \Lambda(\alpha(x),\alpha(y)), \quad \text{for all } x,y\in \mathbb R^{2n}.
\end{equation}
Alternatively in matrix notation, $A\in \mathbb R^{2n\times 2k}$ is called a \emph{symplectic matrix} if
\begin{equation}
	A^T \mathbb{J}_{2n} A = \mathbb{J}_{2k},
\end{equation}
where the superscript $T$ represents the transpose operator. As the symplectic 2-form is preserved under symplectic transformations, the form of the equations of evolution remains invariant \cite{Marsden:2010:IMS:1965128}. The \emph{symplectic inverse} of $A$, is a form of pseudo-inverse given by
\begin{equation}
	A^+ = \mathbb{\mathbb J}_{2k}^T A^T \mathbb J_{2n}.
\end{equation}
It is shown in \cite{Peng:2014di} that $A^+A = I_{2k}$ and that $(A^+)^T$ is a symplectic matrix. Further, one can easily check that $AA^+$ is idempotent and so is a projection operator onto the column span of $A$. 

It is known that symplectic matrices are usually ill-conditioned \cite{Karow:2006cf}. Under some conditions on a symplectic space \cite{da2003introduction}, one can construct a symplectic basis which is also ortho-normal, and thus norm bounded. A basis which is both symplectic and orthogonal is referred to as an \emph{ortho-symplectic basis}. We refer the reader to \cite{da2003introduction} for conditions on existence and construction of an ortho-symplectic basis.

It is natural to expect a numerical integrator that solves (\ref{eq:2.1}) to also satisfy the conservation law in theorem \ref{theorem:2.1}. Common numerical integrators, e.g. the Runge-Kutta methos, do not generally preserve the Hamiltonian. Symplectic numerical integrators are a class of numerical integrators for Hamiltonian systems that preserves the symplectic structure and ensure stability in long-time integration. The Str\"omer-Verlet time stepping scheme is an example of such numerical integrators and is given by

\begin{equation} \label{eq:2.3}
\begin{aligned}
	p_{n+1/2} &= p_n - \frac{\Delta t}{2} \nabla_qH(q_{n},p_{n+1/2}), \\
	q_{n+1} &= q_n + \frac{\Delta t}{2} \left( \nabla_pH(q_{n},p_{n+1/2}) + \nabla_pH(q_{n+1},p_{n+1/2}) \right),\\
	p_{n+1} &= p_{n+1/2} - \frac{\Delta t}{2} \nabla_qH(q_{n+1},p_{n+1/2}),
\end{aligned}
\end{equation}
where $p$ and $q$ are the canonical coordinates. More information on construction and application of symplectic integrators can be found in \cite{Hairer:1250576}.

\subsection{Dissipative Hamiltonian Systems and Hamiltonian Extensions}

Many systems in engineering and physic appear as a perturbation of a Hamiltonian system, where the perturbation is regarded as dissipation. In these systems, the energy tends to decrease over time evolution, and the conservation law in Theorem \ref{theorem:2.1} does not hold anymore. Therefore, it is common to take the conservation of energy as a fundamental principle and consider the dissipative system coupled with a heat bath that absorbs the dissipated energy of the original system. 

To account dissipation for a quadratic Hamiltonian $H(z) = \frac 1 2 z^T K^T K  z$, we rewrite (\ref{eq:2.1}) as a time dispersive and dissipative (TDD) \cite{Figotin:2006jy} system 
\begin{equation} \label{eq:2.4}
	\begin{aligned}
		& \dot {z} = \mathbb J_{2n} K^T f(t), \\
		& z(0) = z_0,
	\end{aligned}
\end{equation}
where $f$ is the solution to the integral equation
\begin{equation} \label{eq:2.4.1}
	f(t) + \int_0^t \chi(t-s) \cdot f(s)\ ds = K z.
\end{equation}
Here $\chi:\mathbb R^+\to \mathbb R^{2n\times 2n}$ is a bounded matrix valued function with respect to the Frobenius norm and is called the \emph{general susceptibility}. Note that integral term in (\ref{eq:2.4.1}) corresponds to the dissipation, whereas if $\chi(s) = 0$ then (\ref{eq:2.3}) is equivalent to (\ref{eq:2.1}). Note that under suitable assumptions on $K$, both the systems (\ref{eq:2.1}) and (\ref{eq:2.4}) are well-posed \cite{Figotin:2006jy}.

\begin{example} \label{example:2.1}
Consider the dynamics of the damped harmonic oscillator
\begin{equation} \label{eq:2.5}
	\ddot q + r \dot q + k q = 0
\end{equation}
where $k$ is the Hooke's constant and $r$ is the spring's damping factor. Note that without a damping term, (\ref{eq:2.5}) is a Hamiltonian system. The TDD formulation for the damped harmonic oscillator takes the form
\begin{equation} \label{eq:2.6}
	\dot q(t) = f(t), \quad \dot p(t) = - k q(t), \quad f(t) + \int_0^t r f(s) \ ds = p(t).
\end{equation}
Here $(q,p)$ are the canonical coordinates and the susceptibility is the constant function $r$.
\end{example}

It is shown in \cite{Figotin:2006jy,Figotin:2005} that under natural assumptions on a linear susceptibility $\chi(t)$ (see below), one can couple a TDD system of the form (\ref{eq:2.4}) with a canonical heat bath where the dissipated energy is captured in the heat bath in a canonical sense. In other words, one can construct a Hilbert space $\mathcal H$ and an isometric injection $I:\mathbb R^{2n} \to \mathbb R^{2n}\times \mathcal H^{2n}$ where the solution $z$ to (\ref{eq:2.4}) is the projection of $x$ onto $\mathbb R^{2n}$, where $x$ is the solution to
\begin{equation} \label{eq:2.7}
	\frac{d}{dt} x = J_{2n} \frac{\delta H_{\text{ex}}}{\delta x}.
\end{equation}
Here $J_2n$ is the symplastic operator and $H_{\text{ex}}:\mathbb R^{2n}\times \mathcal H^{2n} \to \mathbb R$ is an extended quadratic Hamiltonian function defined on $\mathbb R^{2n}\times \mathcal H^{2n}$.

\begin{theorem}
Suppose that $K$ is full ranked and $\chi(t)$ is symmetric. There exists a quadratic extension of the form (\ref{eq:2.8}) to (\ref{eq:2.4}), if
\begin{equation} \label{eq:2.8}
	\text{Im}(\xi\hat{\chi}(\xi)) \geq 0, \quad \forall \xi = \omega + i\eta, \ \eta \geq 0,
\end{equation}
where $\hat{\chi}$ is the Fourier-Laplace transform of $\chi$
\begin{equation} \label{eq:2.9}
	\hat{\chi}(\xi) = \int_0^\infty e^{i\xi t} \chi(t)\ dt.
\end{equation}
\end{theorem}

\begin{proof}
Here we prove the theorem for the case where $\chi$ is a constant symmetric matrix, where condition (\ref{eq:2.8}) corresponds to $\chi$ being positive semi-definite. We refer the reader to \cite{Figotin:2006jy} for the proof of the general case. Consider the Hamiltonian system
\begin{subequations}
\begin{align}
		\label{eq:2.10.a} & \dot{z}(t) = \mathbb J_{2n} K^T f(t), \\
		\label{eq:2.10.b} & \partial_t \theta(t,x) = \partial_x^2 \phi(t,x) + \sqrt 2 \delta_0(x) \cdot \sqrt{\chi}  f(t), \\
		\label{eq:2.10.c} & \partial_t \phi(t,x) = \theta(t,x),
\end{align}
\end{subequations}
together with the initial condition
\begin{equation} \label{eq:2.10.1}
	z(0) = z_0,\quad \theta(0,\cdot) = 0, \quad \phi(0,\cdot) = 0.
\end{equation}
Here $\theta$ and $\phi$ are vector valued functions in $\mathcal H^{2n}$, $\delta_0(s)$ is the Dirac's delta function, $\sqrt{ \chi}$ is the matrix square root of $\chi$ and $f$ is the solution to the equation
\begin{equation} \label{eq:2.11}
	f(t) + \sqrt{2} \cdot \sqrt{ \chi } \phi(t,0) = Kz(t).
\end{equation}
To show that the Hamiltonian system (\ref{eq:2.10.a}) to (\ref{eq:2.10.c}) is an extension to (\ref{eq:2.4}) in the sense discussed before, it is enough to show that the solution $f$ to the equation (\ref{eq:2.11}) also satisfies (\ref{eq:2.4.1}). Equations (\ref{eq:2.10.b}) and (\ref{eq:2.10.c}) are equations to a vibrating string, and can be solved analytically
\begin{equation} \label{eq:2.12}
	\phi(t,x) = \frac {\sqrt 2} 2 \int_0^{t-|x|} \sqrt{\chi} f(s)\ ds,\quad \theta(t,x) = \frac{\sqrt 2}{2} \cdot \sqrt{\chi} f(t - |x|).
\end{equation}
\end{proof}
We can recover (\ref{eq:2.4.1}) by substituting (\ref{eq:2.12}) into (\ref{eq:2.11}). The extended Hamiltonian $H_\text{ex}$ for the system (\ref{eq:2.10.a}) to (\ref{eq:2.10.c}) takes the quadratic from
\begin{equation} \label{eq:2.13}
	H_\text{ex}(z,\phi,\theta) = \frac 1 2 \left( \| Kz - \phi(t,0) \|_2^2 + \| \theta(t) \|^2_{\mathcal H^{2n} } + \| \partial_x\phi(t)\|^2_{\mathcal H^{2n} }\right)
\end{equation}
where $\| \cdot \|_2$ is the Euclidean norm on $\mathbb R^{2n}$ and $\| \cdot \|_{\mathcal H^{2n}}$ is the induced norm from the inner product on $\mathcal H^{2n}$. 

Equations (\ref{eq:2.10.a}) and (\ref{eq:2.10.c}) are called the \emph{hidden strings}. The dissipation of energy in the original system (\ref{eq:2.4}) is carried away, as vibrations, along the added strings which makes the extended system conservative. Hamiltonian extension of the damped harmonic oscillator in Example \ref{example:2.1} is exactly the Lamb model \cite{lamb:1900}. The Lamb model is a harmonic oscillator coupled with a vibrating string, and the tension in the string causes linear dissipation in the dynamics of the harmonic oscillator.

Note that time integration of (\ref{eq:2.10.a})-(\ref{eq:2.10.c}) involves the integration of $f$ in (\ref{eq:2.12}). In general $f(t)$ must be stored and may cause serious storage limitation in long-time integration. However, we are interested solely in finding $z(t)$ which depends on $f(t)$, and $\phi(t,0)$ that is the integral of the history of $f(t)$. So by carefully choosing a quadrature rule that uses the same quadrature nodes as the time integrator we can avoid storing the history of $f$. For example for the trapezoidal rule, we recover the recursive relation
\begin{equation} \label{eq:2.14}
	\int_{0}^{t_n} f(s) \ ds \approx \frac{\Delta t}{2} f(t_n) + \frac{\Delta t}{2} f(t_{n-1}) + \int_{0}^{t_{n-1}} f(s) \ ds,
\end{equation}
where $\Delta t$ is the time discretization step. The recursive relation in (\ref{eq:2.12}) suggests that only storing the value of the integral term together with the state of $f$ in the previous time step is required to evaluate the integral for the new time step. For other interpolation based quadrature rules, we can construct similar recursive rules of the form
\begin{equation}
	\int_{0}^{t_n} f(s) \ ds \approx \sum_{i=0}^{k} \omega_i f(t_{n-i})  + \int_{0}^{t_{n-k}} f(s) \ ds
\end{equation}
for some quadrature weights $\omega_i$, $i=1,\dots,k$ with $k\ll n$. Thus, time integration of (\ref{eq:2.10.a})-(\ref{eq:2.10.c}), requires storing only $k$ evaluations of $f$.

\input{./chapters/3.Model.tex}
\input{./chapters/4.Numer.tex}
\section{Conclusion} \label{sec:5}

In this paper we present the Reduced Dissipative Hamiltonian method. The method preserves the symplectic structure of dissipative Hamiltonian systems and guarantees the correct dissipation of energy through time integration. The RDH method couples the reduced system with a canonical heat bath such that the reduced system forms a closed system.

The main advantage of the RDH method compared to the existing methods is that it enables the reduced system to be integrated using a symplectic integrator which naturally preserves the Hamiltonian structure and the symplectic symmetry of the Hamiltonian systems. Applying a symplectic integrator to a non-conservative system or using a non-symplectic integrator for the reduced system can cause accumulation of local errors or wrong qualitative solution over long-time integration, respectively.

The numerical simulations illustrate that the RDH method preserves the system energy with significantly higher accuracy than other methods. Furthermore, it is shown that the hidden strings assure that the dissipation of energy in the reduce system mimics the dissipation of energy in the full system. This ensures that the local error do not accumulate over long-time integration.




%\begin{acknowledgements}
%If you'd like to thank anyone, place your comments here
%and remove the percent signs.
%\end{acknowledgements}

% BibTeX users please use one of
%\bibliographystyle{spbasic}      % basic style, author-year citations
%\bibliographystyle{spmpsci}      % mathematics and physical sciences
%\bibliographystyle{spphys}       % APS-like style for physics
%\bibliography{}   % name your BibTeX data base

% Non-BibTeX users please use
%\begin{thebibliography}{}
%%
%% and use \bibitem to create references. Consult the Instructions
%% for authors for reference list style.
%%
%\bibitem{RefJ}
%% Format for Journal Reference
%Author, Article title, Journal, Volume, page numbers (year)
%% Format for books
%\bibitem{RefB}
%Author, Book title, page numbers. Publisher, place (year)
%% etc
%\end{thebibliography}

\bibliographystyle{siamplain}
\bibliography{ref}

\end{document}
% end of file template.tex

