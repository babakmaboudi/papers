\section{Introduction}
Parameterized partial differential equations are very often used as tools to model many problems in engineering and the applied sciences. While the need for more accuracy has led to the development of exceedingly complex models, the computational and storage limitations often make direct approaches impossible. Hence, we must seek alternative methods that allow us to approximate the desired output under variation of the input parameters while keeping the computational costs to a minimum.

Reduced basis methods have emerged as a powerful approach for the reduction of the intrinsic complexity of such models \cite{Ito:1998up,Ito:1998ch,Ito:2001ev,Peterson:1989ki}. These methods contain two stages: the offline and the online. In the offline stage, one explors the parameter space to construct a low-dimensional basis that accurately represents the parametrized solution to the partial differential equation. In this stage, evaluation of the solution of the original model for multiple parameter values is required. The online stage comprises a Galerkin projection onto the span of the reduced basis, which allows exploration of the parameter space at significantly reduced complexity \cite{Antoulas:2005:ALD:1088857,Anonymous:2016wl}.

Convectional reduced basis techniques e.g. proper orthogonal decomposition (POD) \cite{Kunisch:2002er,Atwell:2001ja,Ravindran:2002hn}, require the exploration of the entire parameter space. This leads to a very expensive and often impractical offline stages when dealing with multi-dimensional paramtere domains. On the other hand, sampling techniques, usually of a greedy nature, search through the parameter space selectively often guided by an error estimate to certify the accuracy of the basis. This approach, accompanied with an efficient sampling procedure, trades considerable amount of computation with overall accuracy of the reduced-basis \cite{Cuong:2005gd,Rozza:2005ie,Anonymous:2016wl}.

Beside computational complexity, another aspect of reduced order modeling is the preservation of structures and, in particular, stability of the original model. In general, reduced order models do not guarantee that such properties are preserved \cite{Anonymous:pMn0O0Q4}. 

In the context of Hamiltonian and Lagrangian systems, recent work suggests modifications of POD to preserve geometric structure. Lall et al. \cite{Lall:2003iy} and Carlberg et al. \cite{Carlberg:2014ky} suggests that the reduced-order system should be identified by a Lagrangian function on a low-ordered configuration space. In this way the geometric structure of the original system is inherited by the reduced system. For Hamiltonian systems Peng et al. \cite{Peng:2014di}, using a symplectic transformation, constructs a reduced Hamiltonian, as an approximation to the Hamiltonian of the original system. As a result the reduced system preserves the symplectic structure. Although these methods preserve geometric structure, they use a POD-like approach for constructing the reduced basis, and are not well suitable for problems with a high-dimensional parameter domain.

In this paper, we present a greedy approach for the construction of a reduced system that preserves the geometric structure of Hamiltonian systems, as an iterative routine that successively enriches the reduced basis at each iteration. On one hand this technique results in a reduced Hamiltonian system that mimics symplectic properties of the original system and preserves the Hamiltonian structure and stability over the course of time integration. On the other hand, since time integration of the original system is only required once per iteration, the proposed method saves substantial computational cost during the offline stage when compared to alternative POD-like approaches. Moreover, we demonstrate that assumptions, natural for the set of all solutions of the original Hamiltonian system under variation of parameters, lead to exponentially fast convergence of the greedy algorithm. For nonlinear Hamiltonian systems, we show how the basis can be combined with the discrete empirical interpolation method (DEIM) to allow a fast evaluation of nonlinear terms while maintaining the symplectic structure.

This paper is organized as follows. Section \ref{chap:MoOr:1} presents a brief overview of model order reduction, POD and DEIM. In Section \ref{chap:Hasy:1} we cover the required topics from symplectic geometry and Hamiltonian systems. Section \ref{chap:SyMo:1} discusses the greedy generation of a symplectic reduced basis as well as other SVD-based symplectic model reduction techniques. Accuracy, stability, and efficiency of the greedy method compared to other SVD-based methods are discussed in Section \ref{chap:NuRe:1}. Finally we offer the conclusion in Section \ref{chap:Con:1}.
