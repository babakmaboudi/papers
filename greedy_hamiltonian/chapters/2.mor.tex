\section{Model Order Reduction} \label{chap:MoOr:1}
Consider a parameterized, finite dimensional dynamical system that can be described by a set of first order ordinary differential equations on the form
\begin{equation} \label{eq:MoOr:1}
\left\{
\begin{split}
& \frac{d}{dt} \mathbf{x}(t,\omega) = \mathbf f (t,\mathbf x,\omega), \\
& \mathbf x(0,\omega) = \mathbf x_0(\omega).
\end{split}
\right.
\end{equation}
Here $\mathbf x \in \mathbb R^n$ is the state vector, $\omega \in \Gamma$ is a vector containing all the parameters of the system that belongs to a compact set $\Gamma$ ($\subset \mathbb R^d$) and $\mathbf f : \mathbb R \times \mathbb R^n \times \Gamma \to \mathbb R^n$ is a general function of the state variables and parameters.

We define the solution manifold as the set of all solutions to (\ref{eq:MoOr:1}) under variation of parameter $\omega$
\begin{equation} \label{eq:MoOr:2}
	\mathcal M = \{ \mathbf x(\omega) | \omega \in \Gamma \} \subset \mathbb R^n.
\end{equation}
Note that the exact solution and solution manifold is not always available; we assume that we have a numerical integrator in hand that can approximate the solution to (\ref{eq:MoOr:1}) for any realization of $\omega$ with arbitrary accuracy. By abuse of notation we refer to $\mathbf x$ and $\mathcal M$ as the exact solution and the exact solution manifold, respectively, rather than the discrete solution and discrete solution manifold. 

Model order reduction is based on the assumption that $\mathcal M$ is of low dimension \cite{Anonymous:2016wl,Antoulas:2005:ALD:1088857}, and that the span of appropriately chosen basis vectors $\{v_i\}_{i=1}^k$ covers most of the solution manifold with a small error. The set $\{v_i\}_{i=1}^k$ is denoted as the reduced basis and its span as the reduced space. Assuming that a $k$-dimensional $(k\ll n)$ reduced basis is given, the approximated solution can be represented as
\begin{equation} \label{eq:MoOr:3}
	\mathbf x \approx V \mathbf y,
\end{equation}
where $V$ is a matrix containing the reduced basis vectors as its columns and $\mathbf y$ is the coordinates of the approximation in this basis. By substituting (\ref{eq:MoOr:3}) into (\ref{eq:MoOr:1}) we obtain the overdetermined system
\begin{equation} \label{eq:MoOr:4}
	V \frac{d}{dt} \mathbf y = \mathbf f (t , V \mathbf y , \omega) + \mathbf r(t,\omega).
\end{equation}
Here we added the residual $\mathbf r$ to emphasise that (\ref{eq:MoOr:4}) is an approximation of (\ref{eq:MoOr:1}). Taking the Petrov-Galerkin projection \cite{Antoulas:2005:ALD:1088857} we construct a basis $W$ of size $n-k$ that is orthogonal to the residual $\mathbf r$ and assume $W^T V$ is invertible. This yields
\begin{equation} \label{eq:MoOr:5}
	\frac{d}{dt} \mathbf y = (W^TV)^{-1} \mathbf f(t,V\mathbf y,\omega).
\end{equation}
Equation (\ref{eq:MoOr:5}) consists of $k$ equations and is called the reduced system. Solving the reduced system instead of the original system can reduce the computational costs since $k$ is significantly smaller than $n$. For nonlinear systems, the evaluation of $\mathbf f$ may still have computational complexity that depends on $n$. We return to this question in detail in Section \ref{chap:MoOr.DEIM:1}.

\subsection{Proper Orthogonal Decomposition} \label{chap:MoOr.PrOr:1}
Let $\mathbf x (t_i,\omega_j)$ with $i=1,\dots,m$ and $j=1,\dots,n$ be a finite number of samples, referred to as snapshots, from the solution manifold (\ref{eq:MoOr:2}). If we Suppose that a reduced basis $V$ is provided, the projection operator from $\mathbb R^n$ onto the reduced space can be constructed as $VV^T$. The proper orthogonal decomposition (POD) requires the total error of projecting all the snapshots onto the reduced space to be minimum. The POD basis of size $k$ is thus the solution of the optimization problem
\begin{equation} \label{eq:MoOr:6}
\begin{aligned}
& \underset{V\in \mathbb R^{n\times k}}{\text{minimize}}
& & \| S - VV^TS\|_F \\
& \text{subject to}
& & V^TV = I_k
\end{aligned}
\end{equation}
Here $S$ is the snapshot matrix,, containing snapshots $\mathbf x(t_i,\omega_j)$ in its columns, $\|\cdot \|_F$ is the Frobenius norm of a matrix and $I_k$ is the identity matrix of size $k$. According to Schmidt-Mirsky-Eckard-Young theorem \cite{Antoulas:2005:ALD:1088857}, the solution to the optimization (\ref{eq:MoOr:6}) is equivalent to the truncated singular value decomposition (SVD) of the snapshot matrix $S$ given by
\begin{equation} \label{eq:MoOr:7}
	V = \sigma_1 u_1 v^T_1 + \cdots + \sigma_k u_k v^T_k.
\end{equation}
Here $\sigma_i, u_i$ and $v_i$ are the singular values, left singular vectors and right singular vectors of $S$, respectively.


\subsection{Discrete Empirical Interpolation Method (DEIM)} \label{chap:MoOr.DEIM:1} \nocite{Chaturantabut:2010cz}
Let us now discuss the inefficiency of evaluating nonlinearities when using projection based reduced model. Suppose that the right hand side function in (\ref{eq:MoOr:1}) is of the form $\mathbf f(t,\mathbf x , \omega) = L\mathbf x + \mathbf g(t,\mathbf x ,\omega)$, where $L\in \mathbb R^{n\times n}$ reflects the linear part, and $\mathbf g$ is a nonlinear function. Now suppose a $k$-dimensional reduced basis $V$ is provided, the reduced system takes the form
\begin{equation} \label{eq:MoOr:8}
	\frac{d}{dt} \mathbf y = \underbrace{(WV)^{-1} L V}_{\tilde L} \mathbf{y} + \underbrace{(WV)^{-1} \mathbf g(t,V\mathbf y,\omega)}_{\tilde N (\mathbf y)}.
\end{equation}
Here, $\tilde L$ is a $k\times k$ matrix which can be computed before time integration of the reduced system. However, evaluation of $\tilde N (\mathbf y)$ has a complexity that depends on $n$, the size of the original system. Suppose that evaluation of $\mathbf g$ with $n$ components has complexity $\alpha(n)$, for some function $\alpha$. Then the complexity of evaluating $\tilde N(\mathbf y)$ is $\mathcal{O}(\alpha(n) + 4nk)$ which consists of 2 matrix-vector operations and the evaluation of the nonlinear function. This means that estimation of the nonlinear terms can be as expensive as solving the original system.

To overcome this bottleneck we take an approach similar to what we did above. Assume that the manifold $\mathcal M_{\mathbf g} = \{ \mathbf g(t,\mathbf x , \omega)| t\in \mathbb R, \mathbf x \in \mathbb R , \omega \in \Gamma\}$ is of a low dimension and that $\mathbf g$ can be approximated by a linear subspace of dimension $m\ll n$, spanned by the basis $\{ u_1 , \dots , u_m \}$, i.e.
\begin{equation} \label{eq:MoOr:10}
	\mathbf g(t,\mathbf x,\omega) \approx U \mathbf c(t,\mathbf x,\omega).
\end{equation}
Here $U$ contains basis vectors $u_i$ and $\mathbf c$ is the corresponding vector of coefficients. Now suppose $p_1,\dots,p_m$ are $m$ indices from $\{1,\dots,n\}$ and define an $n\times m$ matrix
\begin{equation} \label{eq:MoOr:11}
	P = [e_{p_1},\dots,e_{p_m}],
\end{equation}
where $e_{p_i}$ is the $p_i$-th column of the identity matrix $I_n$. Hence, multiplying $P$ on $\mathbf g$ selects components $p_1,\dots,p_m$ of $\mathbf g$. If we assume that $P^TU$ is non-singular, the coefficient vector $\mathbf c$ can be uniquely determined from
\begin{equation} \label{eq:MoOr:12}
	P^T \mathbf g = (P^TU)\mathbf c.
\end{equation}
Finally the approximation of $\mathbf g$ is determined by
\begin{equation} \label{eq:MoOr:13}
	\mathbf g(t,\mathbf x,\omega) \approx U \mathbf c(t,\mathbf x,\omega) = U (P^TU)^{-1} P^T \mathbf g(t,\mathbf x,\omega),
\end{equation}
which is referred to as the \emph{Discrete Empirical Interpolation} (DEIM) approximation \cite{Chaturantabut:2010cz}. Applying DEIM on the reduced system (\ref{eq:MoOr:5}) yields
\begin{equation} \label{eq:MoOr:14}
	\frac{d}{dt} \mathbf y = \tilde L \mathbf y + (WV)^{-1} U(P^TU)^{-1}P^T \mathbf g(t,V\mathbf y , \omega).
\end{equation}
Note that the matrix $(WV)^{-1} U(P^TU)^{-1}$ can be computed offline and since $\mathbf g$ is evaluated only at $m$ of its components, the evaluation of the nonlinear term in (\ref{eq:MoOr:14}) does not depend on $n$.

To obtain the projection basis $U$, the POD can be applied to the ensemble of the discrete evaluation of $\mathbf g$. There is no additional cost for computing the nonlinear snapshots since they are generated when computing the trajectory snapshot matrix $S$. The interpolating indices $p_1,\dots,p_m$ can be constructed as follows. Given the projection basis $U = \{u_1,\dots,u_m\}$, the first interpolation index $p_1$ is chosen according to the component of $u_1$ with the largest magnitude. The rest of the interpolation indices, $p_2,\dots,p_m$ correspond to the component of the largest magnitude of the residual vector $\mathbf r = u_l - U \mathbf c$. It is shown in \cite{Chaturantabut:2010cz} that if the residual vector is a nonzero vector in each iteration then $P^TU$ is non-singular and (\ref{eq:MoOr:13}) is well defined. We summarize the algorithm of generating interpolating indices in Algorithm \ref{alg:MoOr:1}.

\begin{algorithm} 
\caption{Discrete Empirical Interpolation Method} \label{alg:MoOr:1}
{\bf Input:}  Basis vectors $\{u_1,\dots , u_m\}\subset \mathbb R^n$
\begin{enumerate}
\item pick $p_1$ to be the index of the largest component of $u_1$.
\item $U \leftarrow [u_1]$
\item $P \leftarrow [p_1]$
\item \textbf{for} $i\leftarrow 2$ \textbf{to} $m$
\item \hspace{0.5cm} solve $(P^TU)\mathbf c = P^T u_i$ for $\mathbf c$
\item \hspace{0.5cm} $\mathbf r \leftarrow u_i - U\mathbf c$
\item \hspace{0.5cm} pick $p_i$ to be the index of the largest component of $\mathbf r$
\item \hspace{0.5cm} $U \leftarrow [u_1,\dots,u_i]$
\item \hspace{0.5cm} $P \leftarrow [p_1,\dots,p_i]$
\item \textbf{end for}
\end{enumerate}
\vspace{0.5cm}
{\bf Output:} Interpolating indices $\{p_1,\dots,p_m\}$
\end{algorithm}


If we use an implicit time integration scheme, a similar inefficiency occurs for the computation of Newton iteration. In addition to the nonlinear term, the Jacobian of $\mathbf g(\mathbf x)$ with respect to the reduced state vector must be computed at each iteration. The Jacobian takes the form
\begin{equation} \label{eq:MoOr:9}
	\mathbf J_{\mathbf y}(\mathbf g) = (WV)^{-1} \mathbf J_{\mathbf x}(\mathbf g) V.
\end{equation}
The complexity of computing (\ref{eq:MoOr:9}) is $\mathcal{O}(\alpha(n) +2n^2k+2nk^2+2nk)$ comprising several matrix-vector multiplications and an evaluation of the Jacobian which depends on the size of the original system. Approximating the Jacobian in (\ref{eq:MoOr:9}) is usually both problem and discretization dependent. Very often the nonlinear function $\mathbf g$ is evaluated component-wise i.e.
\begin{equation} \label{eq:MoOr:15}
	\mathbf g(\mathbf x) =
	\begin{pmatrix}
		g_1(x_1,\dots,x_n) \\
		g_2(x_1,\dots,x_n) \\
		\vdots \\
		g_n(x_1,\dots,x_n)
	\end{pmatrix}
	=
	\begin{pmatrix}
		g_1(x_1) \\
		g_2(x_2) \\
		\vdots \\
		g_n(x_n)
	\end{pmatrix}.
\end{equation}
with $\mathbf x$ the configuration vector. In such cases the interpolating index matrix $P$ and the nonlinear function $\mathbf g$ commute. I.e., we have
\begin{equation} \label{eq:MoOr:16}
	\tilde N(\mathbf y) \approx (WV)^{-1} U(P^TU)^{-1}P^T \mathbf g(V\mathbf y) = (WV)^{-1} U(P^TU)^{-1}\mathbf g(P^TV\mathbf y)
\end{equation}
If we now take the Jacobian of the approximate function we recover
\begin{equation}
	\mathbf J_{\mathbf y}(\mathbf g) = \underbrace{ (WV)^{-1} U(P^TU)^{-1} }_{k\times m} \underbrace{ \mathbf J_{\mathbf x}(\mathbf g(P^T V \mathbf y) ) }_{m\times m} \underbrace{P^T V}_{k\times k}.
\end{equation}
The matrix $(WV)^{-1} U(P^TU)^{-1}$ can be computed offline and the Jacobian is evaluated only at $m\times m$ components. Hence the overall complexity of computing the Jacobian is now independent of $n$.
