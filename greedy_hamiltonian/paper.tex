%%
%% This is file `docultex.tex', 
%% Documentation for siam supplemental file macros for use with LaTeX 2e
%% 
%% December 19, 2013
%%
%% Supplemental Files Supported Version 1.0.0
%% 
%% You are not allowed to change this file. 
%% 
%% You are allowed to distribute this file under the condition that 
%% it is distributed together with all of the files in the siam macro 
%% distribution. These are:
%%
%%  siamltex1213.cls (main LaTeX macro for SIAM)
%%  siam10.clo   (size option for 10pt papers)
%%  subeqn.clo   (allows equation numbners with lettered subelements)
%%  siam.bst     (bibliographic style file for BibTeX)
%%  docultex.tex (this file)
%%
%% If you receive only some of these files from someone, complain! 
%% 
%% You are NOT ALLOWED to distribute this file alone. You are NOT 
%% ALLOWED to take money for the distribution or use of either this 
%% file or a changed version, except for a nominal charge for copying 
%% etc. 
%% \CharacterTable
%%  {Upper-case    \A\B\C\D\E\F\G\H\I\J\K\L\M\N\O\P\Q\R\S\T\U\V\W\X\Y\Z
%%   Lower-case    \a\b\c\d\e\f\g\h\i\j\k\l\m\n\o\p\q\r\s\t\u\v\w\x\y\z
%%   Digits        \0\1\2\3\4\5\6\7\8\9
%%   Exclamation   \!     Double quote  \"     Hash (number) \#
%%   Dollar        \$     Percent       \%     Ampersand     \&
%%   Acute accent  \'     Left paren    \(     Right paren   \)
%%   Asterisk      \*     Plus          \+     Comma         \,
%%   Minus         \-     Point         \.     Solidus       \/
%%   Colon         \:     Semicolon     \;     Less than     \<
%%   Equals        \=     Greater than  \>     Question mark \?
%%   Commercial at \@     Left bracket  \[     Backslash     \\
%%   Right bracket \]     Circumflex    \^     Underscore    \_
%%   Grave accent  \`     Left brace    \{     Vertical bar  \|
%%   Right brace   \}     Tilde         \~}

\documentclass[final,leqno,onefignum,onetabnum]{siamltex1213}

%-------------------------------------------------------------------------
%
%	PACKAGES AND OTHER DOCUMENT CONFIGURATIONS
%
%-------------------------------------------------------------------------

\usepackage{./pkg/packages}


\title{Structure Preserving Model Reduction of Parametric Hamiltonian Systems} 

\author{
	Babak Maboudi Afkham\thanks{Department of Mathematics, Chair of Computational Mathematics and Simulation Schience (MCSS), \'Ecole Polytechnique F\'ed\'erale de Lausanne, Switzerland (babak.maboudi@epfl.ch)} 
	\and 
	Jan S. Hesthaven\thanks{Department of Mathematics, Chair of Computational Mathematics and Simulation Schience (MCSS), \'Ecole Polytechnique F\'ed\'erale de Lausanne, Switzerland (jan.hesthaven@epfl.ch)}}


\begin{document}
\maketitle
\slugger{mms}{xxxx}{xx}{x}{x--x}%slugger should be set to mms, siap, sicomp, sicon, sidma, sima, simax, sinum, siopt, sisc, or sirev

\begin{abstract}
While reduced-order models (ROMs) are popular for approximately solving large systems of differential equations, the stability of reduced models over long-time integration remains an open question. We present a greedy approach for ROM generation of parametric Hamiltonian systems which captures the symplectic structure of Hamiltonian systems to ensure stability of the reduced model. Through the greedy selection of basis vectors, two new vectors are added at each iteration to the set of basis vectors to increase the overall accuracy of the reduce basis. We used the error in the Hamiltonian function due to model reduction, as an error indicator to search the parameter space and find the next best basis vectors. We show that the greedy algorithm converges with exponential rate, under natural assumptions on the set of all solutions of the Hamiltonian system under variation of the parameters. Moreover, we demonstrate that combining the greedy basis with the discrete empirical interpolation method also preserves the symplectic structure. This enables the reduction of computational cost for nonlinear Hamiltonian systems. The efficiency, accuracy, and stability of this model reduction technique is illustrated through simulations of the parametric wave equation and the parametric Schr\"odinger equation.


%Through the greedy selection of basis vectors, we used the Hamiltonian function as an error estimator to search for the 


%to add basis vectors, two at a time, and increase the overall accuracy of the reduced basis. It therefor requires a single numerical solution to be computed at each iteration which can save computational costs in stage offline stage. Under


%We also illustrated that convergence rate of the greedy approach is proportional to the best possible convergence rate of any possible iterative update of the reduced basis. Moreover, for nonlinear Hamiltonian systems, we combined the reduced basis obtained from the greedy method with discrete empirical interpolation method and demonstrated the

\end{abstract}

\begin{keywords}Symplectic model reduction, Hamiltonian system, Greedy basis generation, Symplectic Discrete Empirical Interpolation (SDEIM) \end{keywords}

\begin{AMS}\end{AMS}


\pagestyle{myheadings}
\thispagestyle{plain}
\markboth{Babak Maboudi Afkham and Jan S. Hesthaven}{Symplectic Model reduction of Parametric Hamiltonian Systems}


\section{Introduction}
{\edit Parameterized partial differential equations often arise as a model in many problems in engineering and the applied sciences}. While the need for more accuracy has led to the development of exceedingly complex models, the {\edit limitations in computational cost and storage} often make direct approaches {\edit impractical}. Hence, we must seek alternative methods that allow us to approximate the desired output under variation of the input parameters while keeping the computational costs to a minimum.

Reduced basis methods have emerged as a powerful approach for the reduction of the intrinsic complexity of such models \cite{Ito:1998up,Ito:1998ch,Ito:2001ev,Peterson:1989ki}. These methods contain two stages: {\edit the offline stage and the online stage}. In the offline stage, one explores the parameter space to construct a low-dimensional basis that accurately represents the parametrized solution to the partial differential equation. In this stage, the evaluation of the solution of the original model for multiple parameter values is required. The online stage comprises a Galerkin projection onto the span of the reduced basis, which allows exploration of the parameter space at a significantly reduced complexity \cite{Antoulas:2005:ALD:1088857,Anonymous:2016wl}.

Convectional reduced basis techniques, such as {\blue the} Proper Orthogonal Decomposition (POD) \cite{Kunisch:2002er,Atwell:2001ja,Ravindran:2002hn}, require the exploration of the entire parameter space. This leads to a very expensive and often impractical offline {\edit stage} when dealing with multi-dimensional parameter domains. On the other hand, sampling techniques, usually of a greedy nature, search through the parameter space selectively, guided by an error estimate to certify the accuracy of the basis. This approach, accompanied with an efficient sampling procedure, {\edit balances the cost} of computation with the overall accuracy of the reduced-basis \cite{Cuong:2005gd,Rozza:2005ie,Anonymous:2016wl}.

{\edit Besides} computational complexity, another aspect of reduced order modeling is the preservation of structure and, in particular, {\edit the} stability of the original model. In general, reduced order models do not guarantee that such properties are preserved \cite{Anonymous:pMn0O0Q4}. 

In the context of Hamiltonian and Lagrangian systems, recent work suggests modifications of POD to preserve {\edit some} geometric structures. Lall et al. \cite{Lall:2003iy} and Carlberg et al. \cite{Carlberg:2014ky} suggests that the reduced-order system should be identified by a Lagrangian function on a low-dimensional configuration space. In this way, the geometric structure of the original system is inherited by the reduced system. {\edit Model reduction for port-Hamiltonian systems can be found in the works of Beattie et al. \cite{Chaturantabut:2016he}, Polyuga et al. \cite{Polyuga:2010gj} and references therein. These works construct a reduced port-Hamiltonian system using Krylov or POD methods that inherit the passivity and stability of the original system.} For Hamiltonian systems, Peng et al. \cite{Peng:2014di}, using a symplectic transformation, constructs a reduced Hamiltonian, as an approximation to the Hamiltonian of the original system. As a result, the reduced system preserves the symplectic structure. Although these methods preserve {\edit the} geometric structure, they use a POD-like approach for constructing the reduced basis and are not well {\edit suited} for problems with a high-dimensional parameter domain.

In this paper, we present a greedy approach for the construction of a reduced system that preserves the geometric structure of Hamiltonian systems. This technique results in a reduced Hamiltonian system that mimics the symplectic properties of the original system and preserves the Hamiltonian structure and its stability over the course of time. On the other hand, since time integration of the original system is only required once per iteration, the proposed method saves substantial computational cost during the offline stage when compared to alternative POD-like approaches. {\edit It is well known that structured matrices, e.g. symplectic matrices, generally are not well-conditioned \cite{Karow:2006cf}. The greedy update of the symplectic basis presented here, yields a orthosymplectic basis and, therefore, a norm bounded basis.} Moreover, we demonstrate that assumptions, natural for the set of all solutions of the original Hamiltonian system under variation of parameters, lead to exponentially fast convergence of the greedy algorithm. For nonlinear Hamiltonian systems, we show how the basis can be combined with the discrete empirical interpolation method (DEIM) {\edit \cite{Chaturantabut:2010cz,Barrault:2004kz}} to enable a fast evaluation of nonlinear terms while maintaining the symplectic structure.

This paper is organized as follows. Section \ref{chap:MoOr:1} presents a brief overview of model order reduction, POD and DEIM. In Section \ref{chap:Hasy:1} we cover the required topics from symplectic geometry and Hamiltonian systems. Section \ref{chap:SyMo:1} discusses the greedy generation of a symplectic reduced basis as well as other SVD-based symplectic model reduction techniques. Accuracy, stability, and efficiency of the greedy method compared to other SVD-based methods are discussed in Section \ref{chap:NuRe:1}. {\edit Finally we offer some conclusive remarks in Section \ref{chap:Con:1}}.


\section{Model Order Reduction for Time Dependent Problems} \label{sec:mor}

Consider a dynamical system of the form
\begin{equation} \label{eq:1}
	\left\{
	\begin{aligned}
		\frac{d}{dt} u(t) &= f(t,u(t)),\\
		u(0) &= u_0.
	\end{aligned}
	\right.
\end{equation}
Here, $u(t),u_0\in \mathbb R^{n}$ and $f: [0,T]\times \mathbb R^{n} \to \mathbb R^{n}$, for some $T<\infty$, is a Lipschitz function. We may apply the \emph{method of lines} \cite{Edsberg:2008:ICM:1477735} to a system of partial differential equations to obtain a dynamical system of the form \eqref{eq:1}. The \emph{solution manifold} for \eqref{eq:1} is defined as
\begin{equation} \label{eq:2}
	\mathcal M_u := \{ u(t) | t \in [0,T] \}.
\end{equation}
When $\mathcal M_u$ has a low-dimensional representation, it is referred to as \emph{reducible}. Assume that $\mathcal M_u$ can be well approximated by $k$-dimensional linear subspace $\mathcal V_k$, with $k\ll n$ and let $E_k = \{ v_1,\dots,v_k \}$ be the basis vectors for $\mathcal V_k$ and $V_k$ the basis matrix that contains these vectors in its columns. A reduced basis (RB) method assumes that $u \approx \tilde u = V_k v$, where $v$ is the expansion coefficients of $\tilde u$ in the basis $E_k$. Substituting this into \eqref{eq:1} yields
\begin{equation} \label{eq:3}
	V_k \frac{d}{dt} v(t) = f(t,V_{k}v) + r(t,u).	
\end{equation}
Here, $r$ is the error vector in this approximation. The Petrov-Galerkin projection of \eqref{eq:1} onto $\mathcal V_k$ requires $r$ to be orthogonal to a $k$-dimensional subspace $\mathcal W_k$. One can construct a projection operator $P_{\mathcal V_{k},\mathcal W_k}$ that projects elements of $\mathbb R^{n}$ onto $\mathcal V_k$, orthogonal to $\mathcal W_k$ as $P_{\mathcal V_{k},\mathcal W_k} = V_k(W_k^TV_k)^{-1}W_k^T$, where $W_k$ is the basis matrix that contains the basis vectors of $\mathcal W_k$ in its columns and $W_k^TV_k$ is assumed to be invertible. With this projection, \eqref{eq:1} reduces to
\begin{equation} \label{eq:4}
	\left\{
	\begin{aligned}
		\frac{d}{dt} v(t) &= (W_k^TV_k)^{-1} f(t,V_{k}v),\\
		v(0) &= (W_k^TV_k)^{-1}u_0.
	\end{aligned}
	\right.
\end{equation}
When we require $W_k=V_k$, then \eqref{eq:4} is referred to as the \emph{Galerkin} projection of \eqref{eq:1} onto $\mathcal V_k$. Since \eqref{eq:4} has a smaller size, as compared to \eqref{eq:1}, one can expect accelerated evaluation. To numerically identify the best possible subspace $\mathcal V_{k}$ we first discretize the solution manifold to obtain
\begin{equation} \label{eq:5}
	\mathcal M_{u}^{\Delta} = \{ u(t_i) | i\in \{ 1,\dots,N_t \} \}.
\end{equation}
Members of $M_{u}^{\Delta}$ are referred to as \emph{snapshots} of \eqref{eq:1}. One can obtain these snapshots by applying a time-integration scheme, e.g. the Runge-Kutta methods, to \eqref{eq:1} to obtain $\tilde {\mathcal M}_{u}^{\Delta}$ an approximation to $\mathcal M_{u}^{\Delta}$. Throughout this paper, we assume that we can choose $\tilde{\mathcal M}_{u}^{\Delta}$ arbitrary close to $\mathcal M_{u}^{\Delta}$, therefore, by an abuse of notation, we may drop the overscript ``\textasciitilde''. For a Galerkin projection, the best possible basis $V_k$ is the one that minimizes the collective projection error \cite{hesthaven2015certified}, i.e., the solution to the minimization problem
\begin{equation} \label{eq:6}
\begin{aligned}
&  \underset{V_k\in\mathbb R^{n\times k}}{\text{minimize}}
& &  \| S - V_kV_k^TS\|_F, \\
& \text{subject to}
& & V_k^TV_k=I_k.
\end{aligned}
\end{equation}
Here $S$ collects vectors in $\mathcal M_{u}^{\Delta}$ in its columns, referred to as the \emph{snapshot matrix}, $\|\cdot \|_F$ is the Frobenius norm \cite{trefethen97}, and $I_{k}$ is the identity matrix of size $k$. Note that the constraint in \eqref{eq:6} requires $V_k$ to be orthonormal. The basis matrix $V_k$ that solves the minimization problem \eqref{eq:6} is referred to as the \emph{proper orthogonal decomposition} (POD) of $S$ of size $k$ \cite{hesthaven2015certified}, and, according to the Schmidt-Mirsky theorem, can be constructed using the left singular vectors of $S$ as
\begin{equation}
	V_{k} = [u_i]_{i=1}^{k}.
\end{equation}
Here $u_i$, for $i=1,\dots k$, are the first $k$ singular vectors of $S$.

Often MOR is studied in the parametric setting, where the vector $u,u_0$, and the right hand side $f$ of \eqref{eq:1} are of the form $u(t;\mu)$, $u_0(\mu)$, and $f(t,u;\mu)$, respectively. Here $\mu$ belongs to $\mathbb P$ a closed subset of $\mathbb R^d$. In this case, the reduced system can approximate the quantities of interest at an accelerated rate. Since the nature of time, as a parameter, is different from other spacial and physical parameters, in this paper we solely focus on $t$ as the parameter. Nevertheless, it is straight forward to extend the results of this paper to the parameter setting by using POD in time and parameter space, or by using the POD-greedy \cite{haasdonk2013convergence,hesthaven2015certified,quarteroni2015reduced} method to generate a basis $V_k$.

Since the approximated solution $\tilde u$ is a linear combination of the POD basis vectors, $\tilde u$ inherits linear properties of these basis vectors. However, when the solution $u$ to \eqref{eq:1} satisfies some nonlinear invariants, there is no guarantee that, in general, $\tilde u$ also satisfy such invariants \cite{doi:10.1137/140959602,doi:10.1137/140978922,doi:10.1137/17M1111991,MaboudiAfkham2018}. This results in a qualitatively wrong and often unstable solution. In the later sections, we discuss how the skew-symmetric formulation of the fluid flow allows conservation of quadratic invariants, e.g. the kinetic energy, at the level of the reduced system.


\section{Hamiltonian Systems and Symplectic Geometry} \label{chap:Hasy:1}
Let $\edit \mathcal M$ be a manifold and $\edit \Omega:\mathcal M \times \mathcal M \to \mathbb R$ be a closed, nondegenerate {\edit and skew-symmetric} 2-form on $\edit \mathcal M$. The pair $\edit (\mathcal M,\Omega)$ is called a \emph{symplectic manifold} {\edit \cite{Marsden:1999ck}}. 


{\edit Let $(\mathcal M,\Omega)$ be a symplectic manifold and suppose that $H:\mathcal M \to \mathbb R$ is a smooth scalar function. The differential of $H$, denoted by $\mathbf dH$, defines a 1-form on $\mathcal M$. {\blue The nondegeneracy of $\Omega$ implies that there is a unique vector field $X_H$, \emph{the Hamiltonian vector field }\cite{da2003introduction,Marsden:1999ck}, on $\mathcal M$ such that}
\begin{equation} \label{eq:Hasy:1}
	\edit i_{X_H} \Omega = \mathbf dH. 
\end{equation}
}
{\blue Here $i_{X_H} \Omega$ is the interior product of $X_H$ with $\Omega$, i.e.,}
\begin{equation}
	\edit \Omega(X_H,Y) = \mathbf dH(Y),
\end{equation}
{\edit for any vector field $Y$ on $\mathcal M$.} Note that when $\edit \mathcal M$ belongs to a Euclidean space then $\mathbf d H = \nabla_z H$. The equations of evolution are then defined by
\begin{equation} \label{eq:Hasy:2}
	\dot z = X_H(z)
\end{equation}
and known as \emph{Hamilton's equation} \cite{Marsden:1999ck}. A fundamental feature of Hamiltonian systems is the conservation of the Hamiltonian along integral curves on $\edit \mathcal M$. To emphasize the importance of this property we recall {\edit \cite{Marsden:1999ck}}

\begin{theorem} \label{theorem:Hasy:1}
Suppose that $X_H$ is a Hamiltonian vector field with the flow $\phi_t$ on a symplectic manifold $\mathcal M$. Then $H\circ \phi_t = H$.
\end{theorem}

\begin{proof}
$H$ is constant along integral curves since
\begin{equation} \label{eq:Hasy:3}
\begin{aligned}
	\frac{d}{dt}(H\circ \phi_t)(z) &= \mathbf d H(\phi_t(z)) \cdot( \frac{d}{dt} \phi_t(z) ) \\
	&= \mathbf d H (\phi_t(z))\cdot X_H(\phi_t(z)) \\
	&= \Omega_z( X_H(\phi_t(z)), X_H(\phi_t(z)) ) = 0,
\end{aligned}
\end{equation}
{\edit by using} the chain rule and bilinearity of $\Omega$ in the argument.
\end{proof}

For the case where the symplectic manifold is also a linear vector space, the pair $({\edit \mathcal M},\Omega)$ is also referred to as a \emph{symplectic vector space}. We {\edit need} the following theorems regarding symplectic vector spaces and refer the reader to \cite{de2006symplectic,Marsden:1999ck,Silva01lectureson} for detailed proofs.

{\edit
\begin{theorem} \label{theorem:Hasy:1.1} \cite{Marsden:1999ck}
If $(V,\Omega)$ is a symplectic vector space then $\Omega$ is a constant form, that is $\Omega_z$ is independent of $z\in V$. 
\end{theorem}
\begin{theorem} \label{theorem:Hasy:1.2} \cite{Marsden:1999ck}
If $(V,\Omega)$ is a finite-dimensional symplectic manifold then $V$ is even dimensional.
\end{theorem}
\begin{theorem} \label{theorem:Hasy:1.3} \cite{de2006symplectic}
(The Symplectic Gram-Schmidt) If $(V,\Omega)$ is a $2n$-dimensional symplectic vector space, then there is a basis $e_1,\dots e_n,f_1, \dots , f_n$ of $V$ such that
\begin{equation} \label{eq:Hasy:4}
\begin{aligned}
	& \Omega(e_i,e_j) = 0 = \Omega(f_i,f_j), \quad & i\neq j,\\
	& \Omega(e_i,f_j) = \delta_{ij}, & i\leq i,j \leq n.
\end{aligned}
\end{equation}
where $\delta$ is the Kronecker's delta function. {\blue Moreover, if $V = \mathbb{R}^{2n}$ then} we can choose basis vectors $\{e_i,f_i\}_{i=1}^n$ such that
\begin{equation} \label{eq:Hasy:5}
	\Omega(v_1,v_2) = v_1^T \mathbb J_{2n} v_2, \qquad v_1,v_2\in \mathbb R^n,
\end{equation}
with $\mathbb J_{2n}$ being the {\blue standard} symplectic matrix, defined as
\begin{equation} \label{eq:Hasy:6}
	\mathbb{J}_{2n} = 
	\begin{pmatrix}
		0_n & I_n \\
		-I_n & 0_n
	\end{pmatrix}.
\end{equation}
Here $I_n$ and $0_n$ is the identity matrix and the zero square matrix of size $n$, respectively.
\end{theorem}
\begin{theorem} \label{theorem:Hasy:1.4} \cite{Marsden:1999ck}
The classical inner product $\langle \cdot,\cdot \rangle:\mathbb R^{2n}\times \mathbb R^{2n}\to \mathbb R$ can be written in terms of the 2-form as
\begin{equation}
	\langle v,u \rangle = \Omega(\mathbb J_{2n}v,u),\quad \forall u,v \in \mathbb R^{2n}.
\end{equation}
\end{theorem}
}

{\edit 
\begin{definition}  \cite{de2006symplectic}
Suppose $(V,\Omega)$ is a finite dimensional symplectic vector space and $E\subset V$ is a subspace. Then the symplectic complement of $E$ inside $V$ is defined as
\[
	E^{\perp} := \{ v\in V |\ \Omega(v,e) = 0,\ \forall e\in E \}
\]
\end{definition}
Note that $E \cap E^{\perp}$ is not empty in general. 
\begin{definition} \cite{de2006symplectic}
Suppose $(V,\Omega)$ is a finite dimensional symplectic vector space. A subspace $E\subset V$ is called a Lagrangian subspace inside $V$ if $E = E^\perp$.
\end{definition}
\begin{theorem} \label{theorem:Hasy:1.5} \cite{abraham1978foundations}
Suppose $(V,\Omega)$ is a finite dimensional symplectic vector space. If $E\subset V$ is a Lagrangian subspace then $dim(E)=\frac 1 2dim(V)$. Here $dim$ denotes the dimension of the subspace.
\end{theorem}
\begin{definition}
A basis of $(V,\Omega)$ is called orthosymplectic if it is both a symplectic basis and an orthogonal basis with respect to the classical scalar product.
\end{definition}
\begin{theorem} \label{theorem:Hasy:1.6}  \cite{mehl2009perturbation,da2003introduction}
Suppose $(V,\Omega)$ is a $2n$ dimensional symplectic vector space and $E\subset V$ is a Lagrangian subspace. Then there is an orthosymplectic basis for $V$.
\end{theorem}
\begin{proof}
{\blue We are going to summarize the proof given in \cite{mehl2009perturbation}.} Starting from a Lagrangain subspace in $E \subset V$ an orthosymplectic basis can be easily constructed. By Theorem \ref{theorem:Hasy:1.5} $E$ is $n$ dimensional. Suppose that $\{ e'_1,\dots, e'_n \}$ is a basis for $E$, using the classical Gram-Schmidt orthogonalization process we can construct an orthonormal basis $\{ e_1,\dots,e_n \}$. Define a new set of vectors $f_1 = \mathbb J_{2n}^Te_1$, $f_2 =\mathbb J_{2n}^T e_2$, $\dots$, $f_n= \mathbb J_{2n}^Te_n$. We have
\begin{equation}
	\langle f_i, f_j \rangle = e_i^T \mathbb J_{2n} {\mathbb J_{2n}}^T e_j = \delta_{ij}, \quad \langle f_i, e_j \rangle = e_i^T \mathbb J_{2n} e_j = 0, \quad i,j=1,\dots,n,
\end{equation}
where we used the fact that $\mathbb J_{2n} {\mathbb J_{2n}}^T = I_{2n}$ in the first identity and the second identity is due to the fact that the basis $\{ e_1,\dots,e_n \}$ forms a Lagrangian subspace. This shows that the set $\{ e_1,\dots,e_n \}\cup \{ f_1,\dots,f_n \}$ forms an orthonormal basis. Also, it can be easily verified that this is a symplectic basis. Thus $\{ e_1,\dots,e_n \}\cup \{ f_1,\dots,f_n \}$ is an orthosymplectic basis.
\end{proof}
\begin{theorem} \label{theorem:Hasy:1.7} \cite{Marsden:1999ck}
On a finite-dimensional symplectic vector space the relationship (\ref{eq:Hasy:1}) becomes 
\begin{equation} \label{eq:Hasy:7}
\left\{
\begin{aligned}
	&\dot {\mathbf z} = \mathbb{J}_{2n} \nabla_{\mathbf z} H(\mathbf z), \\
	& \mathbf z(0) = \mathbf z_0.
\end{aligned}
\right.
\end{equation}
or, by introducing the canonical coordinates $\mathbf z = (\mathbf q^T, \mathbf p^T)^T$,
\begin{equation} \label{eq:Hasy:8}
\left\{
\begin{aligned}
	&\dot {\mathbf q} = \nabla_{\mathbf p} H(\mathbf q,\mathbf p),\\
	&\dot {\mathbf p} = - \nabla_{\mathbf q} H(\mathbf q,\mathbf p).
\end{aligned}
\right.
\end{equation}
\end{theorem}
}

{\edit Let} us now introduce \emph{symplectic transformations}, i.e., mappings between symplectic manifolds which preserve the 2-form $\Omega$. The accurate numerical treatment of Hamiltonian systems often requires preservation of the symmetry expressed in Theorem \ref{theorem:Hasy:1}. Symplectic transformations can be used to construct such symmetry preserving numerical methods. 

{\edit
\begin{definition}
Let $(V,\Omega)$ and $(W,\Pi)$ be two linear symplectic vector spaces of dimensions $2n$ and $2k$, respectively. A linear mapping $\phi:V \to W$ is called \emph{symplectic} or \emph{canonical} if
\begin{equation} \label{eq:Hasy:9}
	\Omega = \phi^* \Pi
\end{equation}
where $\phi^* \Pi$ is the pullback of $\Pi$ by $\phi$, i.e. for all $\mathbf{z}_1, \mathbf{z}_2\in V$
\begin{equation}
	\Omega(\mathbf{z}_1,\mathbf{z}_2) = \Pi(\phi(\mathbf{z}_1),\phi(\mathbf{z}_2)).
\end{equation}
\end{definition}

Note that if we represent the transformation $\phi$ as a matrix $A\in \mathbb R^{2n\times 2k}$ condition (\ref{eq:Hasy:9}) is equivalent to \cite{Marsden:1999ck}}

\begin{equation} \label{eq:Hasy:11}
	A^T \mathbb{J}_{2n}A = \mathbb{J}_{2k}.
\end{equation}
A matrix of size $2n\times 2k$ satisfying (\ref{eq:Hasy:11}) is called a \emph{symplectic matrix}. {\blue We emphasize that a symplectic matrix is conventionally referred to a square matrix, however, here we may allow symplectic matrices to be also rectangular.}

\begin{definition}
	The \emph{symplectic inverse} of a matrix $A\in \mathbb{R}^{2n\times 2k}$ is denoted by $A^+$ and defined by {\edit \cite{Peng:2014di}}
\begin{equation}\label{eq:Hasy:12}
	A^+ := \mathbb{J}_{2k}^T A^T \mathbb{J}_{2n}.
\end{equation}
\end{definition}
We point out the properties of the symplectic inverse and refer the reader to \cite{Peng:2014di} for detailed proof.
\begin{lemma} \label{lemma:Hasy:1}
Let $A\in \mathbb{R}^{2n\times 2k}$ be a symplectic matrix and $A^+$ its symplectic inverse as defined in (\ref{eq:Hasy:12}). Then ${(A^+)}^T$ is a symplectic matrix and $A^+A = I_{2k}$.
\end{lemma}

{\edit A straight-forward calculation verifies} that $AA^+$ is idempotent, i.e., a symplectic projection onto the column span of $A$.

{\edit It is natural to expect a numerical integrator that solves (\ref{eq:Hasy:7}) to also satisfy the conservation law in Theorem \ref{theorem:Hasy:1}. Common numerical integrators e.g., Runge-Kutta methods, do not generally preserve the Hamiltonian which results in a qualitative wrong behavior of the solution \cite{Hairer:1250576}. Symplectic integrators are a class of numerical integrators for Hamiltonian systems that preserve the symplectic structure and ensure stability in long-time integration.} The St\"ormer-Verlet time stepping scheme is an example of symplectic integrators and is given by
\begin{equation} \label{eq:Hasy:13}
\begin{aligned}
	q_{n+1/2} &= q_n + \frac{\Delta t}{2} \nabla_pH(q_{n+1/2},p_n), \\
	p_{n+1} &= p_n - \frac{\Delta t}{2} \left( \nabla_qH(q_{n+1/2},p_n) + \nabla_qH(q_{n+1/2},p_{n+1}) \right),\\
	q_{n+1} &= q_{n+1/2} + \frac{\Delta t}{2} \nabla_pH(q_{n+1/2},p_{n+1}),
\end{aligned}
\end{equation}
and
\begin{equation} \label{eq:Hasy:14}
\begin{aligned}
	p_{n+1/2} &= p_n - \frac{\Delta t}{2} \nabla_qH(q_{n},p_{n+1/2}), \\
	q_{n+1} &= q_n + \frac{\Delta t}{2} \left( \nabla_pH(q_{n},p_{n+1/2}) + \nabla_pH(q_{n+1},p_{n+1/2}) \right),\\
	p_{n+1} &= p_{n+1/2} - \frac{\Delta t}{2} \nabla_qH(q_{n+1},p_{n+1/2}).
\end{aligned}
\end{equation}
For a general Hamiltonian system, the St\"ormer-Verlet scheme is implicit. However, for separable Hamiltonians, i.e. $H(q,p) = K(p) + U(q)$, this {\edit scheme becomes} explicit. We refer the reader to \cite{Hairer:1250576} for more information about the construction and applications of symplectic and geometric numerical integrators. 


\section{Symplectic Model Reduction} \label{chap:SyMo:1}
We now {\edit discuss} how to modify reduced order modeling to ensure that {\edit the resulting scheme preserves} the symplectic structure of the Hamiltonian system. 



Consider a Hamiltonian system (\ref{eq:Hasy:7}) on a $2n$-dimensional symplectic vector space $\edit (V,\Omega)$. Suppose that the solution manifold $\mathcal M_H$ is well approximated by a low dimensional symplectic subspace $\edit (W,\Omega)$ of dimension $2k$ $(k\ll n)$. We can {\edit then} construct a symplectic basis $A$ for $\edit W$ and approximate the solution to (\ref{eq:Hasy:7}) as
\begin{equation} \label{eq:SyMo:1}
	\mathbf z \approx A\mathbf y.
\end{equation}
Substituting this into (\ref{eq:Hasy:7}) we obtain
\begin{equation} \label{eq:SyMo:2}
	A\mathbf y = \mathbb{J}_{2n} \nabla_{\mathbf z} H(A \mathbf y). 
\end{equation}
Multiplying both sides with the symplectic inverse of $A$ and using the chain rule we have
\begin{equation} \label{eq:SyMo:3}
	\mathbf y = A^+ \mathbb J_{2n} (A^+)^T \nabla_{\mathbf y} H(A\mathbf y).
\end{equation}
Since $A$ is a symplectic basis, Lemma \ref{lemma:Hasy:1} ensures that $(A^+)^T$ is a symplectic matrix i.e., $A^+ \mathbb J_{2n} (A^+)^T = \mathbb{J}_{2k}$. By defining the reduced Hamiltonian $\tilde H:\mathbb R^{2k} \to \mathbb R$ as $\tilde H (y) = H(Ay)$ we obtain the reduced system
\begin{equation} \label{eq:SyMo:4}
\left\{
\begin{aligned}
	 \frac{d}{dt} \mathbf y &= \mathbb J_{2k} \nabla_{\mathbf y} \tilde H(\mathbf y), \\
	 \mathbf y_0 &= A^+ \mathbf z_0.
\end{aligned}
\right.
\end{equation}
The system obtained from the Petrov-Galerkin projection in (\ref{eq:MoOr:5}) is not a Hamiltonian system and does not guarantee conservation of the symplectic structure. On the other hand, we observe that the reduced system in (\ref{eq:SyMo:4}) is of the form (\ref{eq:Hasy:7}) and, hence, is a Hamiltonian system, i.e. the symplectic structure will be conserved along integral curves of (\ref{eq:SyMo:4}). Note that the original and the reduced systems are {\edit endowed with} different Hamiltonians. In the next proposition we show that the error in the Hamiltonian is constant in time. 


\begin{proposition}
Let $\mathbf{z} (t)$ be the solution of (\ref{eq:Hasy:7}) at time $t$. Further suppose that $\tilde{\mathbf{z}} (t)$ is the approximate solution of the reduced system (\ref{eq:SyMo:4}) in the original coordinate system. Then the error in the Hamiltonian defined by
\begin{equation} \label{eq:SyMo:5}
	\Delta H(t)  = |H(\mathbf z(t)) - H(\tilde{\mathbf z}(t))|,
\end{equation}
is constant for all $t\in \mathbb R$.
\end{proposition}

\begin{proof}
	Let $\phi_t$ and $\psi_t$ be the Hamiltonian flow of the original and the reduced system respectively. By definition $\mathbf z(t) = \phi_t(\mathbf z_0)$ and $\mathbf y(t) = \psi_t(\mathbf y_0)$. Using the definition of the reduced Hamiltonian and Theorem \ref{theorem:Hasy:1} we have
\begin{equation} \label{eq:SyMo:6}
\begin{aligned}
	H(\tilde{\mathbf{z}} (t)) = H( A\mathbf y (t) ) = \tilde H(\mathbf y (t)) = \tilde H(\psi_t(\mathbf y_0)) = \tilde H(\mathbf y_0) = \tilde H(A^+ \mathbf z_0) = H(AA^+\mathbf z_0).
\end{aligned}
\end{equation}
The error in the Hamiltonian can then be written in terms of $\mathbf z_0$ and the symplectic basis $A$ as
\begin{equation} \label{eq:SyMo:7}
	\Delta H(t) = |H(\mathbf z_0) - H(AA^+\mathbf z_0)|
\end{equation}
\end{proof}

{\edit 
The following theorems provide a strong indication of the stability of the reduced system. 

\begin{definition} \label{definition:SyMo:1} \cite{bhatia2002stability}
Consider a dynamical system of the form $\dot{\mathbf z} = \mathbf f(\mathbf z)$ and suppose that $\mathbf z_e$ is an equilibrium point for the system so that $\mathbf f(\mathbf z_e) = 0$. $\mathbf z_e$ is called nonlinearly stable or Lyapunov stable if, for any $\epsilon > 0$, we can find $\delta > 0$ such that for any trajectory $\phi_t$, if $\| \phi_0 - \mathbf z_e \|_2 \leq \delta$, then for all $0 \leq t < \infty$, we have $\| \phi_t - \mathbf z_e \|_2 < \epsilon$, where $\| \cdot \|_2$ is the Euclidean norm.
\end{definition}	
The following proposition, also known as Dirichlet's theorem \cite{bhatia2002stability}, states the sufficient condition for an equilibrium point to be Lyapunov stable. We refer the reader to \cite{bhatia2002stability} for the proof.
\begin{proposition} \label{proposition:SyMo:1} \cite{bhatia2002stability}
An equilibrium point $\mathbf z_e$ is Lyapunov stable if there exists a scalar function $W : \mathbb R^{n} \to  \mathbb R$ such that $\nabla W(\mathbf z_e) = 0$, $\nabla^2 W(\mathbf z_e)$ is positive definite, and that for any trajectory $\phi_t$ defined in the neighborhood of $\mathbf z_e$, we have $\frac{d}{dt} W(\phi_t) \leq 0$. Here $\nabla^2W$ is the Hessian matrix of $W$.
\end{proposition}
The scalar function $W$ is referred to as the \emph{Lyapunov function}. In the context of the Hamiltonian systems, a suitable candidate for the Lyapunov function is the Hamiltonian function $H$. The following theorem shows that when $H$ (or $-H$) is a Lyapunov function, then the equilibrium points of the original and the reduced system are Lyapunov stable \cite{abraham1978foundations}. 
\begin{theorem} \label{theorem:SyMo:1}
Consider a Hamiltonian system of the form (\ref{eq:Hasy:7}) together with the reduced system (\ref{eq:SyMo:4}). Suppose $\mathbf z_e$ is an equilibrium point for (\ref{eq:Hasy:7}) and that $\mathbf y_e = A^+\mathbf z_e$. If $H$ (or $-H$) is a Lyapunov function satisfying Proposition \ref{proposition:SyMo:1}, then $\mathbf z_e$ and $\mathbf y_e$ are Lyapunov stable equilibrium points for (\ref{eq:Hasy:7}) and (\ref{eq:SyMo:4}), respectively. 
\end{theorem}
\begin{proof}
	It is a direct consequence of Proposition \ref{proposition:SyMo:1} that $\mathbf z_e$ is a local minimum or maximum of (\ref{eq:Hasy:7}) and also a Lyapunov stable point. It can be easily checked that if $\mathbf z_e$ is a local minimum of $H$ then $\mathbf y_e$ is a local minimum for $\tilde H$ and an equilibrium point for (\ref{eq:SyMo:4}). Also from the chain rule we have
\[
	\nabla^2_{\mathbf y} \tilde H = A^T \nabla^2_{\mathbf z} H A.
\]
So for any $\xi\in \mathbb R^{2k}$
\[
	\xi^T \nabla^2_{\mathbf y} \tilde H \xi = (A\xi)^T \nabla^2_{\mathbf z} H (A\xi) \geq 0.
\]
Here the last inequality is due to the positive definiteness of $H$. Therefore $\tilde H$ is also positive definite. By Proposition \ref{proposition:SyMo:1} we conclude that $\mathbf y_e$ is a Lyapunov stable point.
\end{proof}
}

While the symplectic structure is not guaranteed to be preserved in the reduced systems obtained by the Petrov-Galerkin projection, the reduced system obtained by the symplectic projection guarantees the preservation of the energy up to the error in the Hamiltonian (\ref{eq:SyMo:5}). In the next section we discuss  different methods for obtaining a symplectic basis.

\subsection{Proper Symplectic Decomposition (PSD)} \label{chap:SyMo.PrSy:1}

Similar to Section \ref{chap:MoOr.PrOr:1} we gather snapshots $\mathbf z_i = [q_i^T , p_i^T]^T$ in the snapshot matrix $S$. Suppose that a symplectic basis $A$ of size $2n\times2k$ and its symplectic inverse $A^+$ is provided. {\edit The Proper Symplectic Decomposition} requires that the error of the symplectic projection onto the symplectic subspace {\edit be minimized}. Hence, the PSD symplectic basis of size $2k$ is the solution to the optimization problem

\begin{equation} \label{eq:SyMo:8}
\begin{aligned}
& \underset{V\in \mathbb R^{2n\times 2k}}{\text{minimize}}
& & \| S - AA^+S\|_F \\
& \text{subject to}
& & A^T \mathbb{J}_{2n}A = \mathbb{J}_{2k}
\end{aligned}
\end{equation}
Compared to POD, in (\ref{eq:SyMo:8}) the orthogonal projection is replaced with a symplectic projection $AA^+$. At first, the minimization looks similar to the one obtained by POD. {\edit However, it is well known that symplectic bases are not generally orthogonal, and therefore not norm bounded. This means that numerical errors may become dominant in the symplectic projection \cite{Karow:2006cf} which makes the minimization (\ref{eq:SyMo:8}) a harder problem than (\ref{eq:MoOr:6}).}
	
As the optimization problem (\ref{eq:SyMo:8}) is nonlinear, the direct solution is usually expensive. A simplified version of the optimization (\ref{eq:SyMo:8}) can be found in \cite{Peng:2014di}, but there is no guarantee that the method provides a near optimal basis. 

{\edit Finding eigen-spaces of Hamiltonian and symplectic matrices is studied in the context of optimal control problems \cite{Benner:2000ww,Benner:1997ef,Watkins:2004kz,BunseGerstner:1986dg} and model reduction of Riccati equations \cite{Benner:1997ef}, where also an SVD-like decomposition for Hamiltonian and symplectic matrices has been proposed \cite{Xu:2003kx}. However, the computation of a large snapshot matrix and use of the mentioned methods to compute its eigen-spaces, is usually computationally demanding. Also, these methods generally do not guarantee the construction of a well-conditioned symplectic basis.
	
The greedy approach presented in Section \ref{Chap:Symo.PrSy:3} is an iterative method for construction of a symplectic basis. It avoids the evaluation of the full snapshot matrix, hence substantially reduces the computational cost in the offline stage of the symplectic model reduction. Also, by construction, it yields a orthosymplectic basis and therefore a well-conditioned basis.

In Section \ref{chap:SyMo.PrSy:2} we briefly outline non-direct methods for finding solutions to (\ref{eq:SyMo:8}), proposed by \cite{Peng:2014di}, and assuming a specific structure for $A$. In Section \ref{Chap:Symo.PrSy:3} we introduce a greedy approach for the symplectic basis generation.}

\subsubsection{SVD Based Methods for Symplectic Basis Generation} \label{chap:SyMo.PrSy:2} 


\paragraph{\bf Cotangent lift} Suppose that $A$ is of the form
\begin{equation} \label{eq:SyMo:9}
	A = 
	\begin{pmatrix}
		\Phi & 0 \\
		0 & \Phi
	\end{pmatrix},
\end{equation}
where $\Phi \in \mathbb{R}^{n\times k}$ is an orthonormal matrix. It is easy to check that $A$ is a symplectic matrix, i.e., $A^T \mathbb J_{2n} A = \mathbb J_{2k}$. The construction of $A$ suggests that the range of $\Phi$ should cover both the potential and the momentum spaces. Hence, we can construct $A$ by forming the combined snapshot matrix
\begin{equation} \label{eq:SyMo:10}
	S_{\text{combined}} = [q_1,\dots,q_n,p_1,\dots,p_n], \qquad \mathbf z_i = (q_i^T,p_i^T)^T,
\end{equation}
and define $\Phi=[u_1,\dots,u_k]$, where $u_i$ is the $i$-th left singular vector of $S_{\text{combined}}$. It is shown in \cite{Peng:2014di} that among all symplectic bases of the form (\ref{eq:SyMo:9}) cotangent lift minimizes the projection error.


\paragraph{\bf Complex SVD} Suppose instead that $A$ takes the form \cite{Peng:2014di}
\begin{equation} \label{eq:SyMo:11}
	A = 
	\begin{pmatrix}
		\Phi & -\Psi \\
		\Psi & \Phi
	\end{pmatrix},
\end{equation}
while $\Phi$ and $\Psi$ are real matrices of size $n\times k$ satisfying conditions
\begin{equation} \label{eq:SyMo:12}
\Phi^T \Phi + \Psi^T \Psi = I_k,\quad \Phi^T \Psi = \Psi^T \Phi.
\end{equation}
It can be checked that $A$ forms a symplectic matrix. To construct $A$ we first define the complex snapshot matrix
\begin{equation} \label{eq:SyMo:13}
	S_{\text{complex}} = [ q_1 + i p_1, \dots , q_N + i p_N ].
\end{equation}
Each left singular vector of $S_{\text{complex}}$ now takes the form $u_m = r_m + i s_m$. We define
\begin{equation} \label{eq:SyMo:14}
	 \Phi = [r_1,\dots, r_k], \quad \Psi = [s_1,\dots, s_k].
\end{equation}
One can easily check that (\ref{eq:SyMo:12}) is satisfied {\edit since the matrix of singular vectors is unitary}. It is shown in \cite{Peng:2014di} that among all symplectic bases of the form (\ref{eq:SyMo:11}) the complex SVD minimizes the projection error.

\subsubsection{The Greedy Approach to Symplectic Basis Generation} \label{Chap:Symo.PrSy:3} Greedy generation of the reduced basis is an iterative procedure which, in each iteration, adds the two best possible basis vectors to the symplectic basis to enhance overall accuracy. In contrast to the cotangent lift and the complex SVD methods, the greedy approach does not require the symplectic basis to have a specific structure. This typically results in a more compact basis and/or more accurate reduced systems. For parametric problems, the greedy approach only requires one numerical solution to be computed per iteration hence saving substantial computational cost in the offline stage. 

{\edit The orthonormalization step is an essential step in most greedy approaches for basis generation in the context of model reduction \cite{Anonymous:2016wl,Quarteroni:2016wi}. However common orthonormalization processes, e.g. the QR method, destroy the symplectic structure of the original system \cite{BunseGerstner:1986dg}. Here we use a variation of the QR method known as the SR \cite{Salam2014} method which is based on the symplectic Gram-Schmidt method and yields a symplectic basis. 
}

{\edit
As discussed in Section \ref{chap:Hasy:1}, any finite dimensional symplectic linear vector space has a symplectic basis that satisfies conditions (\ref{eq:Hasy:4}). Further, Theorem \ref{theorem:Hasy:1.6} provides an iterative process for constructing an orthosymplectic basis \cite{Matsuo:2014wl,Salam2014}. To briefly describe the SR method, suppose that an orthosymplectic basis
\begin{equation} \label{eq:SyMo:14.1}
	A_{2k}=\{ e_1 , \dots , e_k \} \cup \{ \mathbb J_{2n}^T e_1 , \dots , \mathbb J_{2n}^T e_k \},
\end{equation}
and a vector $z\not \in \text{span}(A_{2k})$ is provided. We aim to symplectically orthogonalize ($\mathbb J_{2n}$-orthogonalize) $z$ with respect to $A_{2k}$ and seek $\alpha_1,\dots,\alpha_k,\beta_1,\dots,\beta_k \in \mathbb R$ such that
\begin{equation} \label{eq:SyMo:14.2}
	\Omega\left(z + \sum_{i=1}^k \alpha_i e_i + \sum_{i=1}^k \beta_i \mathbb J_{2n}^Te_i , \sum_{i=1}^k \bar{\alpha}_i e_i + \sum_{i=1}^k \bar{\beta}_i \mathbb J_{2n}^Te_i \right) = 0,
\end{equation}
for all possible $\bar{\alpha}_1,\dots,\bar{\alpha}_k,\bar{\beta}_1,\dots,\bar{\beta}_k \in \mathbb R$. It is easily seen that the unique solution is 
\begin{equation} \label{eq:SyMo:14.3}
	\alpha_i = - \Omega(z,\mathbb J_{2n}^Te_i), \quad \beta_i = \Omega(z,e_i),
\end{equation}
for $i=1,\dots,k$. Now define the modified vectors as
\begin{equation} \label{eq:SyMo:14.4}
	\tilde z = z - \sum_{i=1}^k \Omega(z,\mathbb J_{2n}^Te_i) e_i + \sum_{i=1}^k \Omega(z,e_i) \mathbb J_{2n}^Te_i.
\end{equation}
If we introduce $e_{k+1} = \tilde z / \| \tilde z \|_2$, it is easily checked that $e_{k+1}$ is also orthogonal to $A_{2k}$ with respect to the classical inner product. Therefore span$\{e_1,\dots,e_{k+1}\}$ forms a Lagrangian subspace and according to Theorem \ref{theorem:Hasy:1.6} the basis $A_{2k+2}= A_{2k}\cup \{ e_{k+1} , \mathbb J_{2n}^T e_{k+1} \}$ forms an orthosymplectic basis.

Note that the $SR$ method can be replaced with backward stable routines such as the isotropic Arnoldi or the isotropic Lanczos methods \cite{Mehrmann:2000dv}.
}

The key element of the greedy algorithm is the availability of an error function which evaluates the error associated with the model reduction \cite{Anonymous:2016wl}. In the framework of symplectic model reduction, one possible candidate is the error in the Hamiltonian (\ref{eq:SyMo:5}). Correctly approximating symplectic systems relies on preservation of the Hamiltonian, hence the error in the Hamiltonian {\edit arises as a} a natural choice. Moreover, since the error in the Hamiltonian depends on the initial condition and the reduced symplectic basis, evaluation of the error does not require the time integration of the full system. 

Suppose that a $2k$-dimensional {\edit orthosymplectic basis (\ref{eq:SyMo:14.1})} is generated at the $k$-th step of the greedy method and we seek to enrich it by two additional vectors. Using the error in the Hamiltonian (\ref{eq:SyMo:7}) we search the parameter space to identify the value that maximizes the error in the Hamiltonian
\begin{equation} \label{eq:SyMo:14.5}
	\omega_{k+1} := \underset{\omega\in \Gamma}{\text{argmax }}\Delta H(\omega).
\end{equation}
The goal is to approximate the Hamiltonian function as well as possible. 

We then propagate (\ref{eq:Hasy:7}) in time to produce trajectory snapshots 
\begin{equation}
	S=\{ \mathbf z(t_i,\omega_{k+1}) | i = 1,\dots,M \}.
\end{equation} 
The next basis vector is the snapshot that maximises the projection error (\ref{eq:SyMo:8})
{\edit 
\begin{equation} \label{eq:SyMo:14.6}
	z := \underset{s\in S}{\text{argmax }} \| s - A_{2k}{A_{2k}}^+s \|.
\end{equation}
}
Finally, we update the basis as
{\edit
\begin{equation} \label{eq:SyMo:14.7}
	e_{k+1} = \tilde z, \quad A_{2k+1} = A_{2k}\cup \{ e_{k+1} , \mathbb J_{2n}^Te_{k+1} \},
\end{equation}
}
where $\tilde z$ is the vector obtained {\edit after} applying the symplectic Gram-Schmidt process to $z$. 

Since the maximization over the entire parameter space $\Gamma$ is impossible, we discretize the parameter set into a grid with $N$ points: $\Gamma_N = \{ \omega_1,\dots,\omega_N\}$. However, since the selection of parameters only require the evaluation of the error in the Hamiltonian and not time integration of the original system, then $\Gamma_N$ can be chosen {\edit to be} very rich.

We summarize the greedy algorithm for the generation of a symplectic basis in Algorithm \ref{alg:SyMo:3}.




\begin{algorithm} 
\caption{The greedy algorithm for generation of a symplectic basis} \label{alg:SyMo:3}
{\bf Input:} Tolerated loss in the Hamiltonian $\delta$, parameter set $\Gamma_N = \{\omega_1,\dots,\omega_N\}$, initial condition $\mathbf z_0(\omega)$
\begin{enumerate}
\item $\omega^* \leftarrow \omega_1$
\item $e_1 \leftarrow \mathbf z_0(\omega^*)$
\item $A \leftarrow [e_1,-\mathbb J^T_{2n}e_1]$
\item $k \leftarrow 1$
\item \textbf{while} $\Delta H(\omega) > \delta$ for all $\omega \in \Omega_N$
\item \hspace{0.5cm} $w^* \leftarrow$ $\underset{\omega\in \Omega_N}{\text{argmax }}\Delta H(\omega)$
\item \hspace{0.5cm} Compute trajectory snapshots $S=\{ \mathbf z(t_i,\omega^*) | i = 1,\dots,M \}$
\item \hspace{0.5cm} $\mathbf z^* \leftarrow$ $\underset{s\in S}{\text{argmax }} \| s - AA^+s \|$
\item \hspace{0.5cm} Apply symplectic Gram-Schmidt on $\mathbf z^*$
\item \hspace{0.5cm} $e_{k+1} \leftarrow \mathbf z^*/ \| \mathbf  z^*\|$
\item \hspace{0.5cm} $A \leftarrow [e_1,\dots ,e_{k+1} , \mathbb J^T_{2n}e_1,\dots,\mathbb J^T_{2n}e_{k+1}]$
\item \hspace{0.5cm} $k \leftarrow k+1$
\item \textbf{end while}
\end{enumerate}
\vspace{0.5cm}
{\bf Output:} Symplectic basis $A$.
\end{algorithm}

%The convergence and convergence rate of the greedy algorithm with respect to Kolmogorov $n$-width is investigated in Section \ref{chap:SyMo.PrSy:3}. We finish this section by stating the advantage of using the greedy algorithm over the POD-based methods:
%\begin{itemize}
%\item Lack of a symplectic linear algebra toolbox for solving the minimization in (\ref{eq:SyMo:8}) requires a possibly very expensive nonlinear optimizations to recover near optimal symplectic bases. Alternative approaches for finding solutions to (\ref{eq:SyMo:8}) requires assumptions on the structure of the symplectic basis. The greedy algorithm on the other hand, is a cheap alternative to the nonlinear optimization and it allows flexibility in terms of the structure.
%\item The greedy method reduces the total cost of the offline stage by solving the original system (\ref{eq:Hasy:7}) only $k$ times to obtain a reduced system of size $2k$, unlike POD-based methods that require time integration of (\ref{eq:Hasy:7}) for the entire parameter space.
%\item If enrichment of the symplectic basis is required, the greedy algorithm can add new basis vectors. The POD-based methods, are bound to re-compute the basis for the refined parameter set.
%\item POD-based methods, e.g., the cotangent lift and the complex SVD, require performing SVD decompositions on possibly very large matrices. The greedy algorithm avoids this.
%\end{itemize}






\subsubsection{Convergence of the Greedy Method} \label{chap:SyMo.PrSy:3}

To show convergence of the greedy method we {\edit consider} a slightly different version based on the projection error. The error in the Hamiltonian is then introduced as a cheap surrogate to the projection error to accelerate the parameter selection.

Suppose that we are given a compact subset $S$ of $\mathbb R^{2n}$. Our intention is to find a set of vectors $A=\{e_1,\dots,e_k,f_1,\dots,f_k\}$ such that $A$ forms {\edit an orthosymplectic} basis and any $s\in S$ is well approximated by elements of the subspace span$(A)$. The modified greedy method for generating basis vectors $e_i$ and $f_i$ is as follows. In the initial step we pick $e_1$ such that $\edit \|e_1\|_2 = \max_{s\in S} \|s\|_2$. Then define $f_1 = \mathbb{J}_{2n}^T e_1$. It is easy to check that the span of $A_2 = \{e_1,f_1\}$ is {\edit orthosymplectic}, so $A_2$ is the first subspace that approximates elements of $S$. In the $k$-th step of the greedy method, suppose we have a basis $A_{2k} = \{ e_1,\dots, e_k , f_1,\dots ,f_k \}$. We define $P_{2k}$ to be a symplectic projection operator that projects elements of $S$ onto span$(A_{2k})$ and define
\begin{equation} \label{eq:new1}
	\sigma_{2k}(s) := \|s-P_{2k}(s)\|_2,
\end{equation}
as the projection error. Moreover we denote by $\sigma_{2k}$ the maximum approximation error of $S$ using elements in span$(A_{2k})$ as
\begin{equation} \label{eq:new2}
	\sigma_{2k} := \max_{s\in S} \sigma_{2k}(s).
\end{equation}
The next set of basis vectors in the greedy selection are
\begin{equation} \label{eq:new3}
	e_{k+1} := \underset{s\in S}{\text{argmax }}\sigma_{2k}(s), \quad f_{k+1} := \mathbb{J}^T e_{k+1}.
\end{equation}
We emphasisze that the sequence of basis vectors generated by the greedy is generally not unique. 

To estimate the quality of the reduced subspace, it is natural to compare it with the best possible $2k$-dimensional subspace in the sense of the minimum projection (not necessary symplectic) error. For this we introduce the Kolmogorov $n$-width \cite{Kolmogoroff:1936fj,Pinkus:1985vy}.

\begin{definition}
Let $S$ be a subset of $\mathbb R^{m}$ and $Y_n$, $n\leq m$, be a general $n$-dimensional subspace of $\mathbb R^{m}$. The angle between $S$ and $Y_n$ is given by
\begin{equation} \label{eq:new4}
	E(S,Y_n) := \sup_{s\in S} \inf_{y\in Y_n} \|s-y\|_2.
\end{equation}
The Kolmogorov $n$-width of $S$ in $\mathbb R^m$ is given by
\begin{equation} \label{eq:new5}
	d_{n}(S,\mathbb{R}^m) := \inf_{Y_n} E(S,Y_n) = \inf_{Y_n} \sup_{s\in S} \inf_{y\in Y_n} \|s-y\|_2
\end{equation}
\end{definition}

For a given subspace $Y_n$, the angle between $S$ and $Y_n$ measures the worst possible projection error of elements in $S$ onto $Y_n$. Hence the Kolmogorov $n$-width quantifies how well $S$ can be approximated by {\edit an} $n$-dimensional subspace. 

We seek to show that the decay of $\sigma_{2k}$, obtained by the greedy algorithm, has the same rate as of $d_{2k}(S)$, i.e., the greedy method provides the best possible accuracy attained by a $2k$-dimensional subspace.

We start by symplectifying the vectors provided by the greedy algorithm as
\begin{equation} \label{eq:new6}
\begin{aligned}
	& \xi_1 = e_i, & \bar{\xi}_1 = \mathbb{J}_{2n}^T \xi_1, &\\
	& \xi_i = e_i - P_{2(i-1)} (e_i), & \bar{\xi}_i = \mathbb{J}_{2n}^T, \xi_i &\quad i = 2,3,\dots
\end{aligned}
\end{equation}
The projection of a vector $s\in S$ onto span$(A_{2k})$ can be written using the symplectic basis as
\begin{equation} \label{eq:new7}
	P_{2k}(s) = \sum_{i=1}^k \left( \alpha_i(s) \xi_i + \bar{\alpha}_i(s) \bar{\xi}_i \right),
\end{equation}
where $\alpha_i(s)$ and $\bar{\alpha}_i(s)$ for $i=1,\dots,k$ are the expansion coefficients
\begin{equation} \label{eq:new8}
	\alpha_i(s) = - \frac{\Omega(\bar{\xi}_i,s)}{\Omega(\xi_i,\bar{\xi}_i)}, \quad \bar{\alpha}_i(s) = \frac{\Omega(\xi_i,s)}{\Omega(\xi_i,\bar{\xi}_i)},
\end{equation}
for any $s\in S$. Since $\bar{\xi}_i$ is symplectic to span$(A_{2(k-1)})$ we have
\begin{equation} \label{eq:new9}
\begin{aligned}
	|\alpha_i(s)| = \frac{|\Omega(\bar{\xi}_i,s)|}{|\Omega(\xi_i,\bar{\xi}_i)|} = \frac{|\Omega( \bar{\xi}_i, s - P_{2(k-1)}(s))|}{|\Omega(\xi_i,\bar{\xi}_i)|}  &\leq \frac{\|\bar{\xi}_i\|_2 \| s - P_{2(k-1)}(s) \|_2}{ \|\xi_i\|_2 \|\bar{\xi}_i\|_2 } \\
	&= \frac{\| s - P_{2(k-1)}(s) \|_2}{\| e_i - P_{2(k-1)}(e_i) \|_2} \leq 1.
\end{aligned}
\end{equation}
Here, we use the fact that $\edit |\Omega(\xi_i,\bar{\xi}_i)| = \| \xi_i \|^2_2 = \|\bar{\xi}_i\|^2_2$ with the last inequality following from the greedy algorithm which maximizes $e_i$. Similarly we deduce that $|\bar{\alpha}_i(s)|\leq 1$.

We write
\begin{equation} \label{eq:new10}
\begin{aligned}
	\xi_j = \sum_{i=1}^j \left( \mu_i^j e_i + \gamma_i^j f_i \right), \quad \bar{\xi}_j = \sum_{i=1}^j \left( \lambda_i^j e_i + \eta_i^j f_i,  \right), \quad j=1,2,\dots
\end{aligned}
\end{equation}
with
\begin{equation} \label{eq:new11}
\begin{aligned}
	&\mu^j_j = 1, \quad \gamma^j_j = 0, \\
	&\mu_i^j = \sum_{l=i}^{j-1}\left( - \alpha_l(f_j) \mu_i^l  + \bar{\alpha}_l(f_j) \gamma_i^l \right), \quad\gamma_i^j = \sum_{l=i}^{j-1}\left( - \alpha_l(f_j) \gamma_i^l  + \bar{\alpha}_l(f_j) \mu_i^l \right), \\
	&\lambda^j_i = - \gamma ^j_i, \quad \eta^j_i = \mu^j_i,
\end{aligned}
\end{equation}
for $j=2,3,\dots$. By induction and using the bound in (\ref{eq:new9}) we deduce that
\begin{equation} \label{eq:new12}
	\mu^j_i,\gamma^j_i,\lambda^j_i,\eta^j_i \leq 3^{j-i}, \quad \text{for } j\geq i.
\end{equation}
Now let $2k$ be the dimension of the desired reduced space. Looking at the definition of Kolmogorov $n$-width we observe that for any $\theta > 1$ we can find a subspace $Y_{2k}$ such that $E(S,Y_{2k}) \leq \theta d_{2k}(S,\mathbb R^n)$. Hence we can find vectors $v_1,\dots,v_k,u_1,\dots,u_k\in Y_{2k}$ such that
\begin{equation} \label{eq:new13}
\begin{aligned}
	& \|e_i - v_i\|_2 \leq \theta d_{2k}(S,\mathbb R^n), \\
	& \|f_i - u_i\|_2 \leq \theta d_{2k}(S,\mathbb R^n).
\end{aligned}
\end{equation}
Now we construct a set of $2(k+1)$ new vectors
\begin{equation} \label{eq:new14}
\begin{aligned}
	& \zeta_j = \sum_{i=1}^{k+1} \mu_i^j v_i + \gamma^j_i u_i,\quad \bar{\zeta}_j = \sum_{i=1}^{k+1} \lambda_i^j v_i + \eta^j_i u_i.
\end{aligned}
\end{equation}
for $j = 1,\dots,k+1$. Note that since $u_i$ and $v_i$ belong to $Y_{2k}$ so does their linear combination including all $\zeta_j$ and $\bar{\zeta}_j$. We can use the inequality (\ref{eq:new12}) to write
\begin{equation}
	\| \xi_i - \zeta_i \|_2 \leq 3^i \theta d_{2k}(S,\mathbb R^n),\quad \| \bar{\xi}_i - \bar{\zeta}_i \|_2 \leq 3^i \theta d_{2k}(S,\mathbb R^n).
\end{equation}
Moreover since $Y_{2k}$ is of dimension $2k$ we find $\kappa_i$, $i=1,\dots,2(k+1)$ such that
\begin{equation} \label{eq:new15}
	\sum_{i=1}^{2(k+1)} \kappa_i^2 = 1, \quad\sum_{i=1}^{k+1} \kappa_i \zeta_i + \sum_{i=1}^{k+1} \kappa_{i+k+1} \bar{\zeta}_i = 0.
\end{equation}
We have
\begin{equation} \label{eq:new17}
\begin{aligned}
	\left\| \sum_{i=1}^{k+1} \kappa_i \xi_i + \sum_{i=1}^{k+1} \kappa_{i+k+1} \bar{\xi}_i \right\|_2 &= \left\| \sum_{i=1}^{k+1} \kappa_i (\xi_i - \zeta_i) + \sum_{i=1}^{k+1} \kappa_{i+k+1} (\bar{\xi}_i-\bar{\zeta}_i) \right\|_2 \\
	&\leq 2\cdot 3^{k+1} \sqrt{2(k+1)} \theta d_{2k}(S,\mathbb R^n).
\end{aligned}
\end{equation}
We know there exists $1 \leq j\leq 2k+2$ such that $\kappa_j > 1/\sqrt{2(k+1)}$. Without loss of generality let us assume that $j\leq k+1$. This yields
\begin{equation} \label{eq:new18}
	\left\| \xi_j +  \kappa_j^{-1} \sum_{i=1,i\neq j}^{k+1} \kappa_i \xi_i + \kappa_j^{-1}\sum_{i=1}^{k+1} \kappa_{i+k+1} \bar{\xi}_i \right\|_2 \leq 4\cdot 3^{k+1} (k+1) \theta d_{2k}(S,\mathbb R^n).
\end{equation}
Define $c = \kappa_j^{-1} \sum_{i=1,i\neq j}^{k+1} \kappa_i \xi_i + \kappa_j^{-1}\sum_{i=1}^{k+1} \kappa_{i+k+1} \bar{\xi}_i$ and note that $\mathbb{J}_{2n}^T c$ is symplectic to $\xi_j$. We recover
\begin{equation} \label{eq:new19}
\begin{aligned}
	\| \xi_j \|_2 &\leq \| \xi_j \|_2 + \| c \|_2 = \Omega(\xi_j,\mathbb{J}_{2n}^T \xi_j) + \Omega(c,\mathbb{J}_{2n}^T c) \\
	       &= \Omega(\xi_j,\mathbb{J}_{2n}^T \xi_j) + \Omega(c,\mathbb{J}_{2n}^T c) + \Omega(\xi_j,\mathbb{J}_{2n}^T c) + \Omega(c,\mathbb{J}_{2n}^T \xi_j) \\
       	&= \Omega(\xi_j + c, \mathbb{J}^T_{2n} (\xi_j + c)) = \| \xi_j + c \|_2	
\end{aligned}
\end{equation}
Combining this with (\ref{eq:new18}) yields
\begin{equation} \label{eq:new20}
	\| \xi_j \|_2 \leq 4\cdot 3^{k+1} (k+1) \theta d_{2k}(S,\mathbb R^n).
\end{equation}
Finally using the definition of $\xi_j$ for all $s\in S$ we have
\begin{equation} \label{eq:new21}
	\| s - P_{2(j-1)}(s) \|_2 \leq \| f_j - P_{2(j-1)}(f_j) \|_2 = \|\xi_j \|_2 \leq 4\cdot 3^{k+1} (k+1) \theta d_{2k}(S,\mathbb R^n)
\end{equation}
Hence, for any given $\lambda > 1$
\begin{equation} \label{eq:new22}
	\| s - P_{2k}(s) \|_2 \leq \| s - P_{2(j-1)}(s) \|_2 \leq 4\cdot 3^{k+1} (k+1) \theta d_{2k}(S,\mathbb R^n).
\end{equation}
This establishes the following theorem.
\begin{theorem} \label{theorem:SyMo:2}
	Let $S$ be a compact subset of $\mathbb{R}^{2n}$ with exponentially small Kolmogorov $n$-width $\edit d_{k}\leq c\exp(-\alpha k)$ with $\alpha > \log3$. Then there exists $\beta>0$ such that the symplectic subspaces $A_{2k}$ generated by the greedy algorithm provide exponential approximation properties such that
\begin{equation} \label{eq:new23}
	\| s - P_{2k}(s) \|_2 \leq C \exp(-\beta k)
\end{equation}
for all $s\in S$ and some $C>0$.
\end{theorem}

%{\edit
%\begin{theorem} \label{theorem:SyMo:3}
%	The basis $A_{2k}$ generated by the greedy algorithm is orthosymplectic.
%\end{theorem}
%\begin{proof}
%	Direct result of theorem \ref{theorem:Hasy:1.6}.
%\end{proof}
%}

\subsection{Symplectic Discrete Empirical Interpolation Method (SDEIM)} Consider the Hamiltonian system (\ref{eq:Hasy:7}) and its reduced system (\ref{eq:SyMo:4}) equipped with a symplectic transformation $A$. One can split the Hamiltonian function $H = H_1 + H_2$ such that $\nabla H_1 = L\mathbf z$ and $\nabla H_2 = \mathbf g(\mathbf z)$, where $L$ is a constant matrix in $\mathbb R^{n\times n}$ and $\mathbf g$ is a nonlinear function. The reduced system takes the form
\begin{equation} \label{eq:new24}
	\frac{d}{dt} \mathbf y = \underbrace{A^+ \mathbb J_{2n} L A}_{\tilde L} \mathbf y + A^+ \mathbb J_{2n} \mathbf g(A\mathbf y)
\end{equation}
As discussed in Section \ref{chap:MoOr.DEIM:1}, the complexity of evaluating the nonlinear term still depends on $n$, the size of the original system. To overcome this computational bottleneck we use the DEIM approximation for evaluating the nonlinear function $\mathbf g$ as
\begin{equation} \label{eq:new25}
	\frac{d}{dt} \mathbf y = \tilde L \mathbf y + \underbrace{ A^+ \mathbb J_{2n} V (P^TV)^{-1} P^T \mathbf g(A\mathbf y) }_{\tilde N(\mathbf y)}
\end{equation}
For a general choice of $V$ the system (\ref{eq:new25}) is not guaranteed to be a Hamiltonian system, impacting long time accuracy and stability. However, we can guarantee that (\ref{eq:new25}) is a Hamiltonian system by choosing $V=(A^+)^T$. To {\edit see this}, we note that the system (\ref{eq:new25}) is a Hamiltonian system if and only if $\tilde N(\mathbf y) = \mathbb J_{2k} \nabla_{\mathbf y} \mathbf g(\mathbf y)$. Also we have 
\begin{equation} \label{eq:new26}
	\mathbf g(A\mathbf y) = \nabla_{\mathbf z} H_2(\mathbf z) = (A^+)^T \nabla_{\mathbf y} H_2(A \mathbf y),
\end{equation}
where the chain rule is used for the second equality. Substituting this into $\tilde N$ we obtain
\begin{equation} \label{eq:new27}
	\tilde N(\mathbf y)= A^+ \mathbb J_{2n} V (P^TV)^{-1} P^T  (A^+)^T \nabla_{\mathbf y} H_2(A \mathbf y).
\end{equation}
Taking $V = (A^+)^T$ yields
\begin{equation} \label{eq:new28}
	\tilde N(\mathbf y) = A^+ \mathbb J_{2n}(A^+)^T \nabla_{\mathbf y} H_2(A \mathbf y) = \mathbb J_{2k} \nabla_{\mathbf y} H_2(A \mathbf y),
\end{equation}
since $(A^+)^T$ is a symplectic matrix. Hence, $V = (A^+)^T$ is a sufficient condition for (\ref{eq:new25}) to {\edit be Hamiltonian}. 

Regarding the construction of the projection space, suppose that we have already constructed a symplectic basis $A=\{ e_1,\dots , e_k,f_1,\dots f_k \}$ using the greedy algorithm. Note that $(A^+)^T$ is a symplectic basis and $(A^+)^+=A$. Thus, we can move between these two symplectic bases by simply using the transpose operator and the symplectic inverse operator. Let $S_{\mathbf g} = \{ \mathbf g (\mathbf x(t_i,\omega_j)) \}$ with $i = 1,\dots,M$ and $ j = 1 ,\dots,N$ be the nonlinear snapshots that were gathered in the greedy algorithm. We then form $(A^+)^T = \{ e'_1,\dots, e'_k,f'_1,\dots,f'_k\}$ and use a greedy approach to add new basis vectors to $(A^+)^T$. At the $i$-th iteration of the symplectic DEIM, we use $(A^+)^T$ to approximate elements in $S_{\mathbf g}$ and choose the vector that maximizes the error as the next basis vector 
\begin{equation}
	s^* := \underset{s \in S_{\mathbf g}}{\text{argmax }}\| s - (A^+)^T A^+ s \|_2.	
\end{equation}
After applying the symplectic Gram-Schmidt on $s^*$, we update $(A^+)^T$ as
\begin{equation}
\begin{aligned}
	e'_{k+i+1} = \frac{s^*}{\| s^* \|_2},\quad f'_{k+i+1} = \mathbb J_{2n}^T e'_{k+i+1}.
\end{aligned}
\end{equation}
Finally when $(A^+)^T$ approximates elements $S_{\mathbf g}$ with the desired accuracy, we transpose and symplectically invert $(A^+)^T$ to obtain $A$. We summarize the symplectic DEIM algorithm in Algorithm \ref{alg:SyMo:4}.



\begin{algorithm} 
\caption{Symplectic Discrete Empirical Interpolation Method} \label{alg:SyMo:4}
{\bf Input:} Symplectic basis $A=\{ e_1,\dots,e_k,f_1,\dots,f_k \}$, nonlinear snapshots $S_{\mathbf g} = \{ \mathbf g(\mathbf x(t_i,\omega_j)) \}$ and tolerance $\delta$
\begin{enumerate}
\item Compute $(A^+)^T = \{ e'_1,\dots,e'_k,f'_1,\dots,f'_k \}$
\item $i \leftarrow 1$
\item \textbf{while} max$\| s - (A^+)^T A^+s \| > \delta$ for all $s\in S_{\mathbf g}$
\item \hspace{0.5cm} $s^* \leftarrow \underset{s \in S_{\mathbf g}}{\text{argmax }}\| s - (A^+)^T A^+ s \|$
\item \hspace{0.5cm} Apply symplectic Gram-Schmidt on $s^*$
\item \hspace{0.5cm} $e'_{k+i} = s^* / \| s^* \|$
\item \hspace{0.5cm} $f'_{k+i} = \mathbb J_{2n} e'_{k+i}$
\item \hspace{0.5cm} $(A^+)^T \leftarrow [e'_1,\dots,e'_{k+i},f'_1,\dots,f'_{k+i}]$
\item \hspace{0.5cm} $i\leftarrow i+1$
\item \textbf{end while}
\item take transpose and symplectic inverse of $(A^+)^T$
\end{enumerate}
\vspace{0.5cm}
{\bf Output:} Symplectic basis $A$ that guarantees a Hamiltonian reduced system.
\end{algorithm}

When using an implicit time integration scheme we face inefficiencies when evaluating the Jacobian of nonlinear terms, as discussed in Section \ref{chap:MoOr.DEIM:1}. We recall that the key to fast approximation of the Jacobian is that the interpolating index matrix $P$, obtained in the DEIM approximation, commutes with the nonlinear function. Nonlinear terms in Hamiltonian systems often take the from
\begin{equation}
	\mathbf g (\mathbf z) = \mathbf g (\mathbf q,\mathbf p) = 
	\begin{pmatrix}
		g_1(q_1,p_1) \\
		g_2(q_2,p_2) \\
		\vdots \\
		g_{2n}(q_{n},p_{n})
	\end{pmatrix}.
\end{equation}
Thus, the interpolating index matrix, obtained by Algorithm \ref{alg:MoOr:1} does not necessarily commute with the function $\mathbf g$. To overcome this, when index $\mathfrak p_i$ with $\mathfrak p_i\leq n$ or $\mathfrak p_i>n$ is chosen in Algorithm \ref{alg:MoOr:1} we also include $\mathfrak p_i + n$ or $\mathfrak p_i-n$, respectively. {\edit Simple calculations verifies that $\mathbf g$ and $P$ commute.}


\section{Numerical Results} \label{chap:NuRe:1}
In this section, we illustrate the performance of the greedy generation of a symplectic basis. The parametric linear wave equation is considered to compare SVD based methods with the greedy method. {\blue The symplectic model reduction of nonlinear systems} is then illustrated by considering the parametric nonlinear Schr\"odinger equation. {\edit Finally we discuss the numerical convergence of the greedy method introduced in Algorithm \ref{alg:SyMo:3}.}

\subsection{Parametric Linear Wave equation} \label{chap:NuRe:1.1} Consider the {\edit parametric} linear wave equation
\begin{equation} \label{eq:NuRe:1}
\left\{
\begin{aligned}
& u_{tt}(x,t,\omega) = \kappa(\omega) u_{xx}(x,t,\omega), \\
& u(x,0) = u^0(x),
\end{aligned}
\right.
\end{equation}
where $x$ belongs to a one-dimensional torus of length $L$, $\omega = (\omega_1,\dots,\omega_4)$ and
\begin{equation} \label{eq:NuRe:2}
	\kappa(\omega) = c^2\left( \sum_{l=1}^4 \frac{1}{l^2} \omega_l \right).
\end{equation}
{\edit Here $\omega_l \in [0,1]$ for $l=1,\dots,4$} and $c\in \mathbb{R}$ is a constant number. By rewriting (\ref{eq:NuRe:1}) in canonical form, using the change of variable $q = u$ and $\partial q/ \partial t= p$, we obtain the symplectic form
\begin{equation} \label{eq:NuRe:3}
\left\{
\begin{aligned}
& q_t(x,t,\omega) = p(x,t,\omega), \\
& p_t(x,t,\omega) = \kappa(\omega) q_{xx}(x,t,\omega),
\end{aligned}
\right.
\end{equation}
with the associated Hamiltonian
\begin{equation} \label{eq:NuRe:4}
	H(q,p,\omega) = \frac 1 2 \int_0^L p^2 + \kappa(\omega) q_x^2 \ dx.
\end{equation}
We discretize the torus into $N$ equidistant points and define $\Delta x = L/N$, $x_i = i\Delta x$, $q_i=q(t,x_i,\omega)$ and $p_i=p(t,x_i,\omega)$ for $i = 1, \dots, N$. Furthermore, we discretize (\ref{eq:NuRe:3}) using a standard central finite differences scheme to obtain
\begin{equation} \label{eq:NuRe:5}
	\frac{d}{dt} \mathbf z = \mathbb{J}_{2N} L\mathbf z,
\end{equation}
where $\mathbf z=(q,\dots,q_N,p_q,\dots,p_n)^T$ and
\begin{equation} \label{eq:NuRe:6}
L = 
\begin{pmatrix}
	I_n & 0_N \\
	0_N & \kappa(\omega)D_{xx}
\end{pmatrix},\quad 
\end{equation}
with $D_{xx}$ the central finite differences matrix operator. The discrete Hamiltonian can finally be written as
\begin{equation} \label{eq:NuRe:7}
	H_{\Delta x}(\mathbf z) = \frac{\Delta x}2 \sum_{i=1}^{N} \left( p_i^2 + \kappa(\omega) \frac{(q_{i+1} - q_i)^2}{2\Delta x ^ 2} + \kappa(\omega) \frac{(q_{i} - q_{i-1})^2}{2\Delta x ^ 2} \right).
\end{equation}
The initial condition is given by
\begin{equation} \label{eq:NuRe:8}
	q_i(0) = h( 10\times|x_i - \frac{1}{2}| ), \quad p_i = 0, \quad i=1,\dots,N
\end{equation}
where $h(s)$ is the cubic spline function
\begin{equation} \label{eq:NuRe:9}
h(s) = 
\left\{
\begin{aligned}
& 1 - \frac{3}{2}s^2 + \frac{3}{4}s^3, \quad & 0\leq s \leq 1, \\
& \frac{1}{4}(2-s)^3, & 1< s \leq 2, \\
& 0, & s > 2.
\end{aligned}
\right.
\end{equation}
This will result in waves propagating in both directions on the torus.

For numerical time integration we {\edit use} the Str\"omer-Verlet (\ref{eq:Hasy:13}) scheme, {\edit which is explicit since} the Hamiltonian is separable for the linear wave-equation. The full model uses the following parameter set \vspace{0.5cm}
\begin{center}
\begin{tabular}{|l|l|}
\hline
Domain length & $L = 1$ \\
No. grid points & $N = 500$ \\
Space discretization size & $\Delta x = 0.002$ \\
Time discretization size & $\Delta t = 0.01$ \\
Wave speed & $c^2 = 0.1$ \\
\hline
\end{tabular}
\end{center}
\vspace{0.5cm}
We compare the reduced system obtained by the greedy algorithm with the methods based on SVD. To generate snapshots, we discretize the parameter space $[0,1]^4$ into in total of $5^4$ equidistant grid points. For the SVD based methods and POD, snapshots are gathered in the snapshot matrices $S$, $S_{\text{combined}}$ and $S_{\text{complex}}$, respectively, and the SVD is performed to construct the reduced basis. The greedy method is applied following Algorithm \ref{alg:SyMo:3}; as input, the tolerance for the error in the Hamiltonian is set to $\delta = 5 \times 10^{-3}$. All reduced systems are taken to have an identical size ($k=80$ for POD and $k=40$ for the symplectic methods). We use the {\edit Str\"omer-Verlet} scheme for symplectic methods and a second order Runge-Kutta method for the POD. {\edit The choice of different time integration routines is due to the fact that the POD destroys the canonical form of the original equations and a symplectic integrator cannot be applied. One can alternatively use separate reduced subspaces for the potential and the momentum spaces, which however is not a standard model reduction approach and requires further analysis.} Finally we use transformation (\ref{eq:SyMo:1}) to transfer the solution of the reduced systems into the high-dimensional space for illustration purposes.



We reduced the cost by 50\% in the offline stage when using the greedy method as compared to SVD-based methods (cotangent lift and complex SVD method). This happens because the SVD-based methods require time integration of the full system for all discrete parameter points, while the greedy method picks a number of parameters from the parameter space. 


%We summarized the duration of offline and online stages of different model reduction methods in the table below.
%
%\vspace{0.5cm}
%\begin{center}
%\begin{tabular}{c|cc}
% & offline (seconds) & online (seconds) \\
%\hline
%full system & - & 0.19 \\
%cotangent lift & 61.4 & 0.11 \\
%complex SVD & 61.4 & 0.11 \\
%greedy & 33.7 & 0.11 \\
%\end{tabular}
%\end{center}
%\vspace{0.5cm}
%
%In our implementations the offline stage of cotangent lift and the complex SVD are substantially longer compared to the greedy method. This is because SVD based methods need to explore the entire parameter space while the greedy method only hand pick a number of parameters from the parameter space. We see that the online stage of all reduced systems are faster than solving the full model. Of course efficiency of reduced systems depend on the implementation. 



\begin{figure}

\begin{minipage}{.5\linewidth}
\centering
\subfloat[$t=0$]{\label{fig:NuRe:1a}\includegraphics[width=1\textwidth]{./figs/wave/solution/solution_t0}}
\end{minipage}%
\begin{minipage}{.5\linewidth}
\centering
\subfloat[$t=1$]{\label{fig:NuRe:1b}\includegraphics[width=\textwidth]{./figs/wave/solution/solution_t1}}
\end{minipage}\par\medskip
\centering
\subfloat[$t=2$]{\label{fig:NuRe:1c}\includegraphics[width=0.5\textwidth]{./figs/wave/solution/solution_t2}}
\caption{The solution $q$ at $t=0$, $t=1$ and $t=2$ of the linear wave equation for parameter value $c= 0.1019$ different from training parameters. Here, the solution of the full system together with the solution of the POD, cotangent lift, complex SVD and the greedy reduced system is shown.}
\label{fig:NuRe:1}
\end{figure}

\begin{figure}

\begin{minipage}{.5\linewidth}
\centering
\subfloat[]{\label{fig:NuRe:2c}\includegraphics[width=\textwidth]{./figs/wave/error}}
\end{minipage}%
\begin{minipage}{.5\linewidth}
\centering
\subfloat[]{\label{fig:NuRe:2b}\includegraphics[width=\textwidth]{./figs/wave/hamiltonian}}
\end{minipage}\par\medskip
\centering
%\subfloat[]{\label{fig:NuRe:2c}\includegraphics[width=0.5\textwidth]{./figs/wave/error}}

\caption{{\edit (a) The $L^2$-error between the solution of the full system and the reduced system for different model reduction methods for $t \in [0,30]$.} (b) Plot of the Hamiltonian function for $t \in [0,30]$. }
\label{fig:NuRe:2}
\end{figure}



Figure \ref{fig:NuRe:1a} shows the solution of the linear wave equation for parameter values {\edit $(\omega_1,\omega_2,\omega_3,\omega_4) = (0.8456,0.1320,0.9328,0.5809)$ or $\kappa(\omega) = 0.1019$}, chosen to be different from training parameters, at $t=0$, $t = 1$ and $t=2$. While we see instability and divergence from the exact solution for the POD reduced system, the symplectic methods provide a good approximation of the full model. 

{\edit The decay of the singular values for the POD are shown in Figure \ref{fig:NuRe:5a}. The decay} of the singular values suggests that a low dimensional solution manifold indeed exists. However, since the linear subspace, constructed by the POD, is not symplectic, we observe blow up of the Hamiltonian function in Figure \ref{fig:NuRe:2b} and the instability of the solution in Figure \ref{fig:NuRe:1}. The symplectic methods (using a reduced basis of the same size as POD) preserve the Hamiltonian function as shown in Figure \ref{fig:NuRe:2b}.

Figure \ref{fig:NuRe:2c} shows the $L^2$-error between the solution of the full model and the reduced systems constructed by different methods. We note that the error for the POD reduced system rapidly increases, confirming that {\edit the} projection based reduced system does not yield a stable solution. Furthermore, the symplectic methods provide a better approximation since the geometric structure of the original system is preserved. Although the greedy method is almost {\edit twice faster than the SVD-based methods} in the offline stage, its accuracy is comparable. {\edit The cotangent lift method provides a more accurate solution, on the other hand the cotangent lift basis (\ref{eq:SyMo:9}) takes a less general form and usually computationally more demanding than the greedy method.}

{\edit For complex systems were the solution of the full system is expensive and for high dimensional parameter domains, POD-based methods become impractical \cite{Anonymous:2016wl,Quarteroni:2016wi}. However, the greedy method requires substantially fewer (proportional to the size of the reduced basis) evaluation of the time integration of the original system.}


%\begin{figure*}[t]
%\begin{center}
%\begin{tabular}{cc}
%	\includegraphics[width=0.4\textwidth]{./figs/wave/solution/solution_t0} &
%	\includegraphics[width=0.4\textwidth]{./figs/wave/solution/solution_t1} \\
%	$t=0$ & $t = 1$
%\end{tabular}
%\end{center}
%\end{figure*}

%\begin{figure}[t]
%\centering
%\begin{tabular}{cc}
%\includegraphics[width=0.4\textwidth]{./figs/wave/solution/solution_t2} \\
%	$t=2$
%\end{tabular}
%\caption{The solution $q$ at $t=0$, $t=1$ and $t=2$ of the linear wave equation for parameter value $c= 0.1019$ different from training parameters. Here the solution of full system together with the solution of the POD, cotangent lift, complex SVD and the greedy reduced system is presented.}\label{fig:NuRe:3}
%\end{figure}



%\begin{figure*}[t]
%\begin{center}
%\begin{tabular}{cc}
%	\includegraphics[width=0.5\textwidth]{./figs/wave/singular} &
%	\includegraphics[width=0.5\textwidth]{./figs/wave/hamiltonian} \\
%	(a) & (b)
%\end{tabular}
%\end{center}
%\end{figure*}

%\begin{figure}[t] 
%\begin{center}
%\begin{tabular}{c}
%	\includegraphics[width=0.5\textwidth]{./figs/wave/error} \\
%	(c)
%\end{tabular}
%\end{center}
%\caption{(a) The decay of singular values for POD, cotangent lift and complex SVD methods. (b) The $L^2$ error between the solution of th full system and the reduced system for different model reduction methods for $t \in [0,30]$. (c) Plot of the Hamiltonian function for $t \in [0,30]$.}\label{fig:NuRe:4}
%\end{figure}


%\begin{figure*}[t]
%  \centering
%  \includegraphics[width=0.5\textwidth]{./figs/wave/hamiltonian}
%  \label{fig:testfig}
%\end{figure*}



\subsection{Nonlinear Schr\"odinger equation} \label{chap:NuRe:1.2} Let us consider the one-dimensional parametric Schr\"odinger equation
\begin{equation} \label{eq:NuRe:10}
\left\{
\begin{aligned}
	& i u_t(t,x,\epsilon) = - u_{xx}(t,x,\epsilon) - \epsilon |u(t,x,\epsilon)|^2 u(t,x,\epsilon),\\
	& u(0,x) = u_0(x),
\end{aligned}
\right.
\end{equation}
where $u$ is a complex valued wave function, $i$ is the imaginary unit, $|\cdot|$ is the modulus operator and $\epsilon$ is a parameter that belongs to the interval $\Gamma = [0.9,1.1]$. We consider periodic boundary conditions, i.e., $x$ belongs to a one-dimensional torus of length $L$. We consider the initial condition
\begin{equation} \label{eq:NuRe:11}
	u_0(x) = \frac{\sqrt 2}{\cosh(x - x_0)} \exp(i\frac{c(x-x_0)}{2}),
\end{equation}
for a positive constant $c$. In quantum mechanics, the quantity $|u(t,x)|^2$ represents the probability of finding the system in state $x$ at time $t$. For the choice of $\epsilon = 1$, $|u(x,t)|$ becomes a solitary wave, and the initial condition will be transported in the positive $x$ direction with a constant speed. For other choices of $\epsilon$, the solution comprises an ensemble of solitary waves, moving in either direction \cite{Faou:2012vh}. 

By introducing the real and imaginary variables $u = p + iq$, we can rewrite (\ref{eq:NuRe:10}) in canonical form as
\begin{equation} \label{eq:NuRe:12}
\left\{
\begin{aligned}
 q_t &= p_{xx} + \epsilon (q^2+p^2)p, \\
 p_t &= -q_{xx} - \epsilon (q^2 + p^2)q,
\end{aligned}
\right.
\end{equation}
with the Hamiltonian function
\begin{equation} \label{eq:NuRe:13}
	H(q,p) = \int_{0}^{L} (q_x^2 + p_x^2) + \frac \epsilon 2 (q^2 + p^2)^2\ dx.
\end{equation}
We discretize the torus into $N$ equidistant points and take $\Delta x = L/N$, $x_i = i\Delta x$, $q_i=q(t,x_i,\epsilon)$ and $p_i = p(t,x_i,\omega)$ for $i = 1 ,\dots,N$. A central finite differences scheme is used to discretize (\ref{eq:NuRe:12}) as
\begin{equation}  \label{eq:NuRe:14}
	\frac{d}{dt} \mathbf z = \mathbb J_{2N} L\mathbf z + \mathbb J_{2N} \mathbf g(\mathbf z).
\end{equation}
Here $\mathbf z = (q_1,\dots,q_N,p_1,\dots,p_n)^T$ and
\begin{equation}  \label{eq:NuRe:15}
	L = 
	\begin{pmatrix}
		D_{xx} & 0_N \\
		0_N & D_{xx}
	\end{pmatrix}.
\end{equation}
Here $\mathbf g$ is a vector valued nonlinear function defined as
\begin{equation}  \label{eq:NuRe:16}
	\mathbf g(\mathbf z) =
	\begin{pmatrix}
	(q_1^2 + p_1^2)q_1 \\
	\vdots \\
	(q_N^2 + p_N^2)q_N \\
	(q_1^2 + p_1^2)p_1 \\
	\vdots \\
	(q_N^2 + p_N^2)p_N
	\end{pmatrix}.
\end{equation}
We discretize the Hamiltonian to obtain
\begin{equation}  \label{eq:NuRe:17}
	H_{\Delta x}(\mathbf z) = {\Delta x}\sum_{i=1}^{N} \left( \frac{q_i q_{i-1} - q_i^2}{\Delta x ^2} + \frac{p_i p_{i-1} - p_i^2}{\Delta x ^2} + \frac \epsilon 4 (p_i^2 + q_i^2)^2  \right),
\end{equation}
and use a Str\"omer-Verlet (\ref{eq:Hasy:13}) scheme for time integration. Since the Hamiltonian function (\ref{eq:NuRe:17}) is non-separable, this scheme becomes implicit so in each time iteration, a system of nonlinear equations is solved using Newton's iteration. We summarize the physical and numerical parameters for the full model in the following table

\vspace{0.5cm}
\begin{center}
\begin{tabular}{|l|l|}
\hline
Domain length & $L = 2\pi /l$ \\
Domain scaling factor & $l = 0.11$ \\
wave speed & $c =1$\\
No. grid points & $N = 256$ \\
Space discretization size & $\Delta x = 0.2231$ \\
Time discretization size & $\Delta t = 0.01$ \\
\hline
\end{tabular}
\end{center}
\vspace{0.5cm}
Regarding computation of the nonlinear terms of reduced systems, {\edit we} compare the DEIM with the symplectic DEIM. For generation of the DEIM reduced basis we apply Algorithm \ref{alg:MoOr:1} to the set of nonlinear snapshots. Algorithm \ref{alg:SyMo:4} is used to construct a reduced basis appropriate for the symplectic DEIM. As input, we provide the symplectic basis generated by Algorithm \ref{alg:SyMo:3} with the set of nonlinear snapshots and a tolerance for the error $\delta = 10^{-4}$.

We compare the reduced system obtained using the greedy algorithm with the cotangent lift, the complex SVD, DEIM, the symplectic DEIM and also the POD. For the SVD-based methods, we discretize the parameter space $[0.9,1.1]$ into $M=500$ equidistant grid points {\edit across} the discrete parameter space $\Gamma_M = \{\epsilon_1,\dots,\epsilon_M \}$, and gather trajectory snapshots for each $\epsilon_i$ for $i = 1,\dots,M$ in the snapshots matrix $S$. All reduced systems are taken to have identical sizes ($k=90$ for the symplectic methods and $k=180$ for the POD method). Following Algorithm \ref{alg:SyMo:3} we construct the reduced system using the same discrete parameter space $\Gamma_M$. The tolerance for the error in the Hamiltonian is set to $\delta = 10^{-3}$. Moreover, for DEIM and symplectic DEIM, we construct bases of size $k'=80$. Note that the reduced system, generated in the symplectic DEIM, will be of size $k+k'=170$.

The cost of the offline stage is reduced to 20\% when using the greedy method for constructing a symplectic basis of size $k=90$, as compared to the SVD-based methods. The online stage, i.e., time integration for a new parameter in $\Gamma$, is generally more than 3 times faster than {\edit for} the original system. We point out that the efficiency of reduced systems are implementation and platform dependent {\edit and we expect further reduction as the size of the problem increases.}

%The duration of the online and offline stages for different model reduction techniques are summarized in table below. The offline time for the cotangent lift and the complex SVD consist of time integration of the full model for all the parameters in the discrete parameter space $\Omega_M$. This is while the greedy method requires substantially shorter offline stage to construct a reduced basis of size $k = 90$. For a reduced system of size $k=90$, all methods almost have the same online duration. The efficiency of reduced systems compared to the full system are heavily implementation dependent, nevertheless solving the reduced systems with our implementation is more than 3 times faster than solving the original system.

%\vspace{0.5cm}
%\begin{center}
%\begin{tabular}{c|cc}
% & offline (hours) & online (seconds) \\
%\hline
%full system & - & 187 \\
%cotangent lift & 25.8 & 59$\pm 1$ \\
%complex SVD & 25.8 & 59$\pm 1$ \\
%greedy & 4.6 & 59$\pm 1$ \\
%greedy + SDEIM & 4.7 & ???
%\end{tabular}
%\end{center}
%\vspace{0.5cm}


%\begin{figure*}[t]
%\begin{center}
%\begin{tabular}{cc}
%	\includegraphics[width=0.5\textwidth]{./figs/schrodinger/solution/solution_t0} &
%	\includegraphics[width=0.5\textwidth]{./figs/schrodinger/solution/solution_t10} \\
%	$t = 0$ & $t = 10$
%\end{tabular}
%\end{center}
%\end{figure*}
%
%\begin{figure}[t] 
%\begin{center}
%\begin{tabular}{c}
%	\includegraphics[width=0.5\textwidth]{./figs/schrodinger/solution/solution_t20}\\
%	$$t = 20$$
%\end{tabular}
%\end{center}
%\caption{The solution $|u(t,x)| = \sqrt{q^2 + p^2}$ at $t=0$, $t=10$ and $t=20$ of the Nonlinear Schr\"odinger equation for parameter value $\epsilon = 1.0932$. Here the solution of full system together with the solution of the POD, cotangent lift, complex SVD and the greedy reduced system is presented.}\label{fig:NuRe:1}
%\end{figure}
%
%\begin{figure*}[t]
%\begin{center}
%\begin{tabular}{cc}
%	\includegraphics[width=0.5\textwidth]{./figs/schrodinger/singular} &
%	\includegraphics[width=0.5\textwidth]{./figs/schrodinger/hamiltonian} \\
%	(a) & (b)
%\end{tabular}
%\end{center}
%\end{figure*}
%
%\begin{figure}[t] 
%\begin{center}
%\begin{tabular}{c}
%	\includegraphics[width=0.5\textwidth]{./figs/schrodinger/error} \\
%	(c)
%\end{tabular}
%\end{center}
%\caption{(a) The decay of singular values for POD, cotangent lift and complex SVD methods. (b) The $L^2$ error between the solution of th full system and the reduced system for different model reduction methods for $t \in [0,30]$. (c) Plot of the Hamiltonian function for $t \in [0,30]$.}\label{fig:NuRe:2}
%\end{figure}

  
\begin{figure}

\begin{minipage}{.5\linewidth}
\centering
\subfloat[$t=0$]{\label{fig:NuRe:3a}\includegraphics[width=1\textwidth]{./figs/schrodinger/solution/solution_t0}}
\end{minipage}%
\begin{minipage}{.5\linewidth}
\centering
\subfloat[$t=10$]{\label{fig:NuRe:3b}\includegraphics[width=\textwidth]{./figs/schrodinger/solution/solution_t10}}
\end{minipage}\par\medskip
\centering
\subfloat[$t=20$]{\label{fig:NuRe:3c}\includegraphics[width=0.5\textwidth]{./figs/schrodinger/solution/solution_t20}}
\caption{The solution $|u(t,x)| = \sqrt{q^2 + p^2}$ at $t=0$, $t=10$ and $t=20$ of the Nonlinear Schr\"odinger equation for parameter value $\epsilon = 1.0932$. Here the solution of {\edit the} full system, together with the solution of the POD, cotangent lift, complex SVD and the greedy reduced system, is shown.}
\label{fig:NuRe:3}
\end{figure}

\begin{figure}

\begin{minipage}{.5\linewidth}
\centering
\subfloat[]{\label{fig:NuRe:4c}\includegraphics[width=\textwidth]{./figs/schrodinger/error}}
\end{minipage}%
\begin{minipage}{.5\linewidth}
\centering
\subfloat[]{\label{fig:NuRe:4b}\includegraphics[width=\textwidth]{./figs/schrodinger/hamiltonian}}
\end{minipage}\par\medskip
\centering
%\subfloat[]{\label{fig:NuRe:4c}\includegraphics[width=0.5\textwidth]{./figs/schrodinger/error}}

\caption{{\edit (a) Plot of the Hamiltonian function for $t \in [0,30]$.} (b) The $L^2$ error between the solution of the full system and the reduced system for different model reduction methods for $t \in [0,30]$. }
\label{fig:NuRe:4}
\end{figure}


Figure \ref{fig:NuRe:3} shows the solution of the Schr\"odinger equation for parameter value $\epsilon = 1.0932$ at $t=0$, $t=10$ and $t=20$. We first compare the reduced system obtained by the greedy algorithm with the POD, the cotangent lift, and the complex SVD method. The size of the reduced systems are taken identical for all methods ($k=180$ for POD and $k=90$ for the rest). Although the decay of the singular values in Figure \ref{fig:NuRe:5b} suggests that the accuracy of the POD reduced system should be comparable to that of the other methods, we observe instabilities in the solution at $t=10$. The greedy, the cotangent lift and the complex SVD method, on the other hand, generate a stable reduced system that accurately approximates the solution of the full model.

{\edit In Figure \ref{fig:NuRe:4b} we observe that the symplectic methods preserve the Hamiltonian function, unlike the POD and the DEIM methods. We emphasise that using the reduced basis, obtained by the greedy, together with the DEIM (purple line) does not preserve the symplectic structure as suggested in this figure.}

Figure \ref{fig:NuRe:4c} illustrates the $L^2$-error between the solution of the full model with the reduced systems, generated by different methods. We first observe that symplectic methods yield a lower computational error {\edit when} compared to non-symplectic methods. Secondly, we observe that although the reduced systems from the cotangent lift and the complex SVD are of the same size, their accuracy is different by an order of magnitude. We notice that the greedy algorithm is slightly less accurate than the cotangent lift method while its offline computational cost is reduced to 20\% {\edit when compared to the cotangent lift}. Lastly we notice that the combination of the greedy reduced basis and DEIM yields large errors in the solution while the solution using the symplectic DEIM is very accurate. We note that the symplectic DEIM is even more accurate than the greedy itself since it has been enriched by the nonlinear snapshots. 

\subsection{Numerical Convergence} {\edit In this section we discuss the numerical convergence of the symplectic greedy method introduced in Section \ref{chap:SyMo:1}. The exponential convergence properties of the conventional greedy \cite{Quarteroni:2016wi} is presented in \cite{Buffa:2012iz,Binev:2011fj}. Theorem \ref{theorem:SyMo:2} suggests that the symplectic greedy method has similar properties. To illustrate this we compare the convergence of the conventional greedy with the convergence of the symplectic greedy method through the numerical simulations in Sections \ref{chap:NuRe:1.1} and \ref{chap:NuRe:1.2}. 

The decay of the singular values of the snapshot matrix for the parametric wave equation and the nonlinear Schr\"odinger equation are given in Figure \ref{fig:NuRe:5}. The decay rate of the singular values is a strong indicator for the decay rate of the Kolmogorov $n$-width of the solution manifold. We expect that the conventional greedy method and the symplectic greedy method provide a similar rate in the decay of the error.
	
Figure \ref{fig:NuRe:5} shows the maximum $L^2$ error between the original system and the reduced system at each iteration of different greedy methods. In this figure we find the conventional greedy with orthogonal projection error as a basis selection criterion (orange), the symplectic greedy method with a symplectic projection error as a basis selection criterion (green), and the symplectic greedy method with energy loss $\Delta H$ as a basis selection criterion (red).

It is observed that the decay rate of the error for greedy with the orthogonal projection and the greedy with the symplectic projection is similar to the decay of the singular values. This matches our expectation from Theorem \ref{theorem:SyMo:2}. We also notice that the greedy method with the loss in Hamiltonian provides an excellent error indication as a basis selection criterion.
}


\begin{figure}

\begin{minipage}{.5\linewidth}
\centering
\subfloat[]{\label{fig:NuRe:5a}\includegraphics[width=\textwidth]{./figs/wave/convergence}}
\end{minipage}%
\begin{minipage}{.5\linewidth}
\centering
\subfloat[]{\label{fig:NuRe:5b}\includegraphics[width=\textwidth]{./figs/schrodinger/convergence}}
\end{minipage}\par\medskip
\centering
%\subfloat[]{\label{fig:NuRe:4c}\includegraphics[width=0.5\textwidth]{./figs/schrodinger/error}}

\caption{ (a) Convergence of the greedy method for the wave equation. (b) Convergence of the greedy method for the nonlinear Schr\"odinger equation equation. }
\label{fig:NuRe:5}
\end{figure}


\section{Conclusion} \label{chap:Con:1}

In this paper, we {\edit presented} a greedy approach for the construction of a reduced system that preserves the geometric structure of Hamiltonian systems. An iteration of the greedy method comprises searching the parameter space using the error in the Hamiltonian, to find the best basis vectors that increase the overall accuracy of the reduced basis. We argued that for a compact subset with exponentially small Kolmogorov $n$-width we recover exponentially fast convergence of the greedy algorithm. For fast approximation of nonlinear terms, the basis obtained by the greedy was combined with the symplectic DEIM to construct a Hamiltonian reduced system.


%Since the reduced system obtained by the greedy method is Hamiltonian, it preserves stability, and was illustrated in the numerical results.


The numerical results demonstrated that the greedy method can save substantial computational cost in the offline stage as compared to alternative SVD-based techniques. Also since the reduced system obtained by the greedy method is Hamiltonian, the greedy method yields a stable reduced system. Symplectic DEIM also effectively reduced computational cost of approximating nonlinear terms while preserving stability and symplectic structure. Hence, the greedy method is an efficient model reduction technique that provides an accurate and stable reduced system for large-scale parametric Hamiltonian systems.


 % it preserves stability, and was illustrated in the numerical results.
%We showed that the Hamiltonian is preserved in the 
%As the accuracy of the reduced system obtained by the greedy is comparable with the SVD-based methods, the greedy method is a promising dimension reduction technique for parametric Hamiltonian systems. 


\section{Symplectic QR (SQR) decomposition} \label{chap:Append:1}

Similar to the QR decomposition for constructing an orthogonal matrix, for any full rank matrix $M$ of size $2n\times 2n$ we may have a decomposition
\begin{equation}
	M = AR,
\end{equation}
where $A$ is a symplectic square matrix (and not necessarily orthogonal) and $R$ has the structure
\begin{equation}
	R =
	\begin{pmatrix}
		S & T \\
		U & V
	\end{pmatrix}
\end{equation}
where $S$, $T$, $U$ and $V$ are all upper triangular square matrices of size $n$. Also $T$ and $U$ have zeros along the diagonal. The algorithm for a SQR is shown in Algorithm \ref{alg:Append:1}.

\begin{algorithm} 
\caption{Symplectic QR decomposition (SQR)} \label{alg:Append:1}
{\bf Input:} Full rank matrix $M = [u_1,\dots , u_k,v_1,\dots,v_k]$ of size $2n \times 2k$.
\begin{enumerate}
\item $\alpha \leftarrow \Omega(u_1,v_1)$
\item $e_1 \leftarrow \text{sign}(\alpha)\cdot u_1 / \sqrt{|\alpha|} $
\item $f_1 \leftarrow v_1 / \sqrt{|\alpha|}$
\item \textbf{for} $i\leftarrow 2$ \textbf{to} $k$
\item \hspace{0.5cm} $q \leftarrow u_i$
\item \hspace{0.5cm} $p \leftarrow v_i$
\item \hspace{0.5cm} \textbf{for} $j\leftarrow1$ \textbf{to} $i-1$
\item \hspace{0.5cm} \hspace{0.5cm} $q \leftarrow q - \Omega(q,f_j)e_j + \Omega(q,e_j)f_j$
\item \hspace{0.5cm} \hspace{0.5cm} $p \leftarrow p - \Omega(p,f_j)e_j + \Omega(p,e_j)f_j$
\item \hspace{0.5cm} \textbf{end for}
\item \hspace{0.5cm} $\alpha \leftarrow \Omega(q,p)$
\item \hspace{0.5cm} $e_i \leftarrow \text{sign}(\alpha) \cdot q / \sqrt{|\alpha|} $
\item \hspace{0.5cm} $f_i \leftarrow p / \sqrt{|\alpha|}$
\item \textbf{end for}
\item $A \leftarrow [e_1, \dots , e_k , f_1, \dots , f_k]$
\item $R \leftarrow A^+M$
\end{enumerate}

\vspace{0.5cm}
{\bf Output:} Matrix decomposition $M = AR$.
\end{algorithm}



\bibliography{ref}
\bibliographystyle{plain}



\end{document}
%% end of file `docultex.tex'
