%%%%%%%%%%%%%%%%%%%% author.tex %%%%%%%%%%%%%%%%%%%%%%%%%%%%%%%%%%%
%
% sample root file for your "contribution" to a contributed volume
%
% Use this file as a template for your own input.
%
%%%%%%%%%%%%%%%% Springer %%%%%%%%%%%%%%%%%%%%%%%%%%%%%%%%%%


% RECOMMENDED %%%%%%%%%%%%%%%%%%%%%%%%%%%%%%%%%%%%%%%%%%%%%%%%%%%
\documentclass[graybox]{svmult}

% choose options for [] as required from the list
% in the Reference Guide

\usepackage{mathptmx}       % selects Times Roman as basic font
\usepackage{helvet}         % selects Helvetica as sans-serif font
\usepackage{courier}        % selects Courier as typewriter font
\usepackage{type1cm}        % activate if the above 3 fonts are
                            % not available on your system
%
\usepackage{makeidx}         % allows index generation
\usepackage{graphicx}        % standard LaTeX graphics tool
                             % when including figure files
\usepackage{subfigure}
\usepackage{epstopdf}

\usepackage{multicol}        % used for the two-column index
\usepackage[bottom]{footmisc}% places footnotes at page bottom
\usepackage{amsmath}
\usepackage{amssymb}
\usepackage{mathtools}
%\usepackage[linesnumbered, ruled, vlined]{algorithm2e}
%\usepackage{algorithm}% http://ctan.org/pkg/algorithms
%\usepackage{algorithmicx} 
%\usepackage{algpseudocode}% http://ctan.org/pkg/algorithmicx
%\renewcommand{\algorithmicrequire}{\textbf{Input:}}
%\renewcommand{\algorithmicensure}{\textbf{Output:}}
\usepackage{enumerate}
\usepackage{multirow}
\usepackage{cleveref}

\usepackage[hang, small,labelfont=bf,up,textfont=it,up]{caption} % Custom captions under/above floats in tables or figures
\usepackage{booktabs} % Horizontal rules in tables
\usepackage{float} % Required for tables and figures in the multi-column environment - they need to be placed in specific locations with the [H] (e.g. \begin{table}[H])
%\usepackage{hyperref} % For hyperlinks in the PDF

% see the list of further useful packages
% in the Reference Guide

\makeindex             % used for the subject index
                       % please use the style svind.ist with
                       % your makeindex program

%%%%%%%%%%%%%%%%%%%%%%%%%%%%%%%%%%%%%%%%%%%%%%%%%%%%%%%%%%%%%%%%%%%%%%%%%%%%%%%%%%%%%%%%%

\begin{document}

\title*{Model Order Redcution of Fluid Flow While Preserving Quadratic Invariants}
% Use \titlerunning{Short Title} for an abbreviated version of
% your contribution title if the original one is too long
\author{Bababk Maboudi Afkham, Nicol\`o Ripamonti, Qian Wang, Jan S. Hesthaven and Karen E. Willcox}
% Use \authorrunning{Short Title} for an abbreviated version of
% your contribution title if the original one is too long
\institute{Name of First Author \at Name, Address of Institute, \email{name@email.address}
\and Name of Second Author \at Name, Address of Institute \email{name@email.address}}
%
% Use the package "url.sty" to avoid
% problems with special characters
% used in your e-mail or web address
%
\maketitle

%\abstract*{Each chapter should be preceded by an abstract (10--15 lines long) that summarizes the content. The abstract will appear \textit{online} at \url{www.SpringerLink.com} and be available with unrestricted access. This allows unregistered users to read the abstract as a teaser for the complete chapter. As a general rule the abstracts will not appear in the printed version of your book unless it is the style of your particular book or that of the series to which your book belongs.
%Please use the 'starred' version of the new Springer \texttt{abstract} command for typesetting the text of the online abstracts (cf. source file of this chapter template \texttt{abstract}) and include them with the source files of your manuscript. Use the plain \texttt{abstract} command if the abstract is also to appear in the printed version of the book.}

\abstract{In the context of model order reduction, conservation of stability in the reduced order systems of partial differential equation, especially with a strong advection term, remains a challenge. Recently preserving structures, invariants and conservation laws, in order to help with stability, and ensure robustness in long-time integrations, has been an area of active research. Energy conservation is a fundamental feature of fluid flows which is often violated in the course of model reduction. In this paper we discuss a model reduction routine, that exploits skew-symmetry of conservative and centered discretization schemes, to recover conservation of energy at the level of reduced system. This results in an, overall, correct evolution of the fluid and ensures robustness of the reduced system. We evaluate the performance of the proposed method through numerical simulation of various fluid flows, and also in a numerical simulation of continuous variable resonance combustor model.}


% sections are added here

\section{Introduction}

To be added

\section{Model Order Reduction for Time Dependent Problems} \label{sec:mor}

Consider a dynamical system of the form
\begin{equation} \label{eq:1}
	\left\{
	\begin{aligned}
		\frac{d}{dt} u(t) &= f(t,u(t)),\\
		u(0) &= u_0.
	\end{aligned}
	\right.
\end{equation}
Here, $u(t),u_0\in \mathbb R^{n}$ and $f: [0,T]\times \mathbb R^{n} \to \mathbb R^{n}$, for some $T<\infty$, is a Lipschitz function. We may apply the \emph{method of lines} \cite{Edsberg:2008:ICM:1477735} to a system of partial differential equations to obtain a dynamical system of the form \eqref{eq:1}. The \emph{solution manifold} for \eqref{eq:1} is defined as
\begin{equation} \label{eq:2}
	\mathcal M_u := \{ u(t) | t \in [0,T] \}.
\end{equation}
When $\mathcal M_u$ has a low-dimensional representation, it is referred to as \emph{reducible}. Assume that $\mathcal M_u$ can be well approximated by $k$-dimensional linear subspace $\mathcal V_k$, with $k\ll n$ and let $E_k = \{ v_1,\dots,v_k \}$ be the basis vectors for $\mathcal V_k$ and $V_k$ the basis matrix that contains these vectors in its columns. A reduced basis (RB) method assumes that $u \approx \tilde u = V_k v$, where $v$ is the expansion coefficients of $\tilde u$ in the basis $E_k$. Substituting this into \eqref{eq:1} yields
\begin{equation} \label{eq:3}
	V_k \frac{d}{dt} v(t) = f(t,V_{k}v) + r(t,u).	
\end{equation}
Here, $r$ is the error vector in this approximation. The Petrov-Galerkin projection of \eqref{eq:1} onto $\mathcal V_k$ requires $r$ to be orthogonal to a $k$-dimensional subspace $\mathcal W_k$. One can construct a projection operator $P_{\mathcal V_{k},\mathcal W_k}$ that projects elements of $\mathbb R^{n}$ onto $\mathcal V_k$, orthogonal to $\mathcal W_k$ as $P_{\mathcal V_{k},\mathcal W_k} = V_k(W_k^TV_k)^{-1}W_k^T$, where $W_k$ is the basis matrix that contains the basis vectors of $\mathcal W_k$ in its columns and $W_k^TV_k$ is assumed to be invertible. With this projection, \eqref{eq:1} reduces to
\begin{equation} \label{eq:4}
	\left\{
	\begin{aligned}
		\frac{d}{dt} v(t) &= (W_k^TV_k)^{-1} f(t,V_{k}v),\\
		v(0) &= (W_k^TV_k)^{-1}u_0.
	\end{aligned}
	\right.
\end{equation}
When we require $W_k=V_k$, then \eqref{eq:4} is referred to as the \emph{Galerkin} projection of \eqref{eq:1} onto $\mathcal V_k$. Since \eqref{eq:4} has a smaller size, as compared to \eqref{eq:1}, one can expect accelerated evaluation. To numerically identify the best possible subspace $\mathcal V_{k}$ we first discretize the solution manifold to obtain
\begin{equation} \label{eq:5}
	\mathcal M_{u}^{\Delta} = \{ u(t_i) | i\in \{ 1,\dots,N_t \} \}.
\end{equation}
Members of $M_{u}^{\Delta}$ are referred to as \emph{snapshots} of \eqref{eq:1}. One can obtain these snapshots by applying a time-integration scheme, e.g. the Runge-Kutta methods, to \eqref{eq:1} to obtain $\tilde {\mathcal M}_{u}^{\Delta}$ an approximation to $\mathcal M_{u}^{\Delta}$. Throughout this paper, we assume that we can choose $\tilde{\mathcal M}_{u}^{\Delta}$ arbitrary close to $\mathcal M_{u}^{\Delta}$, therefore, by an abuse of notation, we may drop the overscript ``\textasciitilde''. For a Galerkin projection, the best possible basis $V_k$ is the one that minimizes the collective projection error \cite{hesthaven2015certified}, i.e., the solution to the minimization problem
\begin{equation} \label{eq:6}
\begin{aligned}
&  \underset{V_k\in\mathbb R^{n\times k}}{\text{minimize}}
& &  \| S - V_kV_k^TS\|_F, \\
& \text{subject to}
& & V_k^TV_k=I_k.
\end{aligned}
\end{equation}
Here $S$ collects vectors in $\mathcal M_{u}^{\Delta}$ in its columns, referred to as the \emph{snapshot matrix}, $\|\cdot \|_F$ is the Frobenius norm \cite{trefethen97}, and $I_{k}$ is the identity matrix of size $k$. Note that the constraint in \eqref{eq:6} requires $V_k$ to be orthonormal. The basis matrix $V_k$ that solves the minimization problem \eqref{eq:6} is referred to as the \emph{proper orthogonal decomposition} (POD) of $S$ of size $k$ \cite{hesthaven2015certified}, and, according to the Schmidt-Mirsky theorem, can be constructed using the left singular vectors of $S$ as
\begin{equation}
	V_{k} = [u_i]_{i=1}^{k}.
\end{equation}
Here $u_i$, for $i=1,\dots k$, are the first $k$ singular vectors of $S$.

Often MOR is studied in the parametric setting, where the vector $u,u_0$, and the right hand side $f$ of \eqref{eq:1} are of the form $u(t;\mu)$, $u_0(\mu)$, and $f(t,u;\mu)$, respectively. Here $\mu$ belongs to $\mathbb P$ a closed subset of $\mathbb R^d$. In this case, the reduced system can approximate the quantities of interest at an accelerated rate. Since the nature of time, as a parameter, is different from other spacial and physical parameters, in this paper we solely focus on $t$ as the parameter. Nevertheless, it is straight forward to extend the results of this paper to the parameter setting by using POD in time and parameter space, or by using the POD-greedy \cite{haasdonk2013convergence,hesthaven2015certified,quarteroni2015reduced} method to generate a basis $V_k$.

Since the approximated solution $\tilde u$ is a linear combination of the POD basis vectors, $\tilde u$ inherits linear properties of these basis vectors. However, when the solution $u$ to \eqref{eq:1} satisfies some nonlinear invariants, there is no guarantee that, in general, $\tilde u$ also satisfy such invariants \cite{doi:10.1137/140959602,doi:10.1137/140978922,doi:10.1137/17M1111991,MaboudiAfkham2018}. This results in a qualitatively wrong and often unstable solution. In the later sections, we discuss how the skew-symmetric formulation of the fluid flow allows conservation of quadratic invariants, e.g. the kinetic energy, at the level of the reduced system.

\section{Skew Symmetric and Centered Schemes for Fluid Flows} \label{sec:skew}

\subsection{Conservation Laws} \label{sec:skew.1}
Transport of conserved quantities, in the context of fluid flows, can be formulated as
\begin{equation} \label{eq:3.1}
	\frac{\partial }{\partial t} \rho \varphi + \nabla \cdot ( \rho u \varphi  ) = \nabla \cdot F_{\varphi}\quad \text{defined in} \quad \Omega \subset \mathbb R^{d}.
\end{equation}
Here, $d = 1,2$ or $3$, $\rho:\Omega\to \mathbb R$ is the density, $u\in \Omega \to \mathbb R^{d}$ is the velocity vector field, $\varphi$ is a measured scalar quantity of the flow, and $F_{\varphi}$ is the flux function associated to $\varphi$. Integration of \eqref{eq:3.1} over $ \Omega$ yields
\begin{equation} \label{eq:3.2}
	\frac{d}{dt} \int_{\Omega} \rho \varphi \ dx = \int_{\partial \Omega} (F_{\varphi} - \rho u \varphi) \cdot \hat n\ ds,
\end{equation}
where $\partial \Omega$ is the boundary of $\Omega$, and $\hat n$ is the unit outward normal vector to $\partial \Omega$. When the right hand side of \eqref{eq:3.2} vanishes, e.g. in case where $\Omega$ is periodic and $\varphi$ is flux-free, the quantity $(\rho \varphi)$ is conserved. Therefore, the form \eqref{eq:3.1} is referred to the \emph{conservative form} and the convective term in \eqref{eq:3.1} is referred to as the \emph{divergence form}. However, using the \emph{continuity equation}
\begin{equation} \label{eq:3.3}
	\frac{\partial }{\partial t} \rho + \nabla \cdot (\rho u) = 0,
\end{equation}
we can rewrite \eqref{eq:3.1} as
\begin{equation} \label{eq:3.4}
	\rho \frac{\partial }{\partial t} \varphi + (\rho u)\cdot \nabla \varphi = \nabla \cdot F_{\varphi}.
\end{equation}
The convective term in this formulation is referred to as the \emph{advective form}. The \emph{skew-symmetric} form for the convective term is obtained by the arithmetic average of the divergent and the advective form:
\begin{equation} \label{eq:3.5}
	\frac{1}{2} \left( \rho \frac{\partial }{\partial t} \varphi + \frac{\partial }{\partial t} (\rho \varphi) \right) + \frac 1 2 \left( (\rho u)\cdot \nabla \varphi + \nabla \cdot (\rho u \varphi) \right) = \nabla \cdot F_{\varphi}.
\end{equation}
Multiplying \eqref{eq:3.5} with $\varphi$ yields
\begin{equation}
	\frac{\partial }{\partial t} \rho \varphi^2 + \nabla \cdot ( \rho u \varphi^2  ) = \varphi \nabla \cdot F_{\varphi}.
\end{equation}
Therefore, $\varphi^2$ is a conserved quantity for a flux-free $\varphi$. Since the divergence, the advective and the skew-symmetric forms are identical at the continuous level, then $\varphi^2$ is a conserved quantity for all forms. However, equivalence of these forms is not preserved through a general discretization scheme and, therefore, we do not expect $\varphi^2$ to be a conserved quantity in discrete level. Assuming proper boundary conditions, the operator 
\begin{equation}
	S_{\rho u}(\cdot) = \frac 1 2 ( [ \nabla \cdot \rho u ] + (\rho u)\cdot \nabla )(\cdot),
\end{equation}
is a skew-adjoint operator with respect to the $L^2$ inner-product. Here, $[\cdot]$ indicates that the inside of brackets act as a differential operator. This skew-adjoint property is used in later sections to show the conservation of some quadratic quantities in \eqref{eq:3.1}. Similarly, we can define the skew-adjoint operator in time variable as
\begin{equation}
	S_{\rho,\partial_t} = \frac{1}{2} \left( \rho \frac{\partial}{\partial t} + [ \frac{\partial}{\partial t} \rho] \right).
\end{equation}
Here, the subscript $\partial t$ is to emphasize that $S_{\rho,\partial_t}$ is a differential operator with respect to $t$. Note that numerical time integration of \eqref{eq:3.5} can be challenging since the time differentiation of different variable is present. However, following derivations in \cite{morinishi2010skew}, one can rewrite \eqref{eq:3.5} as
\begin{equation} \label{eq:3.6}
	\sqrt{\rho } \frac{\partial }{\partial t} (\sqrt \rho \varphi ) + S_{\rho u}(\varphi) = \nabla \cdot F_{\varphi}.
\end{equation}
Time integration of this form is presented in the works of \cite{?}. Note that one can also generate a quasi-skew-symmetric form \cite{blaisdell1991numerical,morinishi2003dns} of \eqref{eq:3.1} as
\begin{equation} \label{eq:3.7}
	\frac{\partial }{\partial t} (\rho \varphi) + S_{\rho u}(\varphi) = \nabla \cdot F_{\varphi}.
\end{equation}
Even though this is not a fully skew-symmetric form (skew-symmetric only in space), the numerical stabilities of this form is significantly better than the divergence and advective form \cite{morinishi2010skew,blaisdell1991numerical,morinishi2003dns}. Note that the quasi-skew-symmetric form is identical to the skew-symmetric form in the incompressible limit.

\subsection{Incompressible Fluid} \label{sec:skew.2}
Consider the governing equations of an incompressible fluid with skew-symmetric convective term:
\begin{equation} \label{eq:3.8}
	\left\{
	\begin{aligned}
	&\nabla \cdot u = 0, \\
	&\frac{\partial}{\partial t} u + S_{u}(u) + \nabla p = \nabla \cdot \tau,
	\end{aligned}
	\right.
\end{equation}
defined on $\Omega$. Here, $p: \Omega \to \mathbb R^+$ is the pressure, $\tau: \Omega \to \mathbb R^{d\times d}$ is the viscous stress tensor, and $S_u = \frac 1 2 ([ \nabla \cdot u] + u\cdot \nabla)$. It is straight forward to check
\begin{equation} \label{eq:3.9}
	\frac{d}{dt} K + \nabla \cdot (Ku) + \nabla \cdot (pu)= \nabla \cdot (\tau u) - (\tau \nabla)\cdot u,
\end{equation}
where $K = \frac 1 2 \sum_{i=1}^d u_i^2 $ is the kinetic energy. Here we used 
\begin{equation} \label{eq:3.10}
	\begin{aligned}
	u\cdot \nabla p &= \nabla \cdot (up) - p\nabla \cdot u, \\
	u \cdot (\nabla \cdot \tau) &= \nabla \cdot (\tau u) - (\tau \nabla)\cdot u,
	\end{aligned} 
\end{equation}
and
\begin{equation} \label{eq:3.11}
	u\cdot S_{u}(u) = \nabla \cdot(Ku).
\end{equation}
The only non-conservative term in \eqref{eq:3.9} is $-(\tau \nabla)\cdot u$, which corresponds to dissipation of kinetic energy. Therefore, in the absence of the viscous terms, $K$ is a conserved quantity of the system, and $\frac d {dt} \int_{\Omega} K \ dx <0$ when $\tau\neq 0$. Note that as long as $\nabla \cdot u = 0$, as discussed in \Cref{sec:skew.1}, the divergence, the convective, and the skew-symmetric forms are identical for the incompressible fluid equation. Thus, kinetic energy is conserved for all forms. However, for a general discretization scheme, these forms are not identical, and often the conservation of kinetic energy, in the discrete sense, is violated.

The skew-symmetric discretization of \eqref{eq:3.8} is a centered scheme that exploits the skew-adjoint property of $S_u$, and ensures conservation of kinetic energy at the discrete level. We uniformly discretize $\Omega$ into $N$ points and denote by $\mathbf u \in \mathbb R^{N\times d}$, $\mathbf p \in \mathbb R^N$, and $T \in R^{N\times d\times d} $ the discrete representation of $u$, $p$, and $\tau$, respectively. Let $D_j$ be the centered finite difference matrix operator for $\partial / \partial x_j$, for $j = 1,\dots,d$. The momentum equation in \eqref{eq:3.8} is then discretized as
\begin{equation} \label{eq:3.13}
	\frac{d}{dt}{\mathbf u}_i + S_{\mathbf u} \mathbf u_i + D_i \mathbf p = \sum_{j=1}^d D_j T_{ij}, \quad i=1,\dots,d.
\end{equation}
where $S_{\mathbf u}$ is the discretization of $S_{u}$ given by
\begin{equation} \label{eq:3.14}
	S_{\mathbf u} = \sum_{j=1}^d D_j U_j + U_j D_j,
\end{equation}
and $U_i$ contains components of $\mathbf u_i$ on its diagonal. We require $D_i$ to satisfy
\begin{enumerate}
	\item $D_i = -D_i^T$
	\item $D_i \mathbf 1 = \mathbf 0$, where $\mathbf 1$ and $\mathbf 0$ are vectors of ones and zeros, respectively.
\end{enumerate}
Condition 1 and 2 yield
\begin{equation} \label{eq:3.15}
	S_{\mathbf u} = -S_{\mathbf u}^T, \quad \mathbf 1^T S_{\mathbf u} \mathbf u_i = 0, \quad i=1,\dots,d.
\end{equation}
Conservation of momentum in the discrete sense reads
\begin{equation} \label{eq:3.16}
	\frac{d}{dt} \sum_{i=1}^d  \mathbf 1^T \mathbf u_i = \sum_{i=1}^d \left( - \mathbf 1^T S_{\mathbf u} \mathbf u_i - \mathbf 1^T D_i \mathbf p + \sum_{j=1}^d \mathbf 1^T D_j T_{ij}  \right) = 0.
\end{equation}
Similarly, using \eqref{eq:3.10}, it is verified that
\begin{equation} \label{eq:3.17}
\frac{d}{dt} \sum_{i=1}^d \left( \frac 1 2 \mathbf u_i^T \mathbf u_i \right) = - \sum_{i,j=1}^d T_{ij}D_j \mathbf u_i \leq 0.
\end{equation}

Conditions 1 and 2 for $D_i$ can be easily checked for a centered finite differences scheme on a periodic domain. For other types of boundaries, e.g., wall boundary and inflow/outflow, we refer the reader to \cite{morinishi1998fully,desjardins2008high} for construction of the proper discrete centered differentiation operator. 

\subsection{Compressible Fluid} \label{sec:skew.3}
Consider the equation governing the evolution of a compressible fluid in a skew-symmetric form in one spacial dimension
\begin{equation} \label{eq:3.18}
\left\{
\begin{aligned}
	&\frac{\partial}{\partial t} \rho + \frac{\partial }{\partial x}(\rho u) = 0, \\ 
	& S_{\rho,\partial_t}(u)+ S_{\rho u}(u) + \frac{\partial }{\partial x} p = \frac{\partial }{\partial x} \tau, \\
	&\frac{\partial}{\partial t} \rho E + \frac{\partial}{\partial x}(u E + up) = \frac{\partial }{\partial x}(u\tau - \phi).
\end{aligned}
\right.
\end{equation}
Here $E= e + u^2/2$ is the total energy per unit mass, with $e = \frac{p}{\rho(\gamma - 1)}$ the internal energy, $\gamma$ the adiabatic gas index, and $\phi = -\lambda \frac{\partial T}{\partial x}$ is the heat flux, with $\lambda$ the heat conductivity. The rest of the variables are the same as those discussed in \Cref{sec:skew.2}. Following \cite{reiss2014conservative}, the evolution of the momentum equation reads
\begin{equation} \label{eq:3.19}
	\begin{aligned}
	\frac{\partial}{\partial t}(\frac{\rho u^2}{2}) + \frac{\partial }{\partial x}(\rho u \frac{u^2}{2}) &= \frac 1 2 u( \frac{d}{dt} \rho u + \rho \frac{d}{dt} u ) + \frac 1 2 u ( [ \frac{\partial}{\partial x} \rho u ] u + \rho u \frac{\partial }{\partial x} u ) \\
	& = - u \frac{\partial}{\partial x} p + u \frac{\partial}{\partial x} \tau.
	\end{aligned}
\end{equation}
Substituting this into the energy equation in \eqref{eq:3.18} while assuming a constant adiabatic index yields
\begin{equation} \label{eq:3.20}
	\frac{1}{\gamma -1} \frac{d}{dt} p + \frac{\gamma}{\gamma -1} \frac{\partial }{\partial x} up - u \frac{\partial }{\partial x}(p) = - u \frac{\partial}{\partial x} \tau + \frac{\partial }{\partial x}(u\tau - \phi).
\end{equation}

We discretize the real line, uniformly, into $N$ grid points and denote by $\mathbf r, \mathbf u, \mathbf p \in \mathbb R^{N}$, the discrete representations of $\rho$, $u$, and $p$, respectively. Using the matrix differentiation operator $D\in \mathbb R^{N\times N}$ (we omit the subscript ``$i$'' for the one dimensional case), introduced in \Cref{sec:skew.2}, we define the skew-symmetric matrix operator $S_{\mathbf r \mathbf u} = \frac 1 2 (DUR + RUD)$, where $R$ is the matrix that contains $r$ in its diagonal.Now, the semi-discretization of \eqref{eq:3.18} and \eqref{eq:3.20} takes now takes the form
\begin{equation} \label{eq:3.21}
\left\{
\begin{aligned}
	& \frac{d}{dt} \mathbf r + DU\mathbf r = 0, \\
	& S_{\mathbf r,\partial_t} (\mathbf u) + S_{\mathbf r \mathbf u} \mathbf u + D \mathbf p = D T, \\
	&\frac{1}{\gamma -1} \frac{d}{dt} \mathbf p + \frac{\gamma}{\gamma -1} D U \mathbf p - UD\mathbf p = - UDT + D(UT - \mathbf \phi).
\end{aligned}
\right.
\end{equation}

Considering conditions 1 and 2 for $D$, discussed in \Cref{sec:skew.2}, it is easily verified that
\begin{equation} \label{eq:3.22}
	S_{\mathbf r \mathbf u}^T = - S_{\mathbf r \mathbf u}, \quad \mathbf 1^T S_{\mathbf r \mathbf u} \mathbf u = - \mathbf u^T DU \mathbf r.
\end{equation}
Conservation of mass yields
\begin{equation} \label{eq:3.23}
	\frac{d}{dt} (\mathbf 1^T \mathbf r) = - \mathbf 1^T DR\mathbf u = 0. 
\end{equation}
Furthermore, we recover conservation of momentum in the discrete sense as
\begin{equation} \label{eq:3.24}
\begin{aligned}
	\frac{d}{dt}(\mathbf r^T \mathbf u) &= \frac{1}{2} \frac{d}{dt}(\mathbf r^T \mathbf u) + \frac{1}{2} \left( \mathbf r^T \frac d{dt} \mathbf u +\mathbf u^T \frac{d}{dt} \mathbf r \right)\\
	&= \frac{1}{2}u^T \frac d{dt} \mathbf r + \mathbf 1^T S_{\mathbf r,\partial_t} (\mathbf u) \\
	&= -\frac 1 2 \mathbf u^T DU \mathbf r  - \frac 1 2 \mathbf 1^T S_{\mathbf r \mathbf u} \mathbf u - \mathbf 1^T D\mathbf p +  \mathbf 1^T D T = 0.
\end{aligned}
\end{equation}
Here we used \eqref{eq:3.22} and the mass and the momentum equation in \eqref{eq:3.21}. Similarly, for the conservation of the total energy, we have
\begin{equation} \label{eq:3.25}
\begin{aligned}
	\frac{d}{dt} \left( \frac{1}{\gamma - 1} \mathbf 1^T \mathbf p + \frac 1 2 \mathbf (R\mathbf u)^T \mathbf u  \right) &= \frac{d}{dt} \left( \frac{1}{\gamma - 1} \mathbf 1^T \mathbf p \right) + \frac 1 2 \mathbf u^T  S_{\mathbf r,\partial_t} (\mathbf u) = 0.
\end{aligned}
\end{equation}
In addition to the conservation of the total energy, skew-symmetric form of \eqref{eq:3.21} also conserves the evolutions of the kinetic energy:
\begin{equation} \label{eq:3.26}
\begin{aligned}
	\frac{d}{dt} ( \frac 1 2 \mathbf u^T R\mathbf u) = \frac 1 2 \mathbf u^T S_{\mathbf r,\partial_t} (\mathbf u) &= -\mathbf u ^T S_{\mathbf r \mathbf u} \mathbf u + \mathbf u^T D p + \mathbf u^T DT \\
	&= \mathbf u^T D p + \mathbf u^T DT.
\end{aligned}
\end{equation}
Here, we used the skew-symmetry of $S_{\mathbf r \mathbf u}$. Therefore, only the pressure and viscous terms contribute to the change in the level of kinetic energy.

We point out that there are other methods to obtain a skew-symmetric form for \eqref{sec:skew.3}, that result in the conservation of other quantities. An entropy preserving skew-symmetric form can be found in \cite{sjogreen2010skew}. Furthermore, a fully quasi-skew-symmetric form for \eqref{sec:skew.3}, where all quadratic fluxes are in a skew-symmetric form, is shown to have lowest aliasing error \cite{honein2004higher,honein2005numerical}

\subsection{Time integration}
Following steps in \cite{reiss2014conservative,morinishi2010skew} we can construct a fully discrete scheme for \eqref{sec:skew.3} as
\begin{equation} \label{eq:3.27}
	\left\{
	\begin{aligned}
	&\frac 1 2 \sqrt{\mathbf r} ^{n+1/2} \frac{\sqrt{\mathbf r}^{n+1} - \sqrt{\mathbf r}^{n}}{\Delta t} + DU^{n+1/2} \mathbf r^{n} = 0, \\
	& \sqrt{\mathbf r} ^{n+1/2}  \frac{\sqrt{ \mathbf R}^{n+1} u^{n+1} - \sqrt{\mathbf R}^{n}u^n}{\Delta t} + S_{\mathbf r^{n} \mathbf u^n} \mathbf u^{n+1/2}_\alpha + D \mathbf p^{n} = DT^{n}, \\
	& \frac 1 {\gamma -1} \frac{\mathbf p^{n+1} - \mathbf p^n}{\Delta t} + \frac{\gamma}{\gamma -1} D U^{n} \mathbf p - U^{n} D \mathbf p = - U^{n}D T^{n} + D (U^nT^n - \phi^n).
	\end{aligned}
	\right.
\end{equation}
Here, $\Delta t$ is the time discretization unit, superscript $n$ denotes evaluating the variable at $t = n\Delta t$, superscript $n+1/2$ denotes the arithmetic average of the variable at $t=n\Delta t$, the square root sign denotes element-wise application of square root,  and $t=(n+1)\Delta t$ and 
\begin{equation}
	\mathbf u_{\alpha}^{n+1/2} = \frac{\sqrt{\mathbf R}^{n+1} \mathbf u^{n+1} + \sqrt{\mathbf R}^{n} \mathbf u^{n}}{2\sqrt{\mathbf r}^{n+1/2} }
\end{equation}
It is discussed in \cite{reiss2014conservative} that this time discretization scheme preserves the symmetries expressed in \eqref{eq:3.17}, \eqref{eq:3.24}, \eqref{eq:3.25}, and \eqref{eq:3.26}. In the incompressible case, the method reduces to the implicit mid-point scheme \cite{hairer2006geometric}. For further information see \cite{reiss2014conservative,morinishi2010skew}.

\section{Model Reduction of Fluid Flow} \label{sec:mor_skew}
Straight forward model reduction of \eqref{eq:3.8} and \eqref{eq:3.18} does not generally preserves symmetries and conservation laws that were presented in \Cref{sec:skew}. In this section we discuss how to exploit the discrete skew-symmetric structure of \eqref{eq:3.13} and \eqref{eq:3.21} to recover conservation of mass, momentum, and energy at the level of reduced system.

Let $V_{\mathbf r}$, $V_{\mathbf r \mathbf u}$, and $V_{\mathbf u_i}$ be the reduced bases for the snapshots of $\mathbf r$, $R \mathbf u$, and $\mathbf u$, respectively. For the purpose of simplicity, we assume that all bases have the size $k$. In case of an incompressible fluid, $V_{ \mathbf r}$ and $V_{\mathbf r \mathbf u}$ are omitted. We seek to project $S_{\mathbf u}$ and $S_{\mathbf r \mathbf u}$ onto the reduced space, such that the projection preserves the skew-symmetry property. The projected operators, using a Galerkin projection, read
\begin{equation} \label{eq:4.1}
	S^r _{\mathbf u} = V_{ \mathbf u_i}^T S _{\mathbf u} V_{ \mathbf u_i}, \quad i=1,\dots,d,
\end{equation}
and
\begin{equation} \label{eq:4.2}
	S^r_{\mathbf r ,\partial_t} =V_{\mathbf r \mathbf u}^T  S_{\mathbf r ,\partial_t} V_{\mathbf u}, \quad S^r _{\mathbf r \mathbf u} = V_{\mathbf r \mathbf u}^T  S _{\mathbf r \mathbf u} V_{\mathbf u}.
\end{equation}
Note that $S^r_{\mathbf r ,\partial_t}$ is not computed explicitly. It is seen that the $S^r _{\mathbf u}$ is already in a skew-symmetric form. On the other hand, $S^r_{\mathbf r ,\partial_t}$ and $S^r _{\mathbf r \mathbf u}$ are not, in general, skew-adjoint and skew-symmetric, respectively. This can be ensured, however, by requiring $V_{\mathbf r \mathbf u} = V_{\mathbf u}$. We denote such a basis by $V_{\mathbf r \mathbf u, \mathbf u}$.Using \eqref{eq:4.1} and \eqref{eq:4.2}, the Galerkin projection of momentum equation in \eqref{eq:3.13} and equations governing the compressible fluid in \eqref{eq:3.21} take the form
\begin{equation} \label{eq:4.3}
	\frac{d}{dt} {\mathbf u^r}_i + S^r_{\mathbf u} \mathbf u^r_i + V_{\mathbf u_i} ^T D_i \mathbf p = \sum_{j=1}^d V_{k_3, \mathbf u_i}^T D_j T_{ij}(V_{ \mathbf u_i} \mathbf u^r_i), \quad i=1,\dots,d,
\end{equation}
and
\begin{equation} \label{eq:4.4}
\left\{
\begin{aligned}
	& \frac{d}{dt} \mathbf r^r + \sum_{i=1}^k V^T_{\mathbf r}DU_iV_{\mathbf r}\mathbf r^r = 0, \\
	& S^r_{\mathbf r ,\partial_t} + S^r _{\mathbf r \mathbf u} \mathbf u^r + V_{\mathbf r \mathbf u, \mathbf u}^T D \mathbf V_{\mathbf p} \mathbf p^r = V_{\mathbf r \mathbf u, \mathbf u}^T D T, \\
	&\frac{1}{\gamma -1} \frac{d}{dt} \mathbf p^r + \frac{\gamma}{\gamma -1} \mathbf V_{\mathbf p}^T D U V_{\mathbf p} \mathbf p^r - \mathbf V_{\mathbf p}^T UD V_{\mathbf p} \mathbf p^r = - V_{\mathbf p}^T UDT + V_{\mathbf p}^T D(UT - \mathbf \phi).
\end{aligned}
\right.
\end{equation}
Note that in \eqref{eq:4.4}, dependency of $T$ on $V_{\mathbf r \mathbf u , \mathbf u}$ is not shown for abbreviation. In \eqref{eq:4.3} and \eqref{eq:4.4}, $D_i$ is always multiplied from left with a basis matrix or a diagonal matrix. Therefore, the telescoping sum discussed in condition 2 in \Cref{sec:skew.1} cannot be used to show conservation of mass and momentum. However, POD preserves linear properties of snapshots. To demonstrate, let the overscript ``\textasciitilde'' denote the representation of a reduced variable in the high fidelity space. A reduced variable, e.g. density, can be represented as a linear combination of some snapshots as $\mathbf r \approx \tilde{\mathbf r} = \sum_{i=1}^k c_i \mathbf r_i$, for some snapshots $\mathbf r_i$ and some coefficients $c_i \in \mathbb R$, for $i=1,\dots,k$. The conservation mass for the reduced system reads
\begin{equation} \label{eq:4.5}
	\frac{d}{dt} \mathbf 1^T \tilde {\mathbf r} = \sum_{i=1}^k c_i  \left( \mathbf 1^T \frac{d}{dt} \mathbf r_i \right) = - \sum_{i=1}^k c_i  \left( \mathbf 1^T DR_i\mathbf u_i \right) = 0.
\end{equation}
Here we used the fact that $\mathbf 1^T D = \mathbf 0^T$. Similarly, we recover conservation of momentum
\begin{equation} \label{eq:4.6}
\begin{aligned}
	\frac{d}{dt}(\tilde {\mathbf r}^T \tilde{\mathbf u}) &= \frac{1}{2} \frac{d}{dt}(\tilde{\mathbf r}^T \tilde{\mathbf u}) + \frac{1}{2} \left( \tilde{ \mathbf r }^T \frac d{dt} \tilde{ \mathbf u } + \tilde {\mathbf u}^T \frac{d}{dt} \tilde {\mathbf r} \right)\\
	&= \sum_{i,j=1}^k d_i c_j \left( \mathbf u_i^T \frac{d}{dt} \mathbf r_j + \left( \mathbf r_j^T \frac d{dt} \mathbf u_i +\mathbf u_i^T \frac{d}{dt} \mathbf r_j \right)\right) = 0.\\
\end{aligned}
\end{equation}
Here, $\tilde{\mathbf u} = \sum_{i=1}^k d_i \mathbf u_i$, for some snapshot $\mathbf u_i$ and coefficients $d_i \in \mathbb R$. Denoting by $\{ R \mathbf u\}^r$ the reduced representation of $R\mathbf u$ in basis $V_{\mathbf r \mathbf u , \mathbf u}$, the evolution of kinetic energy reads
\begin{equation} \label{eq:4.7}
	\begin{aligned}
	\frac{d}{dt}\left( \frac 1 2 \tilde{\mathbf u}^T \tilde{ R } \tilde {\mathbf u} \right) &= \frac{d}{dt}\left( \frac 1 2 {\mathbf u^r}^T V_{\mathbf r \mathbf u , \mathbf u}^T V_{\mathbf r \mathbf u , \mathbf u} \{ R \mathbf u\}^r \right) = \frac{d}{dt}\left( \frac 1 2 {\mathbf u^r}^T \{ R \mathbf u\}^r \right) \\
	&= \frac 1 2 \left( {\mathbf u^r}^T \frac{d}{dt} \{ R \mathbf u \}^r + \{ R \mathbf u \}^r \frac{d}{dt} \mathbf u^r \right) \\
	&= \frac 1 2 \left( {\mathbf u^r}^T V_{\mathbf r \mathbf u , \mathbf u}^T V_{\mathbf r \mathbf u , \mathbf u} \frac{d}{dt} \{ R \mathbf u \}^r + \{ R \mathbf u \}^r \frac{d}{dt} V_{\mathbf r \mathbf u , \mathbf u}^T V_{\mathbf r \mathbf u , \mathbf u} \mathbf u^r \right) \\
	&= {\mathbf u^r}^T S^r_{\mathbf r, \partial_t} \mathbf u^r = { \mathbf u^r }^T V_{\mathbf r \mathbf u , \mathbf u} D V_{\mathbf p} \mathbf P^r + { \mathbf u^r }^T V_{\mathbf r \mathbf u , \mathbf u}^T D T.
	\end{aligned}
\end{equation}
In the missing steps in the last line, the skew-symmetry of $S_{\mathbf r \mathbf u}^r$ is used. Note, that only the reduced pressure and the viscous term contribute to the evolution of kinetic energy. Furthermore, the quantity $ \frac 1 2 {\mathbf u^r}^T \{ R \mathbf u\}^r$ is the kinetic energy associated with the reduced system \eqref{eq:4.4} that approximates the kinetic energy of the high-fidelity system \eqref{eq:3.21} and is a quadratic form with respect to the reduced variables. Conservation of kinetic energy for \eqref{eq:4.3} follows similarly. With what discussed above, it is  straight forward to check that
\begin{equation} \label{eq:4.8}
	\frac{d}{dt} \left( \frac{1}{\gamma - 1} \mathbf 1^T \tilde{\mathbf p} + \frac{1}{2} \tilde{\mathbf u}^T \tilde R \tilde{\mathbf u} \right) = 0,
\end{equation}
i.e., the total energy is conserved. We can immediately recognize that $ \mathbf p^r /(\gamma - 1)$ is the total internal energy of the reduced system. However, the internal energy of the system is a weighted sum, $ b^T\mathbf p^r /(\gamma - 1)$, with $b = V_{\mathbf p}^T \mathbf 1$ which is an approximation of the internal energy of \eqref{eq:3.21}. From \eqref{eq:4.5}, \eqref{eq:4.6}, \eqref{eq:4.7}, and \eqref{eq:4.8} we conclude the following proposition.
\begin{proposition}
	The loss in the mass, momentum and energy associated with model reduction in \eqref{eq:4.4} is constant in time, and therefore, bounded.
\end{proposition}

\subsection{Assembling Nonlinear Terms and Time Integration}
Nonlinear terms that appear in \eqref{eq:4.3} and \eqref{eq:4.4} are of quadratic nature. These terms can be evaluated exactly using a set of precomputed matrices as proposed in \cite{Benner2018}. As an example, consider, for simplicity, in one dimension
\begin{equation}
	S_{\mathbf u}^r = V^T_{\mathbf u} ( DU + UD ) V^T_{\mathbf u}.
\end{equation}
We write $U$ as a linear combination of matrices as $U = \sum_{j=1}^k \mathbf u ^r_j U_j$, where $\mathbf u ^r_j$ is the $j$th component of $\mathbf u ^r$, and $U_j$ contains the $j$th column of $V_{\mathbf u}$ on its diagonal. It follows
\begin{equation}
	S_{\mathbf u}^r = \sum_{j=1}^k \mathbf u ^r_j \left( V^T_{\mathbf u} ( DU_j + U_jD ) V^T_{\mathbf u} \right),
\end{equation}
and the matrices $V^T_{\mathbf u} ( DU_j + U_jD ) V^T_{\mathbf u}$ can be computed prior to the time integration of the reduced system. However, the form of the fully discrete system in \eqref{eq:3.27} introduces cubic and even quartic terms. Although, in principle, the same method can be applied to assemble the nonlinear terms, however, the number of precomputed matrices can grow proportional to the order of nonlinear term. 

To accelerate the assembly of the nonlinear terms that require a large number of precomputed matrices, we may approximately evaluate them using the discrete empirical interpolation method (DEIM). This approximation can affect the accuracy of conserved quantities in \eqref{eq:4.4}. Therefore, the accuracy of DEIM approximation should be chosen considerably higher than the one from POD.

To integrate \eqref{eq:4.4} in time, the fully discrete system \eqref{eq:3.27} is modified prior to model reduction, by multiplying the mass and momentum equation with $\sqrt{\mathbf r}^{n+1}$. Note that since the new form is identical to \eqref{eq:3.27}, it does not affect the conserved quantities. Subsequently, a basis for $\sqrt{\mathbf r}$, denoted by $V_{\sqrt{\mathbf r}}$, is constructed. The nonlinear terms, are evaluated exactly using the quadratic expansion or approximated using the DEIM.



\section{Numerical Experiments} \label{sec:res}

\subsection{The incompressible Euler Equation}

\subsection{The Compressible Euler Equation}

\subsection{Continuous Variable Resonance Combustor}

CVRC is a model rocket combustor designed and operated at Purdue University (Indiana, U.S.) to investigate combustion instabilities \cite{yu2008combustion}. This setup is called the Continuously Variable Resonance Combustor (CVRC) because the length of the oxidizer injector can be varied continuously, allowing for a detailed investigation of the coupling between acoustics and combustion in the chamber \cite{garby2013simulations}. The 2D/3D high-fidelity simulations of CVRC are expensive. Thus to get a fast analysis tool, a quasi-1D model has been proposed by Smith et al. \cite{smith2008computational} and further developed by Frezzotti et al. \cite{frezzotti2015determination,frezzotti2017numerical,frezzotti2018quasi}. 


The CVRC consists of three parts: oxidizer post, combustion chamber and exit nozzle, as shown in Fig. \ref{fig:radius}. The oxidizer is injected from the left end of the oxidizer post and meets the fuel that is injected through an annular ring around the oxidizer injector, at the back-step. The combustion happens in a region around the back-step. The combustion products flow through the chamber and exit the system from the nozzle.  Both the injector and the nozzle are operated at choked condition during the experiment. The length of the oxidizer post $L_{op}$ of the CVRC can be varied continuously, leading to different behavior of the combustion stability. In this paper, we will focus on the case with $L_{op}= 14.0$ cm, in which the combustion is unstable.

The geometry parameters of the quasi-1D CVRC with a oxidizer post length  $L_{op}= 14.0$ cm are shown in Table \ref{tab:geometry_parameters}.  The back-step and the converging part of the nozzle are sinusoidally contoured to avoid discontinuity of the radius that will invalidate the quasi-1D governing equations presented in the next subsection. 

\begin{figure}
	\centering
	\includegraphics[width=0.8\linewidth]{pic/radius}
	\caption{Geometry of quasi-1D CVRC model.}
	\label{fig:radius}
\end{figure}

\begin{table} [h]
	\centering
	\caption{Geometry parameters of the quasi-1D CVRC with an oxidizer post length $L_{op}=14$ cm.}
	\centering
	\begin{tabular}{c c c c c c }
		\toprule
		\centering
		\multirow{2}{*}{Section} &
		\multicolumn{2}{c}{Oxidizer post} &
		\multirow{2}{*}{Chamber} &
		\multicolumn{2}{c}{Nozzle} \\
		\cmidrule(lr){2-3} \cmidrule(lr){5-6}
		& injector & back-step & & converging part & diverging part\\
		\midrule
		Length (cm) & 12.99 & 1.01 & 38.1 & 1.27 & 3.4 \\
		Radius (cm) & 1.02  & $1.02 \sim 2.25$ & 2.25 & $2.25 \sim 1.04$ & $1.04 \sim 1.95$ \\
		\bottomrule
		\label{tab:geometry_parameters}
	\end{tabular} 
\end{table}

The fuel is pure gaseous methane. The oxidizer is a mixture of 42\% oxygen and 58\% water (per unit mass). The oxidizer is injected in the oxidizer post at a temperature $T_{ox}=1030$ K so that both water and oxygen are in the gaseous phase. The operating conditions are listed in Table \ref{tab:operating-conditions}.

\begin{table} [h]
	\centering
	\caption{CVRC operating conditions.}
	\centering
	\begin{tabular}{l l l}
		\toprule
		\centering
		Parameter & Unit & Value \\
		\midrule
		Fuel mass flow rate, $\dot{m}_{f}$ & kg/s & 0.027   \\
		Fuel temperature, $T_{f}$ & K & 300   \\
		Oxidizer mass flow rate, $\dot{m}_{ox}$ & kg/s & 0.32   \\
		Oxidizer temperature, $T_{ox}$ & K & 1030   \\
		$O_2$ mass fraction in oxidizer, $Y_{O_2}$ & -- & 42.4\%   \\
		$H_2O$ mass fraction in oxidizer, $Y_{H_2O}$ & -- & 57.6\%   \\
		Mean chamber pressure & MPa & 1.34 \\
		Equivalence ratio, $E_r$ & -- & 0.8 \\
		\bottomrule
		\label{tab:operating-conditions}
	\end{tabular} 
\end{table}

For the combustion, we consider the one-step reaction model
\begin{equation*}\label{eq:combustion}
CH_4 + 2O_2 \rightarrow CO_2 + 2H_2O
\end{equation*}
We assume that the fuel reacts instantaneously to form products, allowing us to neglect intermediate species and finite reaction rates. As the equivalence ratio is less than one, there is oxidizer left after the combustion. Therefore, only two species need to be considered: oxidizer and combustion products.


The governing equations that describe the conservation of mass, momentum, and energy of the quasi-1D CVRC flow, are the quasi-1D unsteady Euler equations for multiple species, expressed in conservative form as
\begin{equation*}\label{eq:quasi-1D-equation}
\frac{\partial \mathbf{u}}{\partial t} + \frac{\partial \mathbf{f}}{\partial x} = \mathbf{s}_A + \mathbf{s}_f + \mathbf{s}_q.
\end{equation*}
The conserved variable vector $\mathbf{u}$ and the convective flux vector $\mathbf{f}$ are
\begin{equation*}\label{eq:CVRC-u-f}
\mathbf{u}= \left( \begin{gathered}
\rho A  \\
\rho uA  \\
\rho EA  \\
\rho Y_{ox} A \\
\end{gathered} \right), 
\mathbf{f} = \left( \begin{gathered}
\rho uA  \\
\left(\rho u^2 + p\right)A  \\
\left(\rho E + p\right)uA  \\
\rho uY_{ox} A \\
\end{gathered} \right),
\end{equation*}
where $\rho$ is the density, $u$ is the velocity, $p$ is the pressure, $E$ is the total energy, $Y_{ox}$ is the mass fraction of oxidizer, and $A=A(x)$ is the cross section area of the duct. The pressure $p$ can be computed using the conserved variables as
\begin{equation*}\label{eq:total-engery}
E = \frac{p}{\rho (\gamma - 1)} + \frac{u^2}{2} - C_p T_{ref},
\end{equation*}
where $T_{ref}$ is the reference temperature and is set as 298.15 K in this paper. The temperature $T$ is recovered from the equation of state $p = \rho R T$. The gas properties  $C_p$, $R$ and $\gamma$ are computed as $C_p= \sum C_{pi}Y_i$, $R=\sum R_iY_i$ and $ \gamma= C_p/(C_p-R)$, respectively. 

The source terms are
\begin{equation*}\label{eq:source-terms}
\mathbf{s}_A = \left( \begin{gathered}
0  \\
p \frac{dA}{dx}  \\
0  \\
0 \\
\end{gathered} \right), 
\mathbf{s}_f = \left( \begin{gathered}
{\dot \omega}_f  \\
{\dot \omega}_f u  \\
{\dot \omega}_f \left(h_{0}^{f} + \Delta h_{0}^{rel} \right)  \\
{\dot \omega}_{ox} \\
\end{gathered} \right), 
{\mathbf{s}_q} = \left( \begin{gathered}
0  \\
0  \\
q'  \\
0 \\
\end{gathered} \right),
\end{equation*}
where $\dot{\omega}_f$ is the depletion rate of the fuel, $\dot{\omega}_{ox}$ is the depletion rate of the oxidizer, $h_0^f$ is the total enthalpy of the fuel, $\Delta h_{0}^{rel}$ is the heat of reaction per unit mass of fuel and $q'$ is the unsteady heat release term. $\mathbf{s}_A$ accounts for area variations, $\mathbf{s}_f$ and $\mathbf{s}_q$ are related to the combustion. $\mathbf{s}_f$ represents the addition of the fuel and its combustion with the oxidizer, which in turn results in the creation of the combustion products. The depletion rate of the fuel is
\begin{equation}\label{eq:wf}
\dot{\omega}_{f}= \frac{k_f \dot{m}_f Y_{ox} \left(1+sin\xi\right)}{l_f-l_s},
\end{equation} 
where
\begin{equation*}\label{eq:xi}
\xi= -\frac{\pi}{2} + 2\pi\frac{x-l_s}{l_f-l_s}, \hspace{0.5cm} \forall \hspace{0.2cm} l_s < x < l_f.
\end{equation*}
The setting of the fuel injection restricts the combustion to the region $l_s < x < l_f$. The reaction constant $k_f$ is selected to insure that the fuel is consumed within the specified combustion zone. The depletion rate of the oxidizer is computed by 
\begin{equation*}\label{eq:wox}
\dot{\omega}_{ox} = C_{o/f} \dot{\omega}_f,
\end{equation*}
where $C_{o/f}$ is the oxidizer-to-fuel ratio.

The unsteady heat release term $q'$, also called the combustion response function, models the coupling between acoustics and combustion. In this paper, we use the combustion response function designed by Frezzotti et al. \cite{frezzotti2017numerical,frezzotti2018quasi}, which is a function of the velocity, sampled at specific abscissa $\hat{x}$ that is almost coincident with the antinode of the first longitudinal modal shape, with a certain time lag $\tau$, i.e.,
\begin{equation}\label{eq:response-function}
q'\left( x,t\right) = \alpha g\left(x\right)  A\left(x\right) \left[ u\left( \hat{x},t-\tau \right) - \bar{u}\left( \hat{x} \right) \right].
\end{equation}
Here $\bar{u}$ is the time averaged velocity, estimated with the steady-state quasi-1D model assuming $q'=0$, and $g(x)$ is a Gaussian distribution  
\begin{equation*}\label{eq:gx}
g\left(x\right)= \frac{e^{-\frac{\left(x-\mu\right)^2}{2\sigma^2}}}{\sqrt{2\pi\sigma^2}},
\end{equation*}
where $\mu$ is the mean and $\sigma$ is the standard deviation. The amount of heat release due to velocity oscillations is controlled by the parameter $\alpha$.

The boundary conditions for the quasi-1D CVRC flow include the fixed mass flow rate and the stagnation temperature at the head-end of the oxidizer injector, and the supersonic outflow at the exit of the nozzle.

Prior to unsteady simulation, the quasi-1D CVRC needs to be excited, which can be achieved by adding a perturbation to the steady-state solution. The perturbation is added by forcing the mass flow rate with a multi-sine signal
\begin{equation}\label{eq:multisine}
\dot{m}_{ox} \left(t\right)= \dot{m}_{ox,0} \left[1 + \delta\sum_{k=1}^{K}  sin\left(2\pi k\Delta f t\right) \right],
\end{equation}
where $\dot{m}_{ox,0}$ is the oxidizer mass flow rate in Table \ref{tab:operating-conditions}, $\Delta f$ is the frequency resolution and $K$ is the number of frequencies. In this paper, $\Delta f = 50 $ Hz and $K=140$, resulting in a minimal frequency of 50 Hz and a maximal frequency of 7000 Hz. $\delta$ is required to be small to control the amplitude of the perturbation and is set as 0.1\%.

The procedure of the unsteady simulation of the quasi-1D CVRC flow includes three steps:
\begin{enumerate}[(1)]
	\item  Compute the steady-state solution by setting $\dot{m}_{ox}=\dot{m}_{ox,0} $ and $q'=0$.
	\item  Excite the system by adding a perturbation to the oxidizer mass flow rate according to (\ref{eq:multisine}) and setting $q'=0$.
	\item  Perform the unsteady simulation by turning on the combustion response function $q'$ in (\ref{eq:response-function}) and turning off the oxidizer mass flow rate perturbation by setting $\dot{m}_{ox}=\dot{m}_{ox,0} $.	
\end{enumerate}


A high-fidelity quasi-1D CVRC flow solver is built by employing the MUSCL reconstruction \cite{delis1998tvd}, the Lax-Friedrichs flux \cite{shu1988efficient} and the strong stability preserving, three-stage Runge-Kutta (SSP RK3) time stepping \cite{jiang1996efficient}.

\input{./sections/6.con.tex}

\bibliographystyle{plain}
\bibliography{references}

\end{document}
